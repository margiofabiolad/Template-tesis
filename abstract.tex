\chapter*{}

\begin{center}
\Large\textbf{Sincronización en un modelo de actividad neuronal basado en mapas caóticos sometidos a perturbaciones}
\end{center}

\begin{center}
\large\insertauthor
\end{center}

\begin{center}
Propuesta de Proyecto de Grado \\ 
\insertdepartment\\
\insertfaculty ,\insertinstitution 
\end{center}


\textbf{Resumen: }Uno de los fenómenos que se observa en las redes neuronales es el de sincronización que ocurre cuando las neuronas ajustan su ritmo y su frecuencia entre sí. Clínicamente este fenómeno está relacionado con enfermedades neurológicas como la epilepsia y el párkinson. Por medio de un modelo neuronal basado en un mapa caótico propuesto por Chialvo, se recrea el fenómeno de sincronización neuronal en redes plásticas. Se determina la posibilidad de mantener la red de neuronas desincronizada, con la aplicación de perturbaciones externas a los elementos de la red siguiendo cuatro estrategias de selección espacial y temporal. Para todas las estrategias se construye el diagrama de fases en el espacio de los parámetros de control.

\textbf{Palabras Clave: } Sincronización, Neuronas, Control , Redes , Epilepsia


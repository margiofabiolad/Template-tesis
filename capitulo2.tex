\chapter{Marco teórico}

\section*{Preliminares}

{\textit{Hola}}
  
\section{Fundamentls de la Transferencia Radiativa}

    A primera vista podemos considerar un {\textit{campo de radiación}} como líneas rectas viajando por el espacio libre, o un medio homogéneo, en todas las direcciones. Estas líneas pueden considerarse ``rayos" partiendo del hecho de que la escala del sistema considerada es mucho mayor que la longitud de onda de la radiación, y de esta manera la teoría de transferencia puede surgir. Para tener una idea clara de la teoría y la {\textit{ecuación de transferencia radiativa}} que la gobierna, es necesario definir algunas cantidades fundamentales como la intensidad específica, flujo de energía, densidad de energía, presión de radiación, y algunos procesos como la emisión y la absorción que afectan a la intensidad para que esta deje de ser constante.
	
	Para estudiar el campo de radiación se requiere la cantidad de energía radiante $dE_{\nu}$ en un intervalo de frecuencia específica $(\nu, \nu + d\nu)$ la cual es transportada a través de un elemento de área $d\sigma$ y en direcciones confinadas a un elemento de ángulo sólido $d\omega$ durante un tiempo dt (ver fig. \ref{fig:2.1}).
    Así:

        \begin{equation} \label{eq:2.1}
            \boxed{dE_{\nu} = I_{\nu} \cos \theta d\nu d\sigma  d\omega dt,}
        \end{equation}
	 
	donde $I_{\nu}$ es la {\textit{intensidad específica o brillo}} con unidades en el sistema cgs de 

        $$I_{\nu} \rightarrow{\mathrm{erg \, Hz^{-1} cm^{-2}  ster^{-1} s^{-1} }},$$

    siendo $\theta$ el ángulo que forma el haz de radiación con la normal que apunta hacia afuera.
    
    \begin{wrapfigure}[14]{r}{0.5\linewidth} 
            \includegraphics[width=0.7\textwidth]{fig1_1_intensidad.png}
            \caption{{\small{Diagrama esquemático que muestra cómo es definida la intensidad específica. (Referencia: Radiative Transfer, Chandrasekhar S., pp 1.)}}}
        \label{fig:2.1} 
    \end{wrapfigure}

    En un medio que disperse, absorba o emita radiación, la intensidad $I_{\nu}$ puede variar, de manera que es una función que depende del punto y de la dirección además del tiempo. Esto significa que

        $$I_{\nu}  \equiv  I_{\nu}(\bf{r}; \bf{\Omega};t),$$

    con $\bf{r}$ el radio vector y $\bf{\Omega}$ la dirección. Así, en coordenadas cartesianas 
    
        $$I_{\nu} \equiv I_{\nu}(x,y,z; l,m,n;t,),$$

    donde $x,y,z$ y $l,m,n$ definen el punto y la dirección, respectivamente. 

    En astrofísica, un caso de gan interés es el de una atmósfera estratificada en capas de planos paralelos:

        $$I_{\nu} = I_{\nu}(z,\theta, \varphi;t),$$
        
    con $z$ la altura medida en dirección normal al plano de estratificación y $\theta, \varphi$ el ángulo polar y azimutal respectivamente.
    
    Otro caso interesante es el de simetría esférica 
    
        $$I_{\nu} = I_{\nu}(r,\theta;t),$$
        
    donde $r$ es el radio de la esfera y $\theta$ el ángulo que forma el haz de radiación con el radiovector.
	   
	Se dice que un campo de radiación es isotrópico si la intensidad $I_{\nu}$ es independiente de la dirección en ese punto: $I_{\nu} = I_{\nu}(r;t)$. 
	Además, es isotrópico y homogéneo si $I_{\nu}$ es la misma en todos los puntos además de en todas las direcciones: $I_{\nu} = I_{\nu}(t)$. 
	   
	 Por último, la intensidad $I_{\nu}$ integrada
	 sobre todas las frecuencias es denotada por $I$ y se conoce como la {\textit{intensidad integrada}}:
	   
	   \begin{equation} \label{eq:1.2}
	       I = \int_0^{\infty}{I_{\nu} d\nu.}
	   \end{equation}
	   
	   
\subsection{Flujo Neto y Flujo de Momento}

    La ec. (\ref{eq:2.1}) describe la energía en el intervalo de frecuencia $(\nu, \nu + d\nu)$ que fluye a través de un elemento de área $d\sigma$ en una dirección que está inclinada un ángulo $\theta$ y confinado a un elemento de ángulo sólido $d\omega$. Por su parte, el {\textit{Flujo}} $F_{\nu}$ es definido como la cantidad de energía radiante transferida a través de un elemento de área en un tiempo $dt$ y en un intervalo de frecuencia. De la ec. (\ref{eq:2.1}) 
    
        \begin{equation} \label{eq:2.3} 
            dF_{\nu} = \frac{dE_{\nu}}{d\nu d\sigma dt} = I_{\nu} \cos \theta d\omega \hspace{0.5cm} \rightarrow{\mathrm{erg \, Hz^{-1} \, cm^{-2} \,s^{-1}}}.
        \end{equation}
 
    Así, el flujo neto en la dirección $\bf{n}$ es obtenido por integrar sobre todos los ángulos sólidos
 
        \begin{equation} \label{eq:2.4}
            F_{\nu} = \int{I_{\nu} \cos \theta d\omega}.
        \end{equation}
        
    Si la intensidad se encuentra en un campo de radiación isotrópico (no es una función del ángulo) el flujo neto es cero ya que

        $$\int{\cos \theta} = 0,$$

    esto significa que hay tanta energía cruzando $d\sigma$ en la dirección $\bf{n}$ como en la dirección $\bf{-n}$. La dependencia del flujo en la dirección muestra que el flujo es de caracter vectorial. 
    
    \vspace{0.5cm}

{\bf{\large{Flujo de Momento}}}


    \vspace{0.4cm}
 
    El momento de un fotón $\gamma$ viene dado por $E/c$, con $c$ la velocidad de la luz $c = 3 \times 10^{10} \mathrm{cm \, s^{-1}}$. De esta manera, el flujo de momento a lo largo de un rayo es
    
        $$dF_{\nu}/c.$$

    Para obtener la componente del flujo de momento normal a $d\sigma$ multiplicamos por otro factor $\cos \theta$. Integrando
    
        \begin{equation} \label{eq:2.5}
            p_{\nu} = \frac{1}{c}\int{I_{\nu} \cos^2 \theta d\omega} \hspace{0.5cm} \rightarrow{\mathrm{dy \, cm^{-2} Hz^{-1}}}.
        \end{equation}

    El flujo neto $F_{\nu}$ y el flujo de momento $p_{\nu}$ son considerados el primer momento de la distribución angular y el segundo momento, de la intensidad $I_{\nu}$, respectivamente. Ambos son multiplicaciones por potencias del $\cos \theta$ e integración sobre $d\omega$. Integrando sobre todas las frecuencias, obtenemos el flujo total (integrado):
    
        \begin{align} \label{eq:2.6}
            F & = \int{F_{\nu} d\nu} \hspace{2cm} \textrm{y} \\
            p & = \int{p_{\nu}d\nu}. \label{eq:2.7}
        \end{align}

    
\subsection{Densidad de Energía Radiativa $u$}
    
    La cantidad conocida como la {\textit{densidad de energía específica}} $u_{\nu}$ es definida como la energía por unidad de volumen por unidad de rango de frecuencia. Para determinar $u_{\nu}$ es conveniente primero determinar la densidad de energía por unidad de ángulo sólido $d\omega$. 

        \begin{equation} \label{eq:2.8} 
            dE_{\nu} = u_{\nu}(\omega)dVd\omega d\nu, 
        \end{equation}

    con $dV$ el elemento de volumen, $dV=d\sigma ds$. Para un cilindro que rodea un rayo de longitud $ct$ (ver fig \ref{fig:2.2})., el volumen viene expresado como $dV=d\sigma cdt$. Así mismo, la radiación viaja a la velocidad $c$ de manera que en un tiempo $dt$ toda la radiación habrá salido del cilindro. Se tiene entonces
    
    
        \begin{equation} \label{eq:2.9} 
            dE_{\nu} = I_{\nu}  d\nu d\sigma dt d\omega,
        \end{equation}
        
    e igualando con ec.(\ref{eq:2.8})
 
        \begin{equation} \label{eq:2.10} 
            dE_{\nu} = u_{\nu}(\omega)(dAcdt)d\omega d\nu =  I_{\nu}  d\nu d\sigma dt d\omega. 
        \end{equation} 
        
      
        
        EDITAR
        
         \begin{wrapfigure}[13]{r}{0.5\linewidth} 
            \begin{center}                \includegraphics[width=0.6\textwidth]{fig2_2densidad_energia_cilindro.png}
                \caption{{\small{Volumen para un cilindro. (Referencia: Radiative Processes in Astrophysics, Rybicki, G., pp 5.)}}}
                \label{fig:2.2} 
            \end{center}
        \end{wrapfigure}

    Así, la contribución a la energía por unidad de volumen por unidad en el rango de frecuencia $(\nu, \nu + d\nu)$ que viene del ángulo sólido $d\omega$ alrededor de la dirección $\Omega$ es

        \begin{equation} \label{eq:2.11} 
            u_{\nu}(\omega) = \frac{I_{\nu}}{c},
        \end{equation}

    que integrado sobre todos los ángulos sólidos es 

        \begin{equation} \label{eq:2.12} 
            u_{\nu}= \int{u_{\nu}(\omega)d\omega} = \frac{1}{c} \int{I_{\nu} d\omega}.
        \end{equation}
    
    
    Por otra parte, es necesario definir lo que se conoce como la {\textit{intensidad media o intensidad promedio}} $j_{\nu}$

        \begin{equation} \label{eq:2.13} 
            \boxed{j_{\nu}= \frac{1}{4\pi} \int{I_{\nu}d\omega}.}
        \end{equation}

    Finalmente, la {\textit{densidad de energía total}}  es obtenida (junto con sus unidades) por integrar $u_{\nu}$ sobre todas las frecuencias 

        \begin{equation} \label{eq:2.14} 
            u=\int{u_{\nu}d\nu} =  \frac{4\pi}{c} \int{ j_{\nu} d\nu} \hspace{0.4cm} \rightarrow{\mathrm{erg \, cm^{-3}}}.
        \end{equation}
        
\subsection{Presión de Radiación}

    Considerando un cercado reflejante que contiene un campo de radiación isotrópico, cada fotón transfiere el doble de su componente normal de momento en la reflexión, así, de ec. (\ref{eq:2.5})

        \begin{equation} \label{eq:2.15} 
            p_{\nu} = \frac{2}{c}\int{I_{\nu} \cos^2 \theta d\omega}.
        \end{equation}

    Por isotropía $I_{\nu} = j_{\nu}$, entonces de ec. (\ref{eq:2.13})

        \begin{equation} \label{eq:2.16} 
            p_{\nu} = \frac{2}{c}\int{I_{\nu} \cos^2 \theta d\omega} = \frac{2}{c}\int{j_{\nu} \cos^2 \theta d\omega},
        \end{equation}

        $$p_{\nu}  = \frac{2}{c}\int{j_{\nu} d\nu} \int{\cos^2 \theta d\omega},$$
    
    e integrando sobre todas las frecuencias y la parte angular, la presión de radiación es

        \begin{equation} \label{eq:2.17} 
            \boxed{p = \frac{1}{3} u.}
        \end{equation}


\subsection{Conservación de la Enerdía Radiativa}

    Se refieren en la constancia de la intensidad de un haz. Vamos a considerar un emisor situado en un punto 1, y un receptor en el punto 2, entonces la cantidad de energía que sale de 1 y llega a 2 es 

        $$dE{\nu_1}= I_{\nu1} dA_1 d\omega_1 dt d\nu_1,$$

    con $d\omega_1$ el ángulo sólido que abarca el  receptor 2 visto desde  1.  Igualmente, la cantidad de energía que recibe 2 es 
    
        $$dE{\nu_2}= I_{\nu2} dA_ d\omega_2 dt d\nu_2.$$
        
        
        
        \begin{wrapfigure}[18]{r}{0.5\linewidth}
                \begin{center}
                    \includegraphics[width=0.5\textwidth]{fig2_3area1_area2.png}                    \caption{\footnotesize{Constancia de la Intensidad a lo largo de los rayos.}}
            \label{fig:2.3}
            \end{center}
        \end{wrapfigure}
        
    Por conservación de la energía $dE{\nu_1}= dE{\nu_2} $. Esto significa que la cantidad de energía emitida dentro del cono de apertura $d\omega_1$ coincide con la captada en el área $dA_2$ y la recibida dentro del cono de apertura $d\omega_2$ coincide con la emitida dentro del área $dA_1$ (ver fig. \ref{fig:2.3}). De esta manera 
    
        $$dA_1d\omega_1 = dA_2d\omega_2 \hspace{0.4cm}\rightarrow{dI_{\nu1} = I_{\nu2}},  $$ 

    Finalmente, $I_{\nu}=cte$, o en su forma diferencial 

        \begin{equation} \label{eq:2.18} 
            \boxed{\frac{dI_{\nu}}{ds} = 0.}
        \end{equation}
    
    Significa que la intensidad es constante a lo largo de la trayectoria en ausencia de emisores o absorbedor o fuentes de radiación. 

\subsection{Transferencia Radiativa}

    La intensidad específica $I_{\nu}$ no se mantendrá constante cuando un rayo atraviese la materia y sufra procesos de emisión, absorción o dispersión ({\textit{scattering}}).

    \vspace{0.4cm} 

        \begin{enumerate}
            \item {\bf{\large{Absorción:}} coeficiente de absorción} $\alpha_{\nu}$
  
                Un “lápiz de radiación” que atraviesa un medio se verá debilitado por su interacción con la materia.   Así, si la intensidad específica se convierte en $I_{\nu} + dI_{\nu}$ después de haber atravesado un espesor $ds$ en la dirección de su propagación se tiene 

    
                    \begin{equation} \label{eq:2.19} 
                        \boxed{dI_{\nu} = - \kappa_{\nu} \rho I_{\nu} ds,}
                    \end{equation}   
        
        
                siendo $\rho$ la densidad del material y $\kappa_{\nu}$, en unidades de ${\mathrm{cm^{2}\, g^{-1}}}$, el {\textit{coeficiente de absorción de masa}} para la radiación de frecuencia $\nu$. También recibe el nombre de {\textit{coeficiente de opacidad}} y en ocasiones es definido como $\alpha_{\nu} = \kappa_{\nu} \rho$, con $\alpha_{\nu}$ en unidades de ${\mathrm{cm^{-1}}}$. 
                \medskip
  
                No debe ser asumido que la reducción en la intensidad que experimenta un lápiz de radiación al atravesar la materia se pierde necesariamente en el campo de radiación.  Puede ser que una parte de la energía perdida del lápiz incidente pueda reaparecer en otras direcciones como radiación dispersada y que otra parte haya sido “verdaderamente” absorbida en el sentido que representa la transformación de la radiación en otras formas de energía (incluso en radiación de otras frecuencias). 
        
                \vspace{0.6cm}

            \item {\bf{\large{Emisión:}} coeficiente de emisión} $j$
    
                El {\textit{coeficiente de emisión espontánea}} $j$ es definida como la energía emitida por unidad de tiempo por unidad de ángulo sólido y por unidad de volumen 
    
                \begin{equation} \label{eq:2.20}
                    dE=jdVd\omega dt,
                \end{equation}
                
                
                Un {\textit{coeficiente de emisión monocromático}} es igualmente definido como 
    
                \begin{equation} \label{eq:2.21}
                    dE_{\nu}=j_{\nu}dVd\omega dt d\nu,
                \end{equation}
    
            donde $j_{\nu}$ tiene unidades de $\mathrm{erg \, cm^{-3} \, ster^{-1} s^{-1} Hz^{-1}}$. El coeficiente de emisión depende de la dirección en la cual la emisión toma lugar. Para un emisor isotrópico o para una distribución de emisores aleatoriamente orientados 
            
            
                \begin{equation} \label{eq:2.22}
                    j_{\nu} = \frac{1}{4\pi } P_{\nu},
                \end{equation}
    
            con $P_{\nu}$ la potencia radiada por unidad de volumen por unidad de frecuencia. Algunas veces la emisión espontánea es definida como la {\textit{emisividad}} $\epsilon_{\nu}$  (ángulo integrado) definida como energía emitida espontáneamente por unidad de frecuencia por unidad de tiempo por unidad de masa en unidades de $\mathrm{erg \,Hz^{-1} s^{-1} g^{-1}}$. Si la emisión es isotrópica 
            
                \begin{equation} \label{eq:2.23}
                    dE_{\nu}=\epsilon_{\nu} \, \rho \,dVdt d\nu \frac{d\omega }{4\pi},
                \end{equation}

            donde $\rho$ es la masa del medio emisor y $\frac{d\omega }{4\pi}$ considera la fracción de energía radiada en $d\omega$. Igualando la ec. (\ref{eq:2.21}) con (\ref{eq:2.22}) obtenemos la relación entre la emisividad $\epsilon_{\nu}$  y el coeficiente de emisión
            espontánea $j_{\nu}$
    
                \begin{equation} \label{eq:2.24}
                    \boxed{j_{\nu} =   \frac{\epsilon_{\nu}\rho }{4\pi}.}
                \end{equation}
    
            Al recorrer una distancia $ds$, un haz de sección transversal $d\sigma$ viaja a través de un volumen $dV=d\sigma ds$. Así, la intensidad agregada al haz por emisión espontánea es
    
            \begin{equation} \label{eq:2.25}
                dI_{\nu} = j_{\nu} ds.
            \end{equation}
    
    
        
        \end{enumerate}
        
        
\subsection{Ecuación de Transferencia Radiativa}

    La conservación de la energía en un haz que se propaga desde el punto 1 al punto 2 implica  $I_{\nu1} = I_{\nu2}$, lo cual es generalizada a través de la ec. (\ref{eq:2.18}). Ahora, al considerar la interacción de la radiación con el medio, deja de cumplirse la conservación de energía radiativa y se tiene la {\textit{ecuación de transferencia radiativa}},  la cual incluye los efectos de absorción  y emisión (una única ecuación que da la variación de la intensidad a lo largo de un rayo. Así, de las ecs. (\ref{eq:2.19}) y (\ref{eq:2.25}) se obtiene

        \begin{equation} \label{eq:2.26} 
            \boxed{\frac{dI_{\nu}}{ds} = -\alpha_{\nu} I_{\nu} + j_{\nu},}  
        \end{equation}
 
    la cual proporciona una gran herramienta para resolver la intensidad en un medio que emite o absorbe. Incorpora además aspectos macroscópicos de la radiación que los relaciona con el coeficiente de absorción y emisión.
    
    Conocidos los coeficientes  $\alpha_{\nu}$ y $ j_{\nu}$  es relativamente sencillo resolver la ec. (\ref{eq:2.26}). Vamos a considerar el caso de ``solo absorción'' y de ``solo emisión''. 
    
    \vspace{0.5cm}
 
        \begin{enumerate}
            \item {\bf{\large{Solo absorción:}}} $ j_{\nu}=0$
     
                Siendo $\alpha_{\nu} = \kappa_{\nu} \rho$,
     
                    \begin{equation} \label{eq:2.27}
                        \frac{dI_{\nu}}{ds} = -\alpha_{\nu}I_{\nu}.
                    \end{equation}

     
                Con solución:
            
                    \begin{equation} \label{eq:2.28}
                        I_{\nu}(s) = I_{\nu}(s_0) e^{-\int_{s_0}^s{\alpha_{\nu}(s')ds'}}. 
                    \end{equation}
     
                Así, el brillo decrece a lo largo del rayo por la exponencial del coeficiente de absorción integrado a lo largo de la línea de visión.
            
            
                \vspace{0.5cm}
     
            \item {\bf{\large{Solo emisión:}}} $ \alpha_{\nu}=0$.
    
                Por lo tanto:
    
                    \begin{equation} \label{eq:2.29}
                        \frac{dI_{\nu}}{ds} =  j_{\nu}. 
                    \end{equation}
                    
                    
                Con solución 
      
                    \begin{equation} \label{eq:2.30}
                        I_{\nu}(s) = I_{\nu}(s_0) + \int_{s_0}^s{j_{\nu}(s')ds'}.
                    \end{equation}

                El incremento de brillo es igual al coeficiente de emisión integrado a lo largo de la línea de visión. $s$ es el camino ??
    
        \end{enumerate}
    
\subsection{Profundidad Óptica y Función Fuente}

    La ecuación de transferencia (\ref{eq:2.26})  puede tomar una forma más simple si en vez de usar $s$ se utiliza una cantidad conocida como {\textit{profundidad óptica}}, la cual introduce menos incertidumbres en estimar distancias y tamaños de objetos astrofísicos. Se define como $d\tau_{\nu} = \alpha_{\nu} ds$ o

        \begin{equation} \label{eq:2.31}
            \tau_{\nu} = \int_{s_0}^s{\alpha_{\nu}(s') ds'}.
        \end{equation}
    
    
     Para:
     
        \begin{itemize}
            \item[i.] $\tau_{\nu} < 1$: el medio se dice que es {\textit{ópticamente delgado o transparente}}. Es decir, es un medio donde el fotón típico de frecuencia $\nu$ puede atravesar el medio sin ser absorbido. 
         
            \item[ii.] $\tau_{\nu} > 1$: el medio se dice que es {\textit{ópticamente grueso u opaco}}. El fotón promedio de frecuencia $\nu$ no puede atravesar un medio entero sin ser absorbido. 

         
        \end{itemize}
        
        
    La ecuación de transferencia (\ref{eq:2.26}) puede ser escrita (dividiendo por $\alpha_{\nu}$) como
     
        \begin{equation} \label{eq:2.32}
            \boxed{\frac{dI_{\nu}}{d\tau_{\nu}} = -I_{\nu} + S_{\nu},}
        \end{equation}

    donde $S_{\nu}$ es conocida como la {\textit{función fuente}} definida como la relación del coeficiente de emisión al de absorción 

        \begin{equation} \label{eq:2.33}
            S_{\nu} \equiv \frac{j_{\nu}}{\alpha_{\nu}} \hspace{0.5cm} \textrm{emisión/absorción}.
        \end{equation}
    
    Ya que la función fuente $S_{\nu}$ es a veces una cantidad física más simple que el coeficiente de emisión $j_{\nu}$ y la escala de profundidad óptica  revela  más claramente los intervalos importantes a lo largo de un rayo para el caso de la radiación, se suele sustituir $j_{\nu}$ y $\alpha_{\nu}$ por $\tau_{\nu}$ y $S_{\nu}$. Así mismo, podemos resolver formalmente la ecuación de transferencia si todas las cantidades en función de $s$ pasan a ser función de $\tau_{\nu}$. 
    
     
    Multiplicando la ecuación por el factor integrante $e^{\tau{\nu}}$ y definiendo las cantidades $X \equiv I_{\nu}e^{\tau_{\nu}} $ y $Y \equiv  S_{\nu}e^{\tau_{\nu}} $, se tiene de (\ref{eq:2.32}) 
    
        \begin{equation} \label{eq:2.34}
            \frac{X}{d\tau_{\nu}} = Y. 
        \end{equation} 
        
    Con solución: 
           
        \begin{equation} \label{eq:2.35}
            X(\tau_{\nu}) = X(0) + \int_{0}^{\tau_{\nu}}{Y(\tau_{\nu}')d\tau_{\nu}}. 
        \end{equation}
        
        
    Escribiendo la solución formal en términos de $I_{\nu}$ y $S_{\nu}$:

        \begin{equation} \label{eq:2.36}
            I_{\nu}(\tau_{\nu}) =  I_{\nu}(0)e^{-\tau_{\nu}} + \int_0^{\tau_{\nu}}{e^{-(\tau_{\nu} - \tau_{\nu}')} S_{\nu}(\tau_{\nu}')d\tau_{\nu}'}.
        \end{equation}
     
    Ya que el factor $e^{-\tau_{\nu}}$ es solo un factor sin dimensiones que escala la absorción, la ec. (\ref{eq:2.35}) puede ser interpetada como: la intensidad inicial que disminuye por absorción más la fuente integrada que disminuye también por absorción. 
    
    
\subsection{Camino Libre Medio y Fuerza de Radiación}

    El {\textit{camino libre medio}} de un fotón es una alternativa para describir la absorción. Se define como la distancia promedio que un fotón puede viajar a través de un material absorbente  sin ser absorbido. Se relaciona con el coeficiente de absorción $\alpha_{\nu}$ de un material homogéneo. De la ley de absorción ec. (\ref{eq:2.28}) la probabilidad de que un fotón viaje al menos una profundidad óptica $\tau_{\nu}$ 
    es simplemente $e^{-\tau_{\nu}}$. La profundidad óptica media  recorrida es entonces la unidad

        \begin{equation} \label{eq:2.37}
            \langle \tau_{\nu} \rangle = \int_0^{\infty}{\tau_{\nu}e^{-\tau_{\nu}}d\tau_{\nu}}=1. 
        \end{equation}
        
        
    Así, la distancia física media recorrida en un medio homogéneo es definida como el camino libre medio $l_{\nu}$ y es determinada por 
        $$
            \langle \tau_{\nu} \rangle  = \alpha_{\nu} l_{\nu} = 1,
        $$
        
        o
        
        \begin{equation} \label{eq:2.38}
            l_{\nu}  = \frac{1}{\alpha_{\nu} }.
        \end{equation}
        
    El camino libre medio es el recíproco del coeficiente de absorción para un material homogéneo. 
    
    
    \vspace{0.5cm}
 
    {\bf{\large{Fuerza de Radiación}}}
 
 
        En un medio que absorbe radiación esta ejerce una fuerza sobre el medio ya que la radiación propaga momento. El {\textit{vector de flujo de radiación}} se define como 
        
            \begin{equation} \label{eq:2.39} 
                \vec{F_{\nu}} = \int{\vec{I_{\nu}} \; \vec{n} \, d\omega}, 
            \end{equation}

        con $\vec{n}$ el vector unitario a lo largo de la dirección del rayo. Como el momento de un fotón es $E/c$, el vector momento por unidad de área por unidad de tiempo por unidad de longitud de camino absorbido por el medio es

            \begin{equation} \label{eq:2.40} 
                \vec{\zeta} = \frac{1}{c} \int{\alpha_{\nu} \vec{F_{\nu}} d\nu}.
            \end{equation}
            
            
        Y como $dV = d\sigma ds$, el vector momento es la fuerza  por unidad de volumen impartida en el medio por el campo de radiación. 

        La fuerza por unidad de masa del material es $\vec{f}=\vec{\zeta} / \rho$  o 

            \begin{equation} \label{eq:2.41}
                \vec{f}= \frac{1}{c} \int{\vec{\kappa_{\nu}} \;  \vec{F_{\nu}} d\nu}.
            \end{equation}

        En estas dos últimas ecuaciones, asumen que el coeficiente de absorción es isotrópico y no hay momento impartido por la emisión de radiación como es cierto para la emisión isotrópica. 
        
        
\section{Cosmología: Marco Teórico}


  \subsection{Principio Cosmológico} \label{sec:2.2.1}
  
    La isotropía de la Radiación del Fondo Cósmico de Microondas (CMB) nos dice que a primera aproximación el Universo es isotrópico y homogéneo actualmente. Esto último fue asumido en 1917 por Einstein en su modelo estático, que sería el primer modelo del Universo autoconsistente (Einstein, 1917). También, los modelos de Friedman tiempo después se convertirían en los modelos estándar para las dinámicas a gran escala del Universo.
    
    En 1923, Hermann Weyl introdujo su ``Postulado de Weyl'' (Weyl, 1923) que establecía la noción de geodésicas divergentes, las cuales representan las líneas de mundo de galaxias, que no se intersectan excepto en un {\textit{punto singular}} del pasado finito o infinito. Según esto, existe solo una geodésica pasando a través de cada punto en el espacio - tiempo, excepto en el origen.  De aquí es posible asignar un observador a cada línea de mundo. A esto se le conoce como {\textit{Observadores fundamentales}}, donde se supone que cada uno de ellos lleva consigo un reloj estándar y el tiempo medido en este, desde el punto singular, recibe el nombre de {\textit{tiempo cósmico}}.
    
    
    En 1935, Robertson y Walker (Robertson, 1935; Walker, 1936) derivaron la métrica del espacio - tiempo para todos los modelos de Universos isotrópicos, homogéneos y en expansión uniforme, la cual era independiente de la suposición de que las dinámicas a gran escala eran descritas por la Teoría de la Relativdad General, i.e., no importaba cual fuera la física de la expansión,  la forma de la métrica era la de Robertson - Walker, esto partiendo del supuesto de isotropía y homogeneidad del Universo. 
    
    Otra consideración para construir un modelo cosmológico es el conocido {\bf{principio cosmológico}} con el cual nos atrevemos a afirmar que: ``no estamos ubicados en ningún lugar especial en el Universo'', no hay preferencia, por lo que cualquier observador fundamental ubicado en cualquier lugar del Universo pero en la misma época cósmica observa las mismas características a grande escala que nosotros, i.e., la misma expansión de Hubble de la distribución de galaxias, la misma radiación del CMB isotrópica, la misma estructura esponjosa a gran escala de la distribución de galaxias y vacíos (red cósmica), y así sucesivamente. 
    En un sistema de galaxias que se expanden, cada observador en cada galaxia individual observa el mismo flujo de Hubble en la misma época, así, todos tienen derecho a creer que son el centro de un Universo en expansión.
    
     
        
    En cuanto a la geometría, a finales del siglo XVIII se estaban considerando los espacios no euclidianos. Los padres de la geomtría no euclidiana fueron Nikolai Ivanovich Lobachevsky en Rusia y János Bolyai en Transilvania (Lobachevsky, 1829; Bolyai, 1830). Lobachevsky resolvió el problema de la existencia de geometrías no euclidianas, así la misma se puso sobre una base teórica sólida por los estudios de Bernhard Riemann.
    
    \newpage
    
    \begin{wrapfigure}[16]{r}{0.4\linewidth}                                     \includegraphics[width=0.45\textwidth]{superficie_esferica_pp152.png}
                \caption{\footnotesize{Suma de los ángulos de un triángulo en la superficie de una esfera}}
                \label{fig:2.4} 
        \end{wrapfigure}
    
    En la figura (\ref{fig:2.4}) se tiene el caso más simple de la geometría curvada en 2 dimensiones, la superficie de una esfera: un triángulo con dos líneas (con $90^{\circ}$ entre ellas) que parten del polo norte y llegan hasta el ecuador, y una tercera línea que se dibuja sobre el ecuador. Estas líneas son la distancia más corta entre las tres esquinas del triángulo. En geometría curva son geodésicas. Si el radio de una esfera es $R_c$, el área superficial del triángulo ABC es $A= \theta R_c$. Variando el ángulo se obtiene para $\theta = 90^{\circ} \rightarrow{A = \pi R_c^2/2}$ donde la suma de los ángulos del triángulo es $270º$; y si $\theta=0º$ entonces el área también es cero y la suma de los ángulos $180º$. Así, la diferencia de la suma de los ángulos de $180^{\circ}$ es proporcional al área del triángulo:
        
        \vspace{0.4cm}
        
            $$(\textrm{suma de los ángulos del triángulo} - 180^{\circ} \propto \text{área del triángulo},$$
            
        la cual es una propiedad general de los espacios curvos isotrópicos. 
        
 \subsection{La métrica Espacio - Tiempo para Espacios Isotrópicos Curvos} \label{sec:2.2.2} 

    La distancia entre dos puntos separados por $dx, dy, dz$ en un espacio plano viene representado por 
    
        \begin{equation} \label{eq:2.2.1} 
            dl^2 = dx^2 + dy^2 + dz^2. 
        \end{equation}
    
    Siendo el caso más simple de un espacio curvo bidimensional isotrópico la superficie de una esfera, es conveniente utilizar un sistema coordenado polar para describir posiciones en la superficie (ver fig. \ref{fig:2.4}) que para el caso, las coordenadas ortogonales son $\theta$ y $\phi$ y el aumento de la distancia $dl$ entre dos puntos en dicha superficie es:
    
        \begin{equation} \label{eq:2.2.2} 
            dl^2 = R_c^2 d\theta^2 + R_c^2 \sin^2 \theta d\phi^2,
        \end{equation}
        
    donde $R_c$ es el radio de curvatura de la superficie del espacio bidimensional, i.e., el radio de la esfera. La ec. (\ref{eq:2.2.2}) es la {\textit{métrica de la superficie bidimensional}} y puede generalizarse como:
    
        \begin{equation} \label{eq:2.2.3}
            dl^2 = g_{\mu \nu} dx^{\mu} dx^{\nu}, 
        \end{equation}
    
    siendo $g_{\mu \nu}$ el {\textit{tensor métrico}} contenedor de toda la información acerca de la geometría intrínseca del espacio. Otros sistemas de coordenadas pueden definir las coordenadas de un punto en cualquier superficie bidimensional. Para un plano euclidiano:
    
        \begin{equation} \label{eq:2.2.4}
            dl^2 = dx^2 +  dy^2,
        \end{equation}

    en polares

        \begin{equation} \label{eq:2.2.5}
            dl^2 = dr^2 + r^2 d\phi^2.
        \end{equation}
        
    Gauss mostró que con el tensor métrico $g_{\mu \nu}$ es posible determinar la curvatura intrínseca del espacio. Para tensores que pueden ser reducidos a la forma diagonal (ecs. \ref{eq:2.2.2}, \ref{eq:2.2.4}, \ref{eq:2.2.5}) la curvatura intrínseca viene dada por:
    
         \begin{align*}
        \kappa & = \frac{1}{2 g_{11} g_{22}} \left\{ - \frac{\partial^2g_{11} }{\partial x_2^2} - \frac{\partial^2g_{22} }{\partial x_1^2} + \frac{1}{2 g_{11}} \left[ \frac{\partial g_{11} }{\partial x_1} \frac{\partial g_{22} }{\partial x_1} +   \left( \frac{\partial g_{11} }{\partial x_2} \right)^2 \right] 
        + \frac{1}{2 g_{22}} \left[ \frac{\partial g_{11} }{\partial x_2} \frac{\partial g_{22} }{\partial x_2} + \left( \frac{\partial g_{22} }{\partial x_1} \right)^2 \right] \right\}. 
        \end{align*}
    
    $\kappa$ se conoce como la {\textit{curvatura gaussiana}} del bi-espacio. La extención a tres espacios isotrópicos es sencilla si mantenemos que cualquier sección bidimensional a través de un espacio tridimensional isotrópico debe ser un bi-espacio isotrópico y ya el tensor métrico par este caso es conocido. 
    
    \newpage
    
    
        \begin{wrapfigure}[16]{r}{0.4\linewidth}                                             \includegraphics[width=0.5\textwidth]{superficie_esferica_pp156_angulos.png}
            \caption{\footnotesize{$\varrho$ es la distancia radial alrededor de la esfera desde el polo y el ángulo $\phi$ mide los desplazamientos angulares en el polo.}}
            \label{fig:2.5} 
        \end{wrapfigure}
    
    Fue mencionado que para un bi-espacio isotrópico, las coordenadas adecuadas son las polares esféricas. De la figura (\ref{fig:2.5}), la distancia alrededor del arco de un gran círculo desde el punto $O$ hasta $P$ es $R_c \theta$, por lo que la métrica se escribe:
    
    
        \begin{equation} \label{eq:2.2.6} 
            dl^2 = d\varrho^2 + R_c^2 \sin^2\left(\frac{\varrho}{R_c} \right) d\phi^2,
        \end{equation}

    donde $\varrho$ es la distancia más corta entre $0-P$  en la superficie de una esfera ya que es parte de un gran  círculo. De esta manera, estamos hablando de una {\textit{distancia geodésica}} en el espacio curvo isotrópico. Recordemos que en espacios curvos las geodésicas juegan el rol de las líneas rectas. 
    
    Ahora, (\ref{eq:2.2.6}) puede ser reescrita como: 

        \begin{equation} \label{eq:2.2.7} 
                x = R_c \sin \left(\frac{\varrho}{R_c} \right).
        \end{equation}
    
    Hallando $dx^2$ con $$d\varrho^2 = \frac{dx^2}{1-\kappa x^2},$$
    la métrica puede ser escrita como: 
    
        \begin{equation} \label{eq:2.2.8} 
            dl^2 = \frac{dx^2}{1-\kappa x^2} + x^2d\phi^2.
        \end{equation}
    
    Recordando que $\kappa = 1/R_c^2$, tres casos deben ser considerados:
    
        \begin{equation}
            \label{eq:2.2.9}
            \kappa  = \left\{
	            \begin{array}{ll}
		            > 0      & {\text{espacio esférico}},, \\
		            = 0 & R_c \rightarrow{\infty}, \hspace{0.4cm}  {\text{espacio plano}}, \\
		            < 0     & {\text{espacio hiperbólico}},
	            \end{array}
	       \right.
        \end{equation}
    
        donde $\kappa = 0$ recupera el espacio euclidiano. Para el incremento espacial en un espacio tridimensional curvo recordemos que cualquier sección bidimensional a través de un tri-espacio isotrópico debe ser un espacio isotrópico de dos dimensiones donde la métrica puede ser ec. (\ref{eq:2.2.6}) o (\ref{eq:2.2.8}). En coordenadas polares esféricas, el desplazamiento angular general perpendicular a la dirección radial es 
        
            \begin{equation} \label{eq:2.2.10}
                d\phi^2 = d\theta^2 +  \sin^2 \theta d\phi^2.
            \end{equation}
            
        (Con $\theta$ y $\phi$ diferentes a los usados en la fig. \ref{fig:2.5}). De esta manera, se puede escribir el incremento espacial de (\ref{eq:2.2.6}) o (\ref{eq:2.2.8}) como sigue
        
            \begin{align}
                dl^2 & = d\varrho^2 + R_c^2 \sin^2\left(\frac{\varrho}{R_c} \right) [d\theta^2 +  \sin^2 \theta d\phi^2], \label{eq:2.2.11} \\
                dl^2 &  =  \frac{dx^2}{1-\kappa x^2} + x^2[d\theta^2 +  \sin^2 \theta d\phi^2] \label{eq:2.2.12} .
            \end{align}
            
        De esta manera, la {\textit{métrica de Minkowski}} para un tri-espacio isotrópico viene escrita como: 

            \begin{equation} \label{2.2.13} 
                \boxed{ds^2 = dt^2 - \frac{1}{c^2} dl^2,}
            \end{equation}

        con $dl$ el de las expresiones (\ref{eq:eq:2.2.11} o \ref{eq:2.2.12}). Aunque $x$ y $\varrho$ son medidas de distancias equivalente, su significado físico es bastante distinto. Es ahora posible derivar la {\textit{métrica de Robertson -  Walker}}.
        
 \subsection{Métrica de Robertson - Walker} \label{sec:2.2.3}
 
    Queremos aplicar la métrica de Minkowski, ec. (\ref{2.2.13}), a modelos de mundo homogéneos e isotrópicos. Para eso, debemos recurrir a:

        \begin{enumerate}
            \item Principio cosmológico: A primera aproximación, el Universo es isotrópico y homogéneo en la época actual.
            \item Observadores fundamentales: quienes se mueven de modo tal que el Universo siempre parece isotrópico para ellos. 
            \item Tiempo cósmico: cada uno de estos observadores lleva un reloj y el tiempo propio medido por ese reloj es lo que se conoce como {\textit{tiempo cósmico}}.
        \end{enumerate}
        
    Recordemos que según el postulado de Weyl, las geodésicas de todos los observadores se reunen en un punto en el pasado y el tiempo cósmico puede ser medido desde esa época de referencia. 
    
    Ahora bien, considerando las ecs. (\ref{eq:eq:2.2.11}) y (\ref{eq:2.2.13}), podemos escribir la métrica como:
    
        \begin{equation} \label{eq:2.2.14}
            \boxed{ds^2 = dt^2 - \frac{1}{c^2}( d\varrho^2 + R_c^2 \sin^2\left(\frac{\varrho}{R_c} \right) (d\theta^2 +  \sin^2 \theta d\phi^2)],}
        \end{equation}

    siendo $t$ el tiempo cósmico y $d\varrho$ un incremento de la distancia propia (la cual se ve afectada por la expansión del Universo) en la dirección radial. 
    
    Para un Universo en expansión, esta métrica presenta problemas. Ya que la luz viaja una velocidad finita, observamos todos los objetos astronómicos a lo largo de un cono de luz anterior centrado en la Tierra en la época actual $t_0$ (ver fig. \ref{fig:2.6b}). Así, cuando observamos objetos distantes, no los observamos en la época actual sino en un tiempo anterior $t_1$, cuando el Universo todavía era homogéneo e isotrópico, pero las distancias entre los observadores fundamentales eran menores y la curvatura espacial diferente. Significa entonces que la métrica (\ref{eq:2.2.14}) solo puede ser aplicada a un espacio curvo, isotrópico definido en una única época. 
    
    {\large{\textbf{Cono de Luz}}}
    
    Como una digresión, es importante aclarar acerca del cono de luz y su importancia dentro del marco de la relatividad especial y del Universo. 
    
    El espacio - tiempo consiste de tres elementos que son representados en un diagrama conocido como el {\textit{cono de luz}} (ver fig. \ref{fig:2.6a}). Estos son: 
    
        \begin{itemize}
            \item Puntos que se encuentran en el espacio - tiempo y los cuales representan los {\textit{eventos}}, i.e., algo que ocurre en algún lugar, en algún momento. 
            \item Líneas, que representan las {\textit{líneas de mundo}}, los observadores que se extienden desde el pasado hacia el futuro. Un observador recibe señales de luz que vienen del pasado y transmite señales de luz que salen al futuro. 
            \item Geodésicas nulas que no son otra cosa que los {\textit{rayos de luz}}, que describen líneas a $45^{\circ}$ de la vertical. Para ellas, $ds^2 =0$. 
        \end{itemize}
        
        Las líneas de mundo con un ángulo menor a $45^{\circ}$ de la vertical se conocen como {\textit{líneas temporaloides}} ({\textit{time-like}}), y los eventos están causalmente conectados. Así mismo, las que tienen un ángulo mayor a $45^{\circ}$ con respecto a la vertical son llamadas {\textit{líneas espacialoides}} ({\textit{space-like}}). Eventos sobre estas líneas están  causalmente desconectados; no se pueden afectar mutuamente en sus respectivos conos de luz pasado y futuro ya que están separadas por una distancia mayor que la que la luz puede recorrer en el tiempo que los separa.  
        
        Para medir una distancia propia adecuada que pueda ser incluida en la métrica (\ref{eq:2.2.14})  se pueden alinear un conjunto de observadores fundamentales que estén entre la Tierra y la galaxia. Cada uno de estos observadores va a medir la distancia $d \varrho$ a su próximo observador en un tiempo cósmico particular. Al sumar todos estos $d \varrho$ se puede obtener una distancia adecuada que es medida en una sola época y  que sería posible incluir en la métrica (\ref{eq:2.2.14}). Lo que nosotros observamos son galaxias distantes en una época pasada, i.e., cómo eran en una época anterior, y no se sabe cómo proyectar sus posiciones relativas a nosotros en la época actual hasta que conozcamos la cinemática del Universo en expansión. Es por esto que la distancia depende de la elección del modelo cosmológico que se considere. 
        
        En una expansión uniforme la distancia entre dos observadores fundamentales, $i,j$ en dos épocas $t_1$ y $t_2$ cambia de manera que 

    \begin{align} \label{2.2.15}
        \frac{\varrho_i(t_1)}{\varrho_j(t_1)} & = \frac{\varrho_i(t_2)}{\varrho_j(t_2)} = {\text{constante}} \notag \\
        \frac{\varrho_i(t_1)}{\varrho_i(t_2)} & = \frac{\varrho_j(t_2)}{\varrho_j(t_2)} = ... =  {\text{constante}} = \frac{a(t_1)}{a(t_2)}.
    \end{align}
    
	    
	\begin{figure}[h]
		\centering
    	\begin{subfigure}{\textwidth}
    		\begin{subfigure}{0.4\textwidth}
    		    \includegraphics[width=\textwidth]{light_cone_harrison_pp208.png}
        		\refstepcounter{subfigure}\label{fig:2.6a}
    		\end{subfigure}
    		\begin{subfigure}{0.4\textwidth}
    		    \includegraphics[width=\textwidth]{cono_luz_pp159.png}
        		\refstepcounter{subfigure}\label{fig:2.6b}
    		\end{subfigure}    		
    	\end{subfigure}
		\caption{Izquierda: Espacio de la relatividad especial contiene puntos (eventos), líneas de mundo (cadenas de eventos y cada evento tiene conos de luz), y rayos de luz. Derecha: Diagrama de espacio - tiempo que ilustra la definición de la distancia de la coordenada radial comovil.}
	\end{figure}

    En modelos isotrópicos, $a(t)$ representa una función universal conocida como {\bf{factor de escala}}, la cual explica cómo la distancia entre cualquiera de los dos observadores fundamentales cambia con el tiempo cósmico $t$. Si tomamos $a(t) = 1$ para la época presente, $t_0$, y renombramos el valor de $\varrho$ en el presente como $r$, (\ref{2.2.15}) se escribe como

        \begin{equation} \label{eq:2.2.16} 
            \varrho (t) = a(t)r,
        \end{equation}

    donde $r$ lleva la etiqueta de distancia, la cual está unida a una galaxia u observador fundamental durante todo el tiempo y recibe el nombre de {\bf{coordenada de distancia radial comóvil}}, donde el término ``comóvil'' indica que no se ve afectada por la expansión del Universo. La variación en la distancia propia en el Universo en expansión es $a(t)$. Las distancias propias perpendicular a la línea de visión también pueden cambiar por un factor $a$ entre $t_0$ y $t$, a causa de la isotropía y homogeniedad del modelo,
    
        \begin{equation} \label{2.2.17} 
            \frac{\Delta l(t)}{\Delta l(t_0)} = a(t).
        \end{equation}

    De la métrica (\ref{eq:2.2.14}) 

    \begin{align}
        a(t) & = \frac{R_c(t) \sin[\varrho/R_c(t)] d \theta}{R_c(t_0) \sin[r/R_c(t_0)] d \theta}, \label{2.2.18} \\
        \frac{R_c(t)}{a(t)} \sin \left[\frac{a(t)r}{R_c(t)} \right] & = R_c(t_0) \sin \left[\frac{r}        {R_c(t_0)} \right] \label{2.2.19},
    \end{align}

    válido siempre que 
    
        \begin{equation} \label{eq:2.2.20} 
            R_c (t) = a(t) R_c (t_0),
        \end{equation} 

    de aquí que, para conservar la isotropia y homegeindad, {\bf{la curvatura del espacio cambia a medida que el Universo se expande}} como $k = R_c^{-2} \propto a^{-2}$.
    
    Si ahora el radio de curvatura de la geometría del espacio $R_c$ en la época presente lo etiquetamos como R'
    
        \begin{equation} \label{eq:2.2.21} 
            R_c (t) = a(t) R'.
        \end{equation}
        
    Sustituyendo (\ref{eq:2.2.16}) y (\ref{eq:2.2.20}) en la métrica (\ref{eq:2.2.14})

    \begin{equation} \label{eq:2.2.22} 
        \boxed{ds^2 = dt^2 - \frac{a^2(t)}{c^2} [dr^2 + R'^2 \sin^2(r/R')(d \theta^2 + \sin^2 \theta d\phi^2)],}
    \end{equation}

    obtenemos la {\bf{métrica de Robertson - Walker}}, la cual contiene dos incognitas: 

        \begin{enumerate}
            \item La función desconocida pero etiquetada como el factor de escala $a(t)$, que describe la dinámica del Universo,
            \item La constante, también desconocida, $R'$, la cual describe la curvatura espacial del Universo en la época presente. 
        \end{enumerate} 

    Si se usa la {\bf{distancia de diámetro angular comóvil}}: $r_1 = R' \sin(r/R')$, la métrica puede ser escrita como:
    
        \begin{equation} \label{eq:2.2.23} 
            ds^2 = dt^2 - \frac{a^2(t)}{c^2} \left[ \frac{dr_1^2}{1 - \kappa r_1^2} + r_1^2 (d\theta^2 + \sin^2 \theta d\phi^2) \right],
        \end{equation}
        
    siendo ahora $\kappa = 1/R'^2$. Además, por un reescalamiento adecuado de la coordenada $r_1$: $\kappa r_1^2=r_2^2$, la métrica se escribe:

        \begin{equation} \label{eq:2.2.24} 
            ds^2 = dt^2 - \frac{R_1^2(t)}{c^2} \left[ \frac{dr_2^2}{1 - \kappa r_2^2} + r_2^2 (d\theta^2 + \sin^2 \theta d\phi^2) \right],
        \end{equation}
    
    donde $\kappa = +1, 0 -1$ para universos con geometría esférica, plana o hiperbólica. En este reescalamiento $R_1(t)= R_c(t_0) a= R'a$ por lo que $R_1(t)$ en la época presente es $R'$ más bien que la unidad. 
    Las métricas (\ref{eq:2.2.22}), (\ref{eq:2.2.23}) y (\ref{eq:2.2.24}) pueden definir el intervalo invariante $ds^2$ entre eventos en cualquier época o lugar en el Universo en expansión. Note además que, más adelante, será igualmente válido considerar unidades de $c=1$. 
    
    Finalmente, vamos a repasar algunos aspectos importantes:

        \begin{enumerate}

            \item El {\bf{tiempo cósmico $t$}} es el tiempo medido por un reloj llevado por el observador fundamental,
            \item {\bf{$r$ es la coordenada de distancia radial comóvil}} que está fija a una galaxia para siempre y que es la distancia propia que tendría la galaxia si sus líneas de mundo fueran proyectadas hacia adelante a la época $t_0$ y sus distancias medidas en ese tiempo. 
            \item $a(t)dr$ es el {\bf{elemento de distancia propia (geodésica) en la dirección radial en la época}} $t$. 
            \item $a(t) [R'\sin(r/R')] d\theta = a(t) r_1 d\theta$ es el {\bf{elemento de distancia propia perpendicular a la dirección radial subtendida por el ángulo}} $d\theta$ en el origen. 
            \item $a(t) [R'\sin(r/R')] \sin \theta d\phi = a(t) r_1 \sin \theta d\phi$ es el {\bf{elemento de distancia propia en la dirección}} $\phi$.
        
    \end{enumerate} 
    
      No sabemos nada acerca de la física que gobierna la tasa de expansión de Universo; sin embargo, $a(t)$ absorbe este fenómeno todavía desconocido. 
      

\section{Observaciones en Cosmología}

  \subsection{Corrimiento al Rojo Cosmológico} \label{sec:2.3.1} 

    Cuando hablamos del {\textit{corrimiento al rojo cosmológico}}, nos referimos al desplazamiento de las líneas espectrales hacia longitudes de onda más grandes asociadas a la expansión isotrópica del sistema de galaxias. Así, el corrimiento al rojo $z$ es definido como

        \begin{equation} \label{eq:2.3.1}
            z = \frac{\lambda_0 - \lambda_e}{\lambda_e},
        \end{equation}

    siendo $\lambda_0$ la longitud de onda observada y $\lambda_e$ la longitud de onda de la línea emitida. Si $z$ fuera interpretada como una velocidad de recesión $v$ de una galaxia, $z$ y $v$ serían relacionadas por el desplzamiento Doppler Newtoniano

        \begin{equation} \label{eq:2.3.2} 
            v= cz,
        \end{equation}

    que fue utilizada por Hubble para derivar la relación velocidad-distancia, $v=H_0 r$.
    
    En cosmología, el corrimiento al rojo tiene un significado más profundo. Por ejemplo, si consideramos un paquete de ondas de frecuencia $\nu_1$ emitido entre los tiempos cósmicos $t_1$ y $t_1 +  \Delta t_1$ de una galaxia distante, y es recibido por un observador en la época presente $t_0$ y $t_0 + \Delta t_0$, la señal se propaga a lo largo de {\bf{conos nulos}}, $ds^2=0$, y si además $d\theta = d\phi = 0$, la métrica (\ref{eq:2.2.22}) puede escribir como:
    
        \begin{equation} \label{eq:2.3.3}
            dt = - \frac{a(t)}{c} dr \hspace{0.4cm} \frac{c dt}{a(t)} = -dr,
        \end{equation} 
        
    con $a(t) dr$ el {\bf{intervalo de distancia propia en el tiempo cósmico}} $t$. Para el borde delantero del paquete de ondas 

        \begin{equation} \label{eq:2.3.4} 
            \int_{t_1}^{t_0} \frac{cdt}{a(t)} = - \int_r^0{dr}.
        \end{equation}
        
    El final del paquete de ondas debe viajar la misma distancia en unidades de la coordenada de distancia comovil ya que $r$ es fija a la galaxia para siempre. Así:

        \begin{align} 
            \int_{t_1 + \Delta t_1}^{t_0 + \Delta t_0}{ \frac{cdt}{a(t)}} & = - \int_r^0{dr}, \label{eq:2.3.5}  \\
            \int_{t_1}^{t_0}{ \frac{cdt}{a(t)} +  \frac{c \Delta t_0}{a(t_0)} -\frac{c \Delta t_1}{a(t_1) }}   & =  \int_{t_1}^{t_0}{ \frac{cdt}{a(t)}}. \label{eq:2.3.6} 
        \end{align}
     
     
    Y ya que $a(t_0) = 1$

        \begin{equation} \label{eq:2.3.7} 
            \boxed{\Delta t_0 = \frac{\Delta t_1}{a(t_1)}.}
        \end{equation}


    Esta es la expresión para el fenómeno de la {\bf{dilatación del tiempo}} (equivalente al fenómeno de la relatividad especial). Galaxias distantes son observadas en algún tiempo cósmico anterior $t_1 < 1$, por lo que se observa que los fenómenos tardan más en nuestro marco de referencia que en el de la fuente. Si $\Delta t_1 = \nu_1^{-1}$ es el periodo de las ondas emitidas y  $\Delta t_0 = \nu_0^{-1}$ el período observado, (\ref{eq:2.3.6}) puede ser escrita como
    
        \begin{equation} \label{eq:2.3.8} 
            \nu_0 = \nu_1(t_1) a(t_1),
        \end{equation}
        
    y haciendo uso de (\ref{eq:2.3.1}) y (\ref{eq:2.3.8})
    
        \begin{align}
            z & = \frac{\lambda_0 - \lambda_e}{\lambda_e} = \frac{\lambda_0 }{\lambda_e} - 1 = \frac{\nu_1}{\nu_0} - 1, \label{eq:2.3.9}  \\
            z + 1 & = \frac{\nu_1}{\nu_0} = \frac{1}{a(t_1)} \label{eq:2.3.10}, 
        \end{align}
        
    finalmente 
    
        \begin{equation} \label{eq:2.3.11}
            \boxed{a(t_1)  = \frac{1}{z+1}.} 
        \end{equation}    
        
    
    Esta última expresión representa un resultado muy importante es cosmología y es otra manera de expresar el corrimiento al rojo; una {\bf{medida del factor de escala del Universo cuando la fuente emitió la radiación}}. 
    Cuando observamos una galaxia con desplazamiento al rojo $z = 1$, el factor de escala del Universo cuando se emitió la luz fue $a (t) = 0.5$, i.e., las distancias entre observadores fundamentales (o galaxias) eran la mitad de sus valores actuales. 
    
    De (\ref{eq:2.3.4}) podemos obtener una expresión para la {\bf{coordenada de distancia radial comovil}} $r$:

    \begin{equation} \label{eq:2.3.12}
        r = \int_{t_1}^{t_0}{\frac{c dt }{a (t)}}.
    \end{equation} 

    $r$ {\bf{es una distancia artificial que depende de cómo el Universo se ha expandido entre la emisión y recepción de la radiación}}.
    
    La dilatación del tiempo, expresión (\ref{eq:2.3.7}), está relacionada con las supernovas tipo 1a (SN 1a). La estrecha dispersión en sus magnitudes absolutas que tiene exactamente la misma curva de luz  (la variación en el tiempo en sus luminosidades durante la explosión de la SN) ha hecho que estos objetos sean herramientas útiles en cosmología. Las grandes luminosidades de las SN tipo 1a hace que estas puedas ser observadas a altos corrimiento al rojo. 
    
    \subsection{Ley de Hubble} 

    La ley de Hubble puede escribirse como: 


        \begin{equation} \label{eq:2.3.13} 
            \frac{\varrho}{dt} = H\varrho,
        \end{equation}

    con $\varrho$ la distancia propia, $\varrho = a(t) r$. Usamos $H$ en vez de $H_0$ porque vamos a considerar la {\bf{``constante de Hubble''}} en cualquier época y no en el presente. Sustituyendo $\varrho = a(t) r$,
    
        \begin{align}
            r\frac{da(t)}{dt} & = Ha(t)r,  \label{eq:2.3.14} \\
            \boxed{H = \frac{\dot{a}}{a}.} \label{eq:2.3.15} 
        \end{align}

    Si consideramos la constante de Hubble en el tiempo presente, $t= t_0$, $a=1$, entonces 
    
        \begin{equation} \label{eq:2.3.16}
            \boxed{H_0 = (\dot{a})_{t_0}.}
        \end{equation}
        
    La {\bf{constante de Hubble define la tasa de expansión del Universo en el presente}}. Para cualquier época podemos definir: 

        \begin{equation} \label{eq:2.3.17}
            \boxed{H(t) = \frac{\dot{a}}{a}.}
        \end{equation}
    
 \subsection{Diámetros angulares}

    En la sección anterior, (\ref{sec:2.3.1}) hemos considerado únicamente la parte radial. Siguiendo con la métrica de Roberton - Walker 

        $$ds^2 = dt^2 - \frac{a^2(t)}{c^2} [dr^2 + R'^2 \sin^2(r/R')(d \theta^2 + \sin^2 \theta d\phi^2)],$$
    
    consideraremos ahora el componente espacial relevante $d\theta$. La longitud propia $d$ de un objeto en el desplazamiento al rojo $z$, correspondiente al factor de escala $a(t)$ viene dado por el incremento de la longitud propia perpendicular a la dirección radial:
  
        \begin{align}
            d & = a(t) R' \sin \left(\frac{r}{R'} \right) \Delta \theta = a(t) D \Delta \theta = \frac{D \Delta \theta}{1+z}, \notag \\
            \Delta \theta & = \frac{d(1+z)}{D},\label{eq:2.3.18}
        \end{align}
        
    donde distinguimos una nueva {\textit{medida de distancia}} $D = R' \sin(r/R')$. Para pequeños $z$, $z << 1, r <<R'$ la ec. (\ref{eq:2.3.18}) se reduce a la relación euclidiana $d = r \Delta \theta$. Entonces también puede ser escrita como:

        \begin{equation} \label{eq:2.3.19} 
            \Delta \theta = \frac{d}{D_A},
        \end{equation}
        
    donde se ha definido una nueva medida $D_A$ conocida como {\bf{distancia de diámetro angular}}.
    
    Otro cálculo importante es el del {\bf{diámetro angular de un objeto que continúa participando en la expansión}} como es el caso de una perturbación infinitesimal en la expansión del Universo. Un ejemplo es el diámetro angular que estructuras a gran escala presentes en el Universo actual habrían subtendido en una época anterior, e.g., la recombinación, si simplemente se hubiesen expandido con el Universo. Este se usa para calcular tamaños físicos correspondientes a las escalas angulares de las fluctuaciones observadas en la radiación del CMB. Si el tamaño de un objeto es actualmente de $d(t_0)$ y se expandió con el Universo, su tamaño físico en el desplazamiento al rojo fue de

        $$d(t_0) a(t)=  \frac{d(t_0)}{(1+z)}$$,
    
    por lo tanto, ese objeto subtendió un ángulo

    \begin{equation} \label{eq:2.3.20}
        \Delta \theta = \frac{d(t_0)}{D}.
    \end{equation}
    
\subsection{Intensidades Aparente}

    Se tiene una fuente con luminosidad $L(\nu_1), (\mathrm{W Hz^{-1}})$, y con un corrimiento al rojo $z$. $L$ es la {\textit{energía total emitida sobre $4 \pi$ estereorradián por unidad de tiempo por unidad de intervalo de frecuencia}}. Siendo $\nu_0$ la frecuencia en la época presente $t_0$, queremos calcular la densidad de flujo $S(\nu_0)$ de la fuente en la frecuencia $\nu_0$ en que esta se observa. Específicamente, ¿cuál es la energía recibida por unidad de tiempo, por unidad de área, por unidad de ancho de banda, en unidades de $\mathrm{W m^{-2} Hz^{-1}}$. De ecs. (\ref{eq:2.3.8}) y (\ref{eq:2.3.11})
    
         $$\nu_0 = a(t_1) \nu_1 = \frac{1}{z+1}.$$
         
    Podemos suponer que la fuente emite $N(\nu_1)$ fotones con energía $h \nu_1$, en el ancho de banda $\nu_1, \nu_1 + \Delta \nu_1$ en el intervalo de tiempo propio $\Delta t_1$. De esta manera $L(\nu_1)$ es

        \begin{equation} \label{eq:2.3.21}
            L(\nu_1) = \frac{N(\nu_1) h \nu_1 }{\Delta \nu_1 \Delta t_1}.
        \end{equation}
        
    Imagine que la fuente, en un tiempo $t_1$, es el centro de una ``esfera'' y que los fotones que ha emitido están distribuidos sobre una ``capa'' que rodea a la esfera. Cuando esta capa de fotones llega al observador, en una época $t_0$, cierta cantidad de ellos son interceptados por un telescopio. En $t_0$, los fotones observados tienen frecuencia $\nu_0 = a(t_1) \nu_1$ (ec. \ref{eq:2.3.8}), un intervalo de tiempo propio $ \Delta t_0 = \Delta t_1/a(t_1)$ (ec. \ref{eq:2.3.37}) y en el ancho de banda $ \Delta \nu_0 = a(t_1) \nu_1$.
    
    Para conocer cuántos fotones están sobre esta ``capa'' que rodea la esfera entre la época $t_1$ y $t_0$ es necesario relacionar el diámetro del telescopio $\Delta l$  al diámetro angular $\Delta \theta$ que subtiende en la fuente en la época $t_1$. De ec. (\ref{eq:2.3.18}), si consideramos que para un tiempo presente, $t_0$, $a(t_0) = 1$
    
        \begin{equation} \label{eq:2.3.22}
            \Delta l = D \Delta \theta,
        \end{equation}
    
    siendo $\Delta \theta$ el ángulo medido por un observador fundamental, el cual se encuentra en la fuente. Si ahora introducimos la noción de ángulo sólido $d\Omega$, podemos considerar cómo los fotones que la fuente ha emitido se extienden sobre $d \Omega$, como se observa desde la fuente en la geometría curva. Si se supone que el Universo no se expande, el área superficial sobre la cual se observan los fotones en un tiempo $t$ después de su emisión se escribe como: 

        \begin{equation} \label{eq:2.3.23} 
            dA = R_c^2 \sin^2 \frac{x}{R_c} d\Omega,
        \end{equation}

     siendo $x = ct$. Por otro lado, si el Universo se expande, el radio de curvatura $R_c$ cambia como el Universo lo haga, y en lugar de $x/R_c$, tendríamos 
     
        \begin{equation} \label{eq:2.3.24} 
            \frac{1}{R'} \int_{t_1}^{t_0}{\frac{c dt}{a}} =     \frac{r}{R'},
        \end{equation}
     
     con $r$ la coordenada de distancial radial comóvil. Sustituyendo (\ref{eq:2.3.24}) en $dA$

    \begin{equation} \label{eq:2.3.25} 
        dA = R'^2 \sin^2 \frac{r}{R'} d\Omega.
    \end{equation} 
    
    Consideraremos el diámetro del telescopio como lo vemos desde la fuente como $\Delta l = D \Delta \theta$. Recordemos que las distancia comóviles toman en cuanta la expansión del Universo, a diferencia de las distancias propias. Por lo tanto, el área superficial del telescpio es $\pi \Delta l^2 / 4$ y el ángulo sólido subtendido por esta área superficial en la fuente es $d \Omega = \pi \Delta \theta^2/4$. El número de fotones incidentes sobre el telescopio en un tiempo $\Delta t_0$ es 
    
        \begin{equation} \label{eq:2.3.26} 
            N (\nu_1) \frac{\Delta \Omega}{4 \pi},
        \end{equation}
     
     
    donde ahora ellos son observados con frecuencia $\nu_0$. De esta manera, podemos conocer la expresión para la densidad de flujo de la fuente, i.e., la cantidad de energía recibida por unidad de tiempo, por unidad de área, por unidad de ancho de banda. Esto es
    
        \begin{equation} \label{eq:2.3.27} 
            S(\nu_0) = \frac{N (\nu_1) h \nu_0 \Delta \Omega} {4\pi \Delta t_0 \Delta \nu_0 (\pi/4)\Delta l^2 }.
        \end{equation}
    
    Haciendo uso, nuevamente, de las ecs. (\ref{eq:2.3.7}) y (\ref{eq:2.3.8}), además de las relaciones de la luminosidad, (\ref{eq:2.3.21}) , y el diámetro del telescopio, (\ref{eq:2.3.22}), podemos expresar (\ref{eq:2.3.28}) como:
    
        \begin{align}
            S( \nu_0) & = \frac{[N(\nu_1) h \nu_1 (t_1) a(t_1)] (\pi \Delta \theta^2/4)}{4 \pi [\Delta t_1/a(t_1)] [\nu_1 a(t_1)] (\pi/4)(D^2 \Delta \theta^2)}, \notag \\ 
            \\
            S(\nu_0) & = \frac{L(\nu_1) a(t_1)}{4\pi D^2} =\frac{L(\nu_1)}{4\pi D^2(1+z)}. \label{eq:2.3.29}
        \end{align}
    
    (\ref{eq:2.3.29}) relaciona la intensidad observada $S(\nu_0)$ a la luminosidad intrínseca de la fuente $L(\nu_1)$. 
    Para luminosidades y densidades de flujo {\bf{bolométricas}} se considera entonces la energía total emitida en un ancho de banda finito $\Delta \nu_1$ y que es recibida en el ancho de banda $\Delta \nu_0$. Se tiene
    
        \begin{align}
            L_{bol}= L (\nu_1) \nu_1 & = 4 \pi D^2 S(\nu_0)(1+z) \times \Delta \nu_0 (1+z), \label{eq:2.3.30} \\
            & = 4 \pi D^2 (1+z)^2 S_{bol}, \label{eq:2.3.31}
        \end{align}
        
    de donde se desprende que 
    
        \begin{align}
            S_{bol} & = S(\nu_0) \nu_0, \label{eq:2.3.32} \\
            S_{bol}  & = \frac{L_{bol}}{4 \pi D^2 (1+z)^2} =  \frac{L_{bol}}{4     \pi D_L^2}. \label{eq:2.3.33} 
        \end{align}
    
    Note la nueva cantidad $D_L =  D (1+z) $, la cual es conocida como la {\bf{distancia de luminosidad }} de la fuente. Esta cantidad $D_L$ que relaciona a $L_{bol}$ con $S_{bol}$ hace que la misma luzca como una ley del  del cuadrado inverso. 
    
    Es posible medir la densidad de flujo bolométrica en la presente época, $t_0$, si integramos la luminosidad bolométrica en cualquier ancho de banda adecuado siempre que se utilice el $z$  correspondiente. Así, 
    
        \begin{equation} \label{eq:2.3.34}
            \sum_{\nu_0} S(\nu_0) \Delta \nu_0 = \frac{ \sum_{\nu_1} L(\nu_1) \Delta \nu_1 }{4 \pi D^2 (1+z)^2} =   \frac{\sum_{\nu_1} L(\nu_1) \Delta \nu_1}{4 \pi D_L^2}
        \end{equation}
        
    Si se conoce el espectro de la fuente $L(\nu)$, esta última relación puede escribirse en términos de la luminosidad de la fuente en la frecuencia observada $\nu_0$. Esto es:
    
        \begin{equation} \label{eq:2.3.35} 
            S(\nu_0) = \frac{L(\nu_0)}{4\pi D_L^2} \left[ \frac{L(\nu_1)}{L(\nu_0)}(1+z) \right]. 
        \end{equation}
        
    El término entre corchetes se conoce como {\bf{Corrección-$K$}} y es utilizada para ``corregir'' las magnitudes aparentes de galaxias distantes por los efectos de desplazamiento al rojo de sus espectros cuando las observaciones son hechas a través de filtros estándar con una frecuencia de observación fija media $\nu_0$ (Sandage, 1961b). Así, la relación (\ref{eq:2.3.35}) puede escribirse en función de la magnitud absoluta, 
    
        $$M = cte - 2,5 \log_{10} L(\nu_0),$$ 
        
    y la aparente, 
    
        $$m = cte - 2,5 \log_{10} S(\nu_0),$$ 
    
    encontrándose que
    
        \begin{align}
            M & = m - 5 \log_{10} (D_L) - K(z) - 2.5 \log_{10} (4\pi), \hspace{0.9cm} \label{eq:2.3.36} \textrm{donde} \\
            K & = -2.5 \log_{10} \left[ \frac{L(\nu_1)}{L(\nu_0)}(1+z) \right]. \label{eq:2.3.37} 
        \end{align}
    
     Esta corrección-$K$ es correcta para densidades de flujo y luminosidades {\textit{monocromáticas}}.
     
\subsection{Densidades Numéricas}
    
     Se quiere conocer la cantidad de objetos en un intervalo de corrimiento al rojo $z, \, z + dz$, y varias cosas sabemos: existe una relación uno a uno entre $r$ y $z$, siendo $r$ la {\bf{coordenada de distancia propia radial definida en la presente época}}, resultados que ya fueron dados en sec. (\ref{sec:2.2.2}), por lo que ya se conoce el número de objetos en el intervalo de distancia de coordenadas radial cómovil $r, \, r+dr$. El diagrama de espacio - tiempo (cono de luz, fig. \ref{fig:2.6b}) muestra cómo podemos conocer el número de objetos por trabajar en base a los volúmenes comóviles para la presente época. Para este caso, el radio de curvatura de la geometría curva es $R'$, así que el volumen de la concha esférica de grosor $dr$ en la coordenada de distancia comóvil $r$ es
    
        \begin{equation} \label{eq:2.3.38} 
            dV = 4\pi R'^2 \sin^2 (r/R') dr = 4\pi D^2 dr,
        \end{equation}
    
      donde identificamos a $R'^2 \sin^2 (r/R') = D^2$. Si $N_0$ es la densidad de objetos en el espacio actual y su número es conservado como el Universo se expande
      
        \begin{equation} \label{eq:2.3.39}
            dN = N(z) dz = 4\pi N_0 D^2 dr. 
        \end{equation}

    Podemos afirmar que la ec. (\ref{eq:2.3.39}) da la densidad numérica de los objetos en el intervalo $z, z +dz$, asumiendo que la densidad numérica de objetos no cambia con la época cósmica. En el caso en que la densidad numérica de objetos sí cambiara con la época cósmica, e.g., existe una función que depende de $z$, $f(z)$, con $f(z=0) =1$ entonces la cantidad de objetos esperados en el intervalo $dz$ sería: 

    \begin{equation} \label{eq:2.3.40} 
        dN= n(z) dz = 4\pi N_0 F(z) D^2 dr
    \end{equation}
    
\subsection{Edad del Universo}

     Por último, para determinar la edad del Universo, etiquetada como $T_0$, vamos a recordar la ec. (\ref{eq:2.3.3}). Tenemos:
     
        \begin{equation} \label{eq:2.3.41}
            dt = - \frac{a(t)}{c} dr \hspace{0.4cm} \frac{c dt}{a(t)} = -dr,
        \end{equation} 

    y de esta manera, 

        \begin{equation} \label{eq:2.3.42} 
            T_0 = \int_0^{t_0}{dt } = \int_0^{r_{max}}{\frac{a(t) dr}{c}},
        \end{equation}
    
    donde $r_{{\text{máx}}}$ es la coordenada de la distancia comóvil correspondiente a $a=0$ y $z = \infty$.
    
%%%%%%%%%%%%%%%%%%%%%%%%%%%%%%%%%%%%%%%%%%%%%%%%%%%%%%%%%%%%%%%%%%%%%%%%%
    
\section{Modelos de Mundo de Friedmann}
    
    
  \subsection{Ecuaciones de Campo de Einstein}

     Recapitulemos el {\textit{Principio Cosmológico}} y el {\textit{Postulado de Weyl}} (\ref{sec:2.2.1}), además de la {\textit{Teoría de la Relatividad General}}.
   
        \begin{itemize}
            \item El principio cosmológico, el cual establece un Universo isotrópico, homogéneo y que además se expande uniformente a gran escala. ESto nos conduce también a la {\textit{métrica de Walker -  Robertson}}, (\ref{eq:2.2.22}). 
            \item Weyl postula que existe una única línea de mundo que pasa a través de cada punto en el espacio - tiempo. Líneas que provienen de una singularidad en el pasado finito o infinito. Podemos entonces hablar de un fluido que se mueve a lo largo de estas líneas al ritmo en el que el Universo se expande, entonces su comportamiento es como el de un fluido perfecto, donde el {\textit{tensor de energía - momento}} es 
            
                 \begin{equation} \label{eq:2.4.1} 
                    T_{\alpha \beta} = (\varrho_0 + p) u^{\alpha} u^{\beta} - p g^{\alpha \beta},
                \end{equation}
            
            siendo $g^{\alpha \beta}$ el tensor métrica, $\varrho_0$ la densidad de masa propia del polvo,  $u^{\alpha}, u^{\beta}$ los cuadrivectores de la velocidad y $p$ el trimomento del polvo. 
       
            \item Por último, la relatividad general que relaciona el tensor energía - momento a las propiedades geométricas del espacio - tiempo a través de: 
            
            
                \begin{empheq}[box=\fbox]{align} 
                    R_{\mu \nu } - \frac{1}{2} g_{\mu \nu} R & =  8 \pi G   T_{\mu \nu},  \label{eq:2.4.2} \\ 
                    \notag  \\ 
                     R_{\mu \nu } - \frac{1}{2} g_{\mu \nu} R + \Lambda  g_{\mu \nu} & =  8 \pi G   T_{\mu \nu}, \label{eq:2.4.3}
                \end{empheq}
                
            donde  $R_{\mu \nu }$ es el {\textit{tensor de Ricci}}, $R$ el {\textit{escalar de Ricci o curvatura escalar}}, $G$ la constante de Gravitación Universal, $c$ la velocidad de la luz y $\Lambda$ la famosa {\textit{constante cosmológica}}, introducida por Einstein en 1917 con la intención de crear un Universo estático con geometría cerrada y que permitiera la incorporación del {\textit{Principio de Mach}} a la relatividad general (Einstein, 1917). Ernest Mach fue uno de los más influyentes críticos de los conceptos de espacio y tiempo absolutos Newtonianos. Este principio establecía que todas las fuerzas inerciales son debido a las distribuciones de materia en el Universo, el cual publicó luego en su {\textit{``Mechanics''}} y posteriormente Einstein llamó Principio de Mach (para una explicación más amplia consulte: Harrison E., "The Science of the Universe", pp: 236 - 239).
            
        \end{itemize} 

    Estos tres ingredientes son fundamentales para la construcción de modelos cosmológicos estándar. La relatividad general le permitió a Einstein construir modelos coherentes del Universo.
    
 \subsubsection{Ecuaciones de Friedmann}
    
    La isotropía y homogeneidad del Universo implicaron grandes simplificaciones a las ecuaciones de campo de Einstein, ecs. (\ref{eq:2.4.2}) y (\ref{eq:2.4.3}) reduciéndose entonces a este par de ecuaciones:
    
         \begin{empheq}[box=\fbox]{align}
            \ddot{a} = - \frac{4 \pi G}{3} a \left( \varrho + 3p \right) + \frac{1}{3} \Lambda a, \label{eq:2.4.4} \\
            \notag \\ 
            \dot{a}^2 =  \frac{8 \pi G \varrho}{3} a^2 - \frac{1}{R'^2} + \frac{1}{3} \Lambda a^2, \label{eq:2.4.5}
        \end{empheq}
        
    con $a$ el factor de escala normalizado a la época actual $t_0$, $\varrho$ la densidad de masa inercial total del contenido de materia y radiación en el Universo y $p$ está asociado con  la presión. Recordando, además, que $R'$ es la curvatura de la geometría del modelo de mundo en la época presente $t_0$, por lo tanto, $1/R'^2$ es una constante de integración. 
    
    Las ecuaciones de Friedmann describen la dinámica de un Universo isotrópico y homogéneo. Fueron derivada primero por la relatividad general, pero, ¿por qué no con la teoría de Newton? El problema está en que la mecánica clásica es una teoría global que incluye un potencial gravitatorio que diverge en un Universo isotrópico y homegéneo, a diferencia de la relatividad general que es una teoría local.
    
 \subsubsection{Derivación Newtoniana de la Ecuaciones de Friedmann}
 
     En mecánicla clásica es importante recordar que la noción de conservación de la masa y de la energía son independientes. En la cosmología Newtoniana para referirnos a distancias podemos considerar un fluido cósmico pero teniendo en cuenta que no estamos en un marco de referencia inercial. Así, al momento de aplicar la segunda ley de Newton tenemos que considerar que el punto de referencia puede cambiar. De esta manera, la distancia relativa viene dada por $R(t) = a(t) R$.
     
     La masa  encerrada $M_{int}$ dentro de una esfera de radio $R(t)$ es: 
    
        \begin{equation} \label{eq:2.4.6}
            M_{int} = \int_M{dm} = \frac{4\pi}{3} \rho(t) a^3(t) R^3.
        \end{equation}
    
     Imponiendo que $dM/dt=0$
    
        \[
            \frac{dM}{dt}   = 0 =  \frac{4\pi}{3} \left( \rho(t)3 a^2(t) \dot{a}(t)  R^3 + a^3(t) R^3 \dot{\rho} \right), 
        \]
    
        \begin{equation} \label{eq:2.4.7} 
            \boxed{\dot{\rho} = -3 \rho(t) \frac{\dot{a}}{a}.}
        \end{equation}
        
    Nos referimos a ec. (\ref{eq:2.4.7}) como la {\textit{ecuación de Friedmann Newtoniana de la conservación de masa}}. Este resultado tiene sentido ya que en mecánica clásica el campo no transporta energía por lo que el campo gravitatorio no ejerce presión. 
    
    
    Por el teorema de Gauss, el potencial gravitatorio es: 
    
        \begin{equation} \label{eq:2.4.8} 
            V_{grav} = -4 \pi G \int_0^{R(t)}{\frac{\rho(t)r^2}{r}dr} = -4 \pi G \rho(t) \frac{R^2(t)}{2},
        \end{equation}
        
    donde $R(t)$ es la distancia relativa entre dos puntos del Universo.
    
    Por otro lado, la fuerza gravitatoria que un punto de masa $m$ está experimentando es: 
    
        \begin{equation} \label{eq:2.4.9}
            \vec{F} = -m \vec{\nabla} V_{grav} = - G \frac{M_{int}}{R^2(t)} m \hat{r}.
        \end{equation}
        
     De la segunda ley de Newton $\vec{F} = m \vec{a}$, sustituyendo ecs. (\ref{eq:2.4.6}) y (\ref{eq:2.4.9}),
    
        \begin{equation} \label{eq:2.4.10}
            m \frac{d^2 a(t)R}{dt^2} = - Gm \frac{M_{int}}{a^2(t)R^2} = -Gm \frac{4\pi}{3}\frac{a^3(t)R^3}{a^2(t)R^2}\rho(t).
        \end{equation}
    
     Finalmente, 
    
        \begin{equation} \label{eq:2.4.11}
            \boxed{\frac{\ddot{a}(t)}{a(t)} = - \frac{4\pi G}{3}  \rho(t).}
        \end{equation}
    
    
    De esta manera, se ha obtenido la {\textit{ecuación de aceleración de Friedmann Newtoniana}} que implica un Universo no estático. 
    
    Las ecuaciones de conservación de la masa y aceleración, (\ref{eq:2.4.7} y (\ref{eq:2.4.11}), son independientes por lo que tenemos solo dos ecuaciones de Friedmann independientes lineales.
    
    \large{\bf{Conservación de la Energía}}
    
    Estas ecuaciones tienen oculto el principio de conservación de la energía. Si tomamos (\ref{eq:2.4.11}) y la integramos, considerando que $R(t) = a(t) R$: 
    
        \begin{equation} \label{eq:2.4.12} 
            \ddot{R}(t) = - \frac{4\pi G}{3} \rho(t) R(t)   = - G \frac{M_{int} }{R^2(t)}.
        \end{equation}
    
    Multiplicando por $\dot{R}(t)$ e integrando 
    
        \[
            \dot{R}(t) \left( \ddot{R}(t) = - G \frac{M_{int} }{R^2(t)} \right),
        \]
        
        \begin{equation} \label{eq:2.4.13} 
            \frac{1}{2}  \dot{R}^2(t)  = G \frac{M_{int} }{R(t)} + U,
        \end{equation}
    
    la cual tiene la forma de la ecuación de la conservación de la energía. Manipulándola, llegamos a 
    
        \begin{equation} \label{eq:2.4.14} 
            \boxed{\left(\frac{\dot{a}(t)}{a(t)} \right)^2 = \frac{8\pi G}{3} \rho(t) + \frac{2U}{R^2 a^2(t)}.}
        \end{equation}
    
    \large{\bf{Con Constante Cosmológica $\Lambda$}}
    
    En la teoría clásica, $\Lambda$ no aparece de manera natural, pero gracias a las observaciones actuales sabemos que esta actúa como una fuerza repulsiva proporcional a la distancia radial que sufre la masa $m$. Agredando esto a la segunda ley de Newton, (\ref{eq:2.4.10}), se tiene:
    
        \begin{equation} \label{eq:2.4.15}
            m \frac{d^2 a(t)R}{dt^2} = -Gm \frac{4\pi}{3}\frac{a^3(t)R^3}{a^2(t)R^2}\rho(t) + \frac{\Lambda}{3} a(t) mR,
        \end{equation}
    
    donde el $1/3$ es arbitrario en $\Lambda$. Luego de algo de álgebra, obtenemos la ecuación de aceleración: 
    
        \begin{equation} \label{eq:2.4.16}
            \boxed{\frac{\ddot{a}(t)}{a(t)} = - \frac{4\pi G}{3}  \rho(t) + \frac{\Lambda}{3}.} 
        \end{equation}
    
    La constante de integración $U$ obtenida en (\ref{eq:2.4.13}) seguirá siendo la energía mecánica. 
        
     
    
    \begin{equation} \label{eq:2.4.17}
        \boxed{\left(\frac{\dot{a}(t)}{a(t)} \right)^2 = \frac{8\pi G}{3} \rho(t) + \frac{\Lambda}{3} +  \frac{2U}{R^2 a^2(t)}.}
    \end{equation}
    
     \subsubsection{Derivación Relativista de la Ecuaciones de Friedmann}
     
     
     Retomando el conjunto de ecs. de Friedmann, usando la métrica de Walker - Robertson en su forma (\ref{eq:2.2.23}), la cual considera el principio cosmológico y la expansión del Universo, podemos calcular todos los elementos contenidos en las ecs. de Einstein, (\ref{eq:2.2.6}). En el caso del tensor de momento -  energía se considera un fluido perfecto, un fluido isotrópico, i.e., luce igual en cada dirección en la que nos movemos, y el cual tiene la forma $T_{\nu}^{\mu} = {\textrm{Diag}}(- \rho, p, p, p)$, donde $\rho$ y $p$ vienen siendo la densidad de energía y densidad de presión del fluido, respectivamente.  
     
     Ya que el tensor $T_{\nu}^{\mu}$ es conservado por virtud de las identidades de Bianchi, una tercera ecuación independiente, i.e., derivable de las otras dos (\ref{eq:2.4.4}) y (\ref{eq:2.4.5}), puede ser obtenida:

        $$\dot{\rho} + 3H(\varrho + p) =0$$,

    la cual es una ecuación de continuidad pero con la masa de flujo $\mathbf{j} = (\rho +  P)\mathbf{v}$ (Copeland, Sami, \& Tsujikawa (2006). Esta ecuación puede ser escrita también como una declaración de la conservación de la energía. De la primera ley de la Termodinámica
    
        \begin{equation} \label{eq:2.4.18}
            dQ = 0 = dU + pdV,
        \end{equation}
        
    entonces el cambio en la energía (negativo) depende solo del trabajo hecho por el sistema. $U$ es la energía interna y es igual a la suma de todas las energías que contribuyen a la energía total (energía cinética, energía térmica, etc.), $ \varepsilon_{tot} = \sum_i \varepsilon_i,$ en un fluido en el marco relativista. Con $U = \varepsilon_{tot} V$ la energía contenida en el volumen físico $V$ (reescalado como $a^3$). 
    
    
        \begin{equation} \label{eq:2.4.19}
            dU = - pdV \rightarrow{\frac{d}{dt} (\rho a^3) = -p \frac{d}{dt} a^3 }.
        \end{equation} 
        
    Siendo $\dot{a} = da/dt$ y $d\varrho/da = - 3 (\varrho + p)/a$, si derivamos (\ref{eq:2.4.5}) y dividimos por $\dot{a}$, entonces recobramos la (\ref{eq:2.4.4}), la cual tiene la forma de una ecuación de fuerza que contiene implicitamente la Primera Ley de la Termodinámica, que es posible derivar de la mecánica Newtoniana como ya vimos en (\ref{eq:2.4.16}), solo que el término de presión $p/c^2$ no está contenido en ella. Esta presión puede ser interpretada como una ``corrección'' a la densidad de masa inercial, que es diferente a la fuerza de presión que, e.g., sostiene a las estrellas. Por otro lado, el término $\varrho + 3p/c^2$ puede leerse como la {\textit{densidad de masa gravitatoria activa}}.
    
    Friedmann obtuvo la solución general de (\ref{eq:2.4.5}) para modelos de mundo en expansión (Fridmann A. A., 1922, 1924), donde asumió que la constante cosmológica era diferente de cero ($\Lambda \neq 0$). De esta manera, se le llama {\textit{Modelos de Mundo de Friedmann}} ya que estos son capaces de incluir o no el término $\Lambda$. Friedmann murió en 1925 y nunca supo que los modelos de mundo llevarían su nombre. Cuando George Lema$\hat{i}$tre redescubrió sus soluciones en 1927 logró resaltar las grandes contribuciones de Friendmann y la atención de astrónomos y cosmólogos en 1930 (Lema$\hat{i}$tre, 1927). 
    
 
 
 \section{Modelos de Mundo de Friedmann Estándar con $\Lambda = 0$} 
 
    Cuando en cosmología se habla de {\textit{polvo}} se refiere a un fluido sin presión y por lo tanto $p=0$ en las ecuaciones de Friedmann. Comencemos por estudiar el caso en el que $\Lambda=0$. Por conveniencia, se tomará el valor de la densidad del fluido en la época actual $\varrho_0$ y por conservación de la masa $\varrho=\varrho_0 a^{-3}$. Bajo estas suposiciones, las ecuaciones (\ref{eq:2.4.4} y (\ref{eq:2.4.5}) se reducen a: 
 
        \begin{empheq}[box=\fbox]{align}
            \ddot{a} & = - \frac{4\pi G \varrho_0  }{3 a^2}, \label{eq:2.5.1} \\
            \notag \\
            \dot{a}^2 & = \frac{8\pi G \varrho_0  }{3 a} - \frac{1}{R'^2}. \label{eq:2.5.2}
        \end{empheq}
 
     Por conveniencia, se va a definir una {\textit{densidad crítica $\varrho_c$}} para la densidad de los modelos de mundo. Así:
   
        \begin{equation} \label{eq:2.5.3} 
            \varrho_c = \frac{3H_0^2}{8\pi G} = 1.88 \times 10^{-26} \mathrm{h^2 kg m^{-3}},
        \end{equation}
        
    siendo la constante de Hubble $H_0$ con un valor de $100 \, \mathrm{h \, km\, s^{-1} Mpc^{-1}}$ {\footnote{${\mathrm{h}}= H_0/(100 \, km \, s^{-1}\, Mpc^{-1})$}. En la época presente, la densidad del modelo $\varrho_0$ puede referirse al valor de la densidad crítica a través de lo que se conoce como {\textit{parámetro de densidad} $\Omega_0 = \varrho_0/\varrho_c$, donde este parámetro ha sido definido como
    
        \begin{equation} \label{eq:2.5.4} 
            \Omega_0 = \frac{\varrho_0}{\varrho_c} = \frac{8 \pi G \varrho_0}{3 H_0^2}.
        \end{equation}
        
    Así mismo, el parámetro de densidad bariónico es $\Omega_B$, el de la materia ordinario, o luminosa, $\Omega_{matt}$, a de la materia oscura (DM) como $\Omega_{DM}$. Estos parámetros representan las contribuciones a $\Omega_0$.  De (\ref{eq:2.5.3}) y (\ref{eq:2.5.4}), podemos reescribir las ecs. (\ref{eq:2.5.1}) y (\ref{eq:2.5.2}) como: 
    
         \begin{empheq}[box=\fbox]{align}
            \ddot{a} & = - \frac{\Omega_0 H_0^2 }{ 2a^2}, \label{eq:2.5.5}\\
            \notag \\
            \dot{a}^2 & = \frac{\Omega_0 H_0^2  }{a} - \frac{1}{R'^2}. \label{eq:2.5.6}
        \end{empheq}
        
    Con respecto a esta última ecuación, si establecemos los valores en la época actual $t=t_0$, entonces $a=1$ y $\dot{a} = H_0$, encontramos qué: 
    
        \begin{equation} \label{eq:2.5.7} 
            \boxed{ R' = \frac{1/H_0}{(\Omega_0 - 1)^{1/2}},}
        \end{equation}
        
    y recordando de secciones anteriores el radio de curvatura, ec. (\ref{eq:2.2.21}): $R_c = a R'$, además de la curvatura del espacio $\kappa = R_c^{-2}$, obtenemos:
    
    \begin{equation} \label{eq:2.5.8} 
        \boxed{\kappa = \frac{\Omega_0 - 1}{(1/H_0)^2}.}
    \end{equation}
    
    Existe una relación uno a uno entre la densidad del Universo $\Omega_0$ y la curvatura del espacio $\kappa$; uno de los grandes resultados de los Modelos de Mundo de Friedmann cuando la constante cosmológica $\Lambda=0$.
        
 
 \subsection{Dinámica de los Modelos de Friedmann con $\Lambda = 0$} \label{sec:2.5.1}
 
    Si incluimos la relación (\ref{eq:2.5.7}) en (\ref{eq:2.5.6}) 
    
        \begin{equation} \label{eq:2.5.9}
            \boxed{\dot{a}^2 = H_o^2 \left[ \Omega_0 \left( \frac{1}{a} - 1 \right) + 1 \right].}
        \end{equation}
    
    En el caso de $a >> 1$, $\dot{a}^2$ se escribe como
    
        \begin{equation} \label{eq:2.5.10}
            \boxed{ \dot{a}^2  = H_o^2 (1- \Omega_0).}
        \end{equation}
    
    Y de esta últiam ecuación, se desprende tres casos.
    
   
 
        \begin{itemize}
            \item[i.] Modelos con un parámetro de densidad $\Omega_0 < 1$, los cuales tienen una geometría hiperbólica abierta y se expanden a $a=\infty$. Donde 
            
                $$\dot{a}  = H_o (1- \Omega_0)^{1/2}.$$
        
            \item[ii.] También, modelos con un parámetro $\Omega_0 >1$. Aquellos con geometría esférica cerrada. Dejan de expandirse en algún valor finito de $a$, entonces $a=a_{max}$ (en el infinito tienen ``tasas de expansión imaginarias''), el cual alcanzan en un tiempo máximo 
            
                $$t_{max} = \frac{\pi \Omega_0}{2 H_0 (\Omega_0 - 1)^{3/2}}.$$
        
            \item[iii.] Por último, tenemos el caso cuando el parámetro de densidad $\Omega_0 = 1$, cuyo modelo tiene una velocidad de expansión que tiende a cero cuando $a \rightarrow{\infty}$. 
 
            Se conoce como {\textit{Modelo Einstein- de Sitter}} o {\textit{modelo crítico}}. Plantea un Universo que no colapsa ni tampoco se expande por siempre. El valor del factor de escala varía con el tiempo cósmico como:

                \begin{equation} \label{eq:2.5.11} 
                    a(t) ) \left( \frac{t}{t_0} \right)^{2/3},
                \end{equation} 

            donde $\kappa = 0$ y en la época presente                   
                $$t_0 = (2/3) H_0^{2/3}$$.
    
        \end{itemize}
        
        
        \begin{wrapfigure}[19]{r}{0.5\linewidth}        \includegraphics[width=0.5\textwidth]{grafica_scale_factor_timepp219Malcolm.png}
            \caption{\small{$\Omega = \varrho_0/\varrho_c$. Para  $\Omega_0 > 1$, el Universo colapsa a $a=0$. Para $\Omega_0 <1$, el Universo se expande hasta el infinito y la velocidad de expansión es infinita como $a \rightarrow{\infty}$ . Por último, $\Omega_=1$. Para el del tiempo, este viene en términos del tiempo adimensional $H_0 t$. Siendo $t_0$ es el tiempo actual, cuando $\Omega_0 = 0$, $H_0 t_0=1$; $\Omega_0 = 1$, $ H_0 t_0 = 2/3$ y para $\Omega_0 = 2$, $H_0 t_0 = 0.57$. Las tres curvas tienen la misma pendiente $1$ para $a(1)$.}}
            \label{fig:2.7}
    \end{wrapfigure}
    
    En la figura (\ref{fig:2.7}) se muestran algunas soluciones a la ec. (\ref{eq:2.5.9}) el cual ilustra la relación entre la dinámica y geometría de los modelos de Friedmann sin constante cosmológica. El eje $x$ viene en unidades de $H_0^{-1}$. Nótese la línea de la presente época $a=1$.  Las pendientes en ese punto son siempre $1$. Así mismo, la edad actual del Universo para cada parámetro es dada por la intersección de cada curva con la línea en $a=1$. Se aprecia el caso en que $\Omega_0 = 0$, un modelo de mundo vacío también conocido como {\textit{Modelo de Milne}} donde $a(t) = H_0 t$ y $\kappa = - (H_0/c)^2$. Milne construyó su propia teoría conocida como {\textit{relatividad cinemática}}, donde la gravedad no es un elemento principal, y basado en el principio cosmológico y la relatividad especial crea una visión que explica la naturaleza de la gravedad y otras leyes físicas.  Su Universo consiste de una nube esférica de partículas (galaxias) las cuales se expanden dentro de un espacio plano, que es infinito y por lo tanto vacío.  Las partículas colisionan a velocidades próximas a la de la luz. Lo mismo ocurre con la expansión del borde del Universo (Milne, E.A, 1935: E.A. Milne Relativity Gravitation and World Structure Oxford University Press 1935). Este Universo obedece la métrica de Walker -  Robertson (ec. \ref{eq:2.2.22}), tiene geometría global hiperbólica y los tiempos cósmicos medidos en diferentes marcos de referencia están relacionados por las transformación de Lorentz estándar $t’ = \gamma(t – rv/c^2)$, siendo $\gamma = (1-v^2/c^2)^{-1/2}$.
    
    
\subsection{Modelos de Friedmann con Constante Cosmológica}   
    
  \subsubsection{Constante Cosmológica y la Densidad de Energía del Vacío}
    
    Einstein asumía que la estructura a gran escala del Universo era estática, entonces introdujo, de manera más bien {\textit{ad hoc}},  la constante cosmológica para reconciliar esta visión con su teoría de la relatividad general. $\Lambda$ aparecía en sus ecuaciones de campo como una constante. McVittie (1956) y otros, consideraban que $\Lambda$ era una constante de integración. McVittie (1956) y otros, consideraban que $\Lambda$ era una constante de integración, y por ende no podía tener cualquier valor ((McVittie, G. C., 1956, General Relativity and Cosmology (Chapman and Hall, London). En 1933, Lema\^itre sugiere que $\Lambda$ puede ser interpretada como una {\textit{densidad de energía del vacío}} (Lema\^itre, 1933), idea que desde el punto de vista de la física de partículas surge naturalmente. Además, la escala de energía de $\Lambda$  debería ser mucho mayor que la de la constante de Hubble en el presente $H_0$ siempre que, se origine a partir de la densidad de energía del vacío. Esto es conocido como el ``problema de la constante cosmológica'' (S. Weinberg, Rev. Mod. Phys. 61, 1 (1989)). 
    
    Por supuesto que desde diferentes enfoques de la física se ha intentado buscar una ``solución''. Esto incluye gravedad cambiante (e.g., Van der Bij, 1982; W. Buchmuller, 1988), gravedad cuántica (Baum. E, 1983; Hawking S., 1984; Coleman S. R., 1988), teoría de cuerdas (Kachru S. et al., 2003), aproximación de la espuma espacio - tiempo (Garattini R., 2002) y fluctuaciones de vacío de la densidad de energía (Padmanabhan T., 2005; Gurzadyan V. G. \& Xue S. S., 2003).  

    El lado derecho en ec. (\ref{eq:2.4.2}) es también definido como el tensor de Einstein $G_{\mu \nu}$ el cual, junto con el tensor $T^{\mu \nu}$ satisfacen las identidades de Biachi $\nabla_{\nu} G^{\mu \nu} = 0$ y la conservación de la energía $\nabla_{\nu} T^{\mu \nu}=0$. Debido a que la métrica $g^{\mu \nu}$ es constante con respecto a derivadas covariantes ( $\nabla_{\alpha} g^{\mu \nu} = 0$), existe libertad de agregar un término $\Lambda g_{\mu \nu} $ en las ecs. de Einstein, obteniéndose (\ref{eq:2.4.3}) (para un mejor desarrollo, ver Dadhich N., 2004). Tomando traza sobre ella se obtiene $- R +4 \Lambda = 8 \pi G T$. Combinando este resultado de nuevo con (\ref{eq:2.4.3}) se obtiene:
   
        \begin{equation} \label{eq:2.5.12} 
            R_{\mu \nu } - \Lambda  g_{\mu \nu} =  8 \pi G \left(   T_{\mu \nu} - \frac{1}{2} T g_{\mu \nu} \right).
        \end{equation}
 
    Considerando gravedad en el marco Newtoniano con $g_{\mu \nu} = \eta_{\mu \nu} + h_{\mu \nu }$, siendo $h_{\mu \nu }$ la perturbación alrededor de la métrica de Minkowski $\eta_{\mu \nu}$. Calculando la componente $R_{00}$ en (\ref{eq:2.5.12}), la cual puede ser reescrita como un potencial gravitatorio $\Phi$, se obtiene  
    
        \begin{equation}  \label{eq:2.5.13} 
            \Delta \Phi = 4 \pi G \rho - \Lambda. 
        \end{equation}
    
      Para reproducir la ecuación de Poisson Newtoniana, la constante cosmologica debe ser cero o tener un valor muy pequeño en comparación con $4 \pi G \rho$ para poder ser despreciada.  $\Lambda$ tiene unidades de $[\textrm{longitud}]^{-2}]$ por lo que la escala correspondiente tiene que ser mucho mayor que la escala de objetos estelares en la cual la gravedad Newtoniana funcione correctamente. Esto significa que $\Lambda$ se convierte en importante a escalas muy grandes. 
      
      Las ecuaciones de Einstein modificadas entonces en el contexto de la métrica de Walker - Robertson son:
      
        \begin{empheq}[box=\fbox]{align} 
            \ddot{a} = - \frac{4 \pi G}{3} a \left( \varrho + 3p \right) + \frac{1}{3} \Lambda a, \label{eq:2.5.14}  \\
            \notag \\ 
            \dot{a}^2 =  \frac{8 \pi G \varrho}{3} a^2 - \frac{1}{R'^2} + \frac{1}{3} \Lambda a^2, \label{eq:2.5.15} 
        \end{empheq}
 
    
    Esto demuestra que la constante cosmológica contribuye negativamente a la presión por lo que exhibe un efecto repulsivo. 
    Si consideramos un Universo lleno de polvo, $3p = 0$, simplificándose (\ref{eq:2.5.14}) a:
    
        \begin{equation} \label{eq:2.5.16} 
            \ddot{a} = - \frac{4 \pi G a \varrho  }{3}+ \frac{1}{3} \Lambda a =- \frac{4 \pi G\varrho_0 }{3a^2}+ \frac{1}{3} \Lambda a.
        \end{equation} 
 
     Así se considere un Universo vacío ($\varrho=0$), hay una fuerza neta que actúa sobre una partícula de prueba. Si además de vacío es estático, como el que Einstein propuso, 
    
        \begin{equation} \label{eq:2.5.17} 
            \varrho = \frac{\Lambda}{4 \pi G}, \hspace{0.8cm} \Lambda = \frac{1}{R' a^2}.
        \end{equation}
    
    La idea de un Universo estático fue abandonada con el descubrimiento del corrimiento al rojo de estrellas distantes. 
    
    Si la constante cosmológica surge de la idea de una densidad de energía del vacío, entonces existe un grave problema de {\textit{ajuste fino}}. De las observaciones, $\Lambda$ tiene un valor del orden del valor actual deL parámetro de Hubble, $H_0$
    
        \begin{equation} \label{eq:2.5.18} 
            \Lambda \approx H_0^2 = (2.13h \times 10^{-42} \, \mathrm{GeV})^2,
        \end{equation}
 
    
    que corresponde a una densidad crítica $\varrho_{\Lambda}$ 
    
        \begin{equation} \label{eq:2.5.19} 
            \varrho_{\Lambda} = \frac{\Lambda m_{pl}^2}{8\pi} \approx 10^{-47} \, \mathrm{GeV^4},
        \end{equation}
 
    con $m_{pl} = 1.22 \times 10^{19} \, \mathrm{GeV} $ la masa de Planck. Desde el punto de vista de la teoría cuántica de campos, se puede hacer un análisis para estimar la densidad de energía de los campos vacíos.  La {\textit{energía del punto cero}} de campos constribuye con la densidad de la energía oscura. Así, la densidad de energía del vacío evaluada por la suma de de las energía del punto cero de campos cuánticos con masa $m$ es dada por 
    
        \begin{align}
            \varrho_{vac} & = \frac{1}{2}\int_0^{\infty}{\frac{d^3{\mathbf{k} }}{(2\pi)^3} \sqrt{k^2 + m^2}}, \\
            & = \frac{1}{4 \pi^2}\int_0^{\infty}{dk \, k^2 \sqrt{k^2 + m^2}}. \label{eq:2.5.21}
        \end{align}
 
    Esta integral muestra una divergencia ultravioleta: $\varrho_{vac} \propto k^4$. La suma sobre las energías de modo de punto cero debe ser limitada a una frecuencia alta (o distancia corta) hasta la cual el modelo físico tenga sentido. Así, la integral de la energía de punto cero de modos normales (de número de onda $k$) hasta un número de onda máximo $k_{max}$
 
        \begin{equation} \label{eq:2.5.22}
             \varrho_{vac} =  \hbar \frac{k^4_{max}}{16 \pi^2}.
        \end{equation}
 
    Para el caso de la relatividad general, el valor esperado es justo por debado de la escala de la masa de Planck. Por lo tanto, si $k_{max} = m_{pl}$, el valor de la densidad de energía del vacío es del orden de 
    
        \begin{equation} \label{eq:2.5.23}
            \varrho \approx 10^{74} \, \mathrm{GeV^4},
        \end{equation}
    
    valor que es alrededor de $10^{121}$ órdenes más grande que el valor observado (\ref{eq:2.5.19}). 
    
    Con el desarrollo de la idea del rompimiento de simetría en el modelo estándar de partículas, se planteó una relación de la expansión y enfriamiento del Universo con una secuencia de transiciones de fase que acompañan al rompimiento de simetría. Cada transición de fase de primer orden tiene como una contribución a una constante cosmológica dependiente del tiempo $\Lambda(t)$ o densidad de energía oscura. La disminución del valor de esta densidad de energía oscura en cada transición de fase es mayor que un valor presente. Una posible respuesta es que la energía oscura ahora es despreciada, lo cual parece descabellado pero condujo al escenario de la {\textit{inflación}} que tomó lugar en el Universo muy temprano ($\sim 10^{-36} - 10^{-33} s$). Desde esta visión, este es exactamente el tipo de campo que condujo la expansión inflacionaria. Se debe encontrar una explicación al hecho de que $\varrho_v$ decrece por un factor $10^{120}$ al final de la era {\textit{inflacionaria}}. Un valor de $10^{-120}$ es un valor tan pequeño que podría considerarse que $\Lambda=0$ en el modelo de Friedmann. 
    
    Es posible relacionar una densidad de masa $\varrho_m$ con la densidad de energía del vacío o la energía oscura en la presente época. Consideremos entonces un parámetro de densidad relacionada con la energía oscura $\Omega_{\Lambda}$. Reescribiendo (\ref{eq:2.5.14}):
    
        \begin{equation} \label{eq:2.5.24}
            \ddot{a} = - \frac{4 \pi G a}{3}  \left( \varrho_m -2\varrho_v \right).
        \end{equation}
        
    donde estamos considerando la densidad de `polvo' $\varrho_{m}$, la densidad del vacío $\varrho_v$ y la presión $p_v = - \varrho_v c^2$. En este caso, $\varrho_m = \varrho_0/a^3$, mientras $\varrho_v = cte$.Y sustituyendo
    
        \begin{align} 
            \ddot{a} & = - \frac{4 \pi G \varrho_0}{3 a^2} + \frac{8 \pi G \varrho_v a}{3}, \label{eq:2.5.25} 
            \hspace{0.5cm}  \\
            & \textrm{o para un tiempo presente, ($a=1$):} \notag \\
            \ddot{a}(t_0) & = - \frac{4 \pi G \varrho_0}{3} + \frac{8 \pi G \varrho_v }{3}.  \label{eq:2.5.26} 
        \end{align}
    
    Comparando los términos de esta última ecuación con la (\ref{eq:2.5.14}) concluimos que
    
        \begin{equation} \label{eq:2.5.27}
            \boxed{\Lambda = 8\pi G\varrho_v.}
        \end{equation}
    
    Un parámetro a la densidad del vacío y la constante cosmológica asociado también es derivable: 
    
        \begin{align}
             \Omega_{\Lambda} &  = \frac{\varrho_v}{\varrho_c} = \frac{8 \pi G \varrho_v}{3 H_0^2}, \label{eq:2.5.28} \hspace{0.8cm} \textrm{por lo que} \\
             \Lambda &= 3 H_0^2 \Omega_{\Lambda}. \label{eq:2.5.29}
        \end{align}
    
    De esta manera, sustituyendo (\ref{eq:2.5.4}) además de las dos últimas relaciones, (\ref{eq:2.5.14}) y (\ref{eq:2.5.15}) pueden también escribirse como: 
    
        \begin{empheq}[box=\fbox]{align}
            \ddot{a} & = - \frac{\Omega_0 H_0^2 }{ 2a^2} +  H_0^2 \Omega_{\Lambda} a, \label{eq:2.5.30} \\ 
            \notag \\
            \dot{a}^2 & = \frac{\Omega_0 H_0^2  }{a} - \frac{1}{R'^2} + H_0^2 \Omega_{\Lambda} a^2. \label{eq:2.5.31}
        \end{empheq}
    
    O, para la época actual siendo $\dot{a} = H_0$ y $a=1$ (\ref{eq:2.5.31}) es
    
        \begin{align}
            \frac{c^2}{R'^2} & = H_0^2 [(\Omega_0 + \Omega_{\Lambda}) - 1] \hspace{1cm} \text{o} \label{eq:2.5.32} \\
            \notag \\
            \kappa & = \frac{1}{R'^2} = \frac{[(\Omega_0 + \Omega_{\Lambda})]}{c^2/H_0^2}. \label{eq:2.5.33}
        \end{align}
    
    De aquí, la condición de que las secciones espaciales son espacios planos Euclideos se escribe 
   
        \begin{equation} \label{eq:2.5.34}
            (\Omega_0 + \Omega_{\Lambda}) = 1. 
        \end{equation}
        
    Ya que $R_c$ cambia con el factor de escala, $R_c=aR'$, si la curvatura del espacio es cero en este momento, también debió serlo en todos los tiempos del pasado. 
    
 \subsubsection{Consideraciones Generales de la Dinámica de Modelos de Mundo con $\Lambda \neq 0$} \label{sec:2.5.1.2}
 
    La importancia de estos modelos es que ayudan a determinar y mejoras valores de los parámetros cosmológicos. Algunas consideraciones son: 
    
    \begin{itemize}
        \item $\Lambda < 0$: 
        
            Existe una fuerza adicional a la gravedad la cual ralentiza la expansión del Universo. Si revisamos la relación (\ref{eq:2.5.16}), podemos extraer que independientemente de cuán pequeños sean los valores de $\Omega_{\Lambda}$ y  $\Omega_{0}$ , la expansión del Universo se va a revertir.  
            
        \item $\Lambda, \Omega_{\Lambda} > 0$:
        
            Los modelos con constante cosmológica positiva conducen a una ``fuerza repulsiva'' contraria a la fuerza de la gravedad. Tienen una tasa de expansión mínima. Si $\ddot{a} = 0$ en ec. (\ref{eq:2.5.30}), obtenemos el factor de escala mínimo, $a_{min}$ que, insertado en (\ref{eq:2.5.31}), nos permite conocer la tasa mínima de expansión, $\dot{a}_{min}$
            
                \begin{empheq}[box=\fbox]{align}
                    a_{min} & = \left( \frac{\Omega_0}{ 2\Omega_{\Lambda}} \right)^{1/3}, \label{eq:2.5.35} \\
                    \notag \\
                    \dot{a}_{min}^2 & = \frac{3 H_0^2 }{2} (2\Omega_{\Lambda} \Omega_0^2 )^{1/3} - \frac{c^2}{R'^2}. \label{eq:2.5.36}
                \end{empheq}
    
    \end{itemize} 
    
    Analicemos la ec. (\ref{eq:2.5.35}). 
     
     \begin{itemize}
         \item Cuando el lado derecho de la igualdad es mayor que cero, su comportamiento es reflejado en la fig. (\ref{fig:2.8}a). Para valores grandes de $a$, la dinámica sigue  a la  de Universos de Sitter
         
            \begin{equation} \label{eq:2.5.37}
                a(t) \propto \exp \left[\left(\frac{\Lambda}{3} \right)^{1/2} t \right] = \exp (\Omega^{1/2}_{\Lambda} H_ot)
            \end{equation}
        
       \item Si el lado derecho es menor que cero, la función $a(t)$ tiene dos ramas (fig. \ref{fig:2.8}b). Existe un rango de factores de escala para los cuales no hay solución. 
       
            \begin{itemize}
                \item [Rama A:] La dinámica está dominada  $\Lambda$; la fuerza repulsiva es tan fuerte que el Universo nunca se contrajo a una escala a la que la fuerza atractiva de la gravedad pudiera contrarrestar o vencer su efecto. 
        
                \item [Rama B:] En este caso, el Universo nunca se expandió alcanzando valores suficientemente grades de $a$ como para que la fuerza repulsiva transportada a través de $\Lambda$ pudiera evitar el colapso del Universo. No presenta una singularidad inicial, por lo que el Universo ``rebotó''  bajo la influencia de la fuerza repulsiva. 
            \end{itemize}
            
        En el caso límite, en el que el parámetro de densidad $\Omega_0 = 0$ la dinámica del modelo y su solución vienen dadas por
            
            \begin{align} 
                \dot{a}^2 & = H_0^2 [\Omega_{\Lambda} a^2 - (\Omega_{\Lambda}-1)], \hspace{1cm} \textrm{con solución} \label{eq:2.5.38} \\
                a & = \left(\frac{\Omega_{\Lambda} - 1}{\Omega_{\Lambda}} \right)^{1/2} \cosh \Omega_{\Lambda}^{1/2} H_0 \tau, \label{eq:2.5.39}
            \end{align}
        
        siendo $\tau = t - t_{min}$, el cual es medido desde el momento en el que el Universo `rebotó', i.e., cuando $a= a_{min}$. El comportamiento asintótico es debido al colapso exponencial que experimenta y son soluciones exponenciales de Sitter
        
            \begin{equation} \label{eq:2.5.40}
                a = \left(\frac{\Omega_{\Lambda} - 1}{\Omega_{\Lambda}} \right)^{1/2} \exp (\pm \Omega_{\Lambda}^{1/2} H_0 \tau).
            \end{equation}
            
        En estos Universos `rebotantes', el valor más pequeño de $a$, $a_{min}$, es aquel con el desplazamieto al rojo más grande que un objeto pueda tener. 
        
        \item Cuando $\dot{a}_{min} \approx 0$. La figura (\ref{fig:2.8}c) ilustra lo que se conoce como el {\textit{modelo de Eddington - Lema\^itre}} donde $\dot{a_{min}} = 0$. La interpretación de A, B y C es:
        
            \begin{itemize}
                \item[A:] El Universo se expandió desde un punto en el origen en algún tiempo pasado finito y eventualmente alcanzará un estado estacionario en el futuro infinito. 
                \item[B:] El Universo se está expandiendo lejos de una solución estacionaria en el pasado infinito. 
                \item[C:] Es un estado estacionario y además inestable ya que, de ser perturbado, el Universo va a moverse hacia el estado que colapsa A, o hacia B.
                
                Para el caso del Universo estático de Einstein, el estado estacionario sería alcanzado en el presente día. Para estos modelos con $\dot{a} = 0$, el valor de la constante cosmologica puede ser obtenido de la ec. (\ref{eq:2.5.35}):
                
                
                    \begin{align}
                        \Lambda & = \frac{3}{2} \Omega_0 H_0^2 (1 + z_c)^3 \hspace{0.4cm} \text{o equivalentemente} \label{eq:2.5.41} \\
                        \Omega_{\Lambda} & = \frac{\Omega_0}{2} (1 + z_c)^3, \label{eq:2.5.42}
                    \end{align}
           
                siendo $z_c$ el desplazamiento al rojo de objetos estacionarios. 
            \end{itemize} 
            
         Ya que los Universos estáticos de Eddington - Lema\^itre no presentan dinámica, $\dot{a}=0$, sustituyendo (\ref{eq:2.5.41}) (\ref{eq:2.5.36}) se encuentra 
    
            \begin{equation} \label{eq:2.5.43}
                \Omega_0 = \frac{2}{ (1 + z_c)^3 - 3  (1 + z_c) + 2} = \frac{2}{ z_c^2(z_c + 3)},
            \end{equation}
            
        encontrándose una relación uno a uno entre la densidad media de materia en el Universo $\Omega_0$ y el desplazamiento al rojo en el estado estacionario $z_c$.
        
        Los modelos con constante cosmológica positiva pueden tener edades mayores a $H_0^{-1}$. En casos límites como los modelos de Eddington - Lema\^itre con $\dot{a}_{min}=0$ en el pasado infinito, el Universo fuera infinitamente viejo. 
        
       Los conocidos {\textit{modelos de Lema\^itre}} son otro posible caso de Universos con edades mayores a $H_0^{-1}$, donde el valor de $\Omega_{\Lambda}$ es tal que $\dot{a}_{min} >0$. Como se mencionó arriba, un ejemplo de estos modelos es el de la figura (\ref{fig:2.8}d).
       
    \end{itemize}  
    
    
        
        \begin{figure}[H]  
            \centering
                \includegraphics[width=0.8\textwidth]{graficos73_pp226Mlalcon.png}
                \caption{\footnotesize{Modelos con $\Lambda \neq 0$ }. En el gráfico {\bf{a}} los parámetros cosmológicos son $\Omega_0=0.3$ y $\Omega_{\Lambda}= 0.7$, los cuales son valores que se ajustan bastante bien a las estimaciones actuales. {\bf{b}} ilustra el Universo `rebotante'con $\Omega_0 = 0.05$ y $\Omega_{\Lambda}=5$. El cero en el tiempo cósmico se ha establecido para cuando $\dot{a}=0$. {\bf{c}} muestra un modelo de Eddington - Lema\^itre que tiene un corrimiento al rojo estacionario $z_c = 3$, lo que es equivalente a un factor de escala $a=0.25$. El modelo {\bf{d}} tiene parámetros $\Omega_0 = 0.01$ y $\Omega_{\Lambda}= 0.99$, donde la edad del Universo puede exceder por mucho a $H_0^{-1}$. Por último, {\bf{a}} y {\bf{d}} son conocidos como los modelos de Lema\^itre. (Bondi, 1960)}
                \label{fig:2.8}
        \end{figure}
        
       
    Existe una fuerte evidencia de que Universos con geometría plana tiene un valor de $\Omega_{\Lambda} + \Omega_0$ cercano a la unidad. La dinámica de estos modelos planos con diferentes valores de $\Omega_0$ y $\Omega_{\Lambda}$ se muestran en la figura 8. Dichos modelos ilustran cómo la edad del Universo puede ser más grande que $H_0^{-1}$ para valores de $\Omega_{\Lambda}$ suficientemente grandes.
        
        \begin{wrapfigure}[15]{r}{0.55\linewidth}       
            \centering
                \includegraphics[width=0.6\textwidth]{flat_models_pp229_Malcolm.png}
                \caption{\footnotesize{Dinámica de modelos con geometría plana, $\Omega_{\Lambda} + \Omega_0 = 1$. El eje $x$ tiene unidades de  $H_0^{-1}$. Este gráfico puede ser comparado con la figura (\ref{fig:2.7}), el cual representa el caso donde $\Omega_{\Lambda}=0$ .}}
                \label{fig:2.9}
        \end{wrapfigure}
    
  \subsection{Cosmología Observacional}
  
      La cosmología actual es dominada por modelos de mundo con un valor finito de la constante cosmológica, sin embargo, modelos con $\Omega_{\Lambda}=0$ son igualmente considerados para hacer comparaciones y así obtener mejores estimaciones sobre parámetros cosmológicos.
      
       La constante de Hubble $H_0$ ha sido introducida, y sabemos que mide la tasa de expansión del Universo en el tiempo actual, i.e., $a=1$. Así mismo, también podemos considerar el {\textit{parámetro de desaceleración}} $q_0$ el cual es adimensional y en la presente época viene definido como
       
         \begin{align} 
            q_0 & = - \left( \frac{a\ddot{a}}{\dot{a}^2}\right)_{t_0}, \hspace{1.0cm} \textrm{o para la época presente:} \label{eq:2.5.44} \\
            q_0 & = - \frac{\ddot{a}}{H_0^2}. \label{eq:2.5.45}
        \end{align}
        
    siendo  $a(t_0)=1$ y $\dot{a}=H_0$. Considerado la ecuación dinámica (\ref{eq:2.5.31} 
       
       \begin{equation} \label{eq:2.5.46} 
	        q_0 = \frac{\Omega_0}{2} - \Omega_{\Lambda}.
        \end{equation} 
        
    Usando los valores preferenciales de $\Omega_{\Lambda} = 0.7$ y $\Omega_0 = 0.3$, nosotros obtenemos un valor del parámetro de desaceleración $q_0 = -0.55$, lo cual indica que efectivamente, el Universo se está desacelerando en la presente época producto de que la energía oscura domina sobre la gravedad. 
    
  \subsubsection{Relación Tiempo Cósmico – Corrimiento al Rojo}
  
    Combinando (\ref{eq:2.5.30}) y (\ref{eq:2.5.31}) se obtiene
    
        \begin{align}
	        \dot{a}^2 &= \frac{\Omega_0 H_0}{a} – H_0^2 [(\Omega_0 + \Omega_{\Lambda} -1] +  \Omega_{\Lambda}H_0^2 a^2, \notag \\
	        \dot{a} & = H_0 \left[ \Omega_0 \left( \frac{1}{a} - 1 \right) + \Omega_{\Lambda}(a^2 - 1) + 1 \right]^{1/2}, \label{eq:2.5.47} 
        \end{align}
    
    y considerando que $a = (1+z)^{-1}$,

        \begin{equation} \label{eq:2.5.48} 
	        \frac{dz}{dt} = - H_0(1+z) [ (1+z)^2 (\Omega_0 z + 1) - \Omega_{\Lambda} z (z+2) ]^{1/2}.
        \end{equation}
        
    El tiempo cósmico, desde el Big Bang hasta cualquier $z$ puede ser medido por integrar (\ref{eq:2.5.43}), con límites desde $z = \infty$ hasta $z$. 

        \begin{equation} \label{eq:2.5.49}
	        t = \int_0^t{dt} = - \frac{1}{H_0} \int_{\infty}^z{\frac{dz}{  (1+z)[(1+z)^2 (\Omega_0 z + 1) - \Omega_{\Lambda} z (z+2) ]^{1/2}}.} 
	    \end{equation}
    
         \begin{itemize}
         
	        \item Modelos con $\Omega_{\Lambda}=0$. 

		    Para un parámetro de densidad $\Omega_0 >1$ y $\Omega_0 < 1$, la relación corrimiento al rojo -  tiempo cósmico, respectivamente es 
		
		        \begin{equation} \label{eq:2.5.50}
			        t(z) = \frac{\Omega_0}{H_0( \Omega_0 - 1)^{3/2}}[\sin^{-1} x^{1/2} - x^{1/2} (1-x)^{1/2}].
		        \end{equation}
		  
		    y
		  
		        \begin{equation} \label{eq:2.5.51}
	                t(z) = \frac{\Omega_0}{H_0(1 - \Omega_0)^{3/2}}[ y^{1/2} (1+y)^{1/2} + \sinh^{-1} y^{1/2}].
                \end{equation}
		
		    Para $z>>1$ y $\Omega_0 z >>1$ (\ref{eq:2.5.50}) y (\ref{eq:2.5.51}) se reducen a 
		 
		        \begin{equation} \label{eq:2.5.52}
                    t(z) = \frac{2}{3H_0 \Omega_0^{1/2}} z^{-3/2},
                \end{equation}
                
            y entonces hallar la edad del Universo para diferentes valores de $\Omega_0$ es posible. Por ejemplo, para el caso $\Omega_0=1$:
            
                \begin{equation} \label{eq:2.5.53}
                    t_0 = \frac{2}{3H_0}.
                \end{equation}
            
            Los casos más útiles y simples son estos con $\Omega=1$, el caso de un Universo vacío tipo Milne, $\Omega_0 =0$ con una edad de $H_0^{-1}$; y el caso con $\Omega_0=2$, cuya edad calculada es $0.571 H_0^{-1}$. 
            
            \item Modelo con $\Omega_{\Lambda} \neq 0$:
            
            Para que se cumpla la condición de que la curvatura del espacio es cero $(\Omega_{\Lambda}  + \Omega_0) =1$, $R'\rightarrow{\infty}$. De (\ref{eq:2.5.49})
                
                \begin{equation} \label{eq:2.5.54}
                    t = \int_0^t{dt} = - \frac{1}{H_0}\int_{\infty}^z{\frac{dz}{(1+z)[\Omega_0(1+z)^3+\Omega_{\Lambda}]^{1/2}}},
                \end{equation}
            con solución 
        
                \begin{align}
                    t &= \frac{2}{3H_0\Omega_{\Lambda}^{1/2}} \ln \left(\frac{1+\cos \theta }{\sin \theta} \right)  \label{eq:2.5.55}  \hspace{0.5cm} \text{donde:} \notag \\
                    \tan \theta &= \left(\frac{\Omega_0}{\Omega_{\Lambda}} \right)^{1/2} (1+z)^{3/2}
                \end{align} 
                
        Para la época presente, $z=0$. Así: 
        
            \begin{equation} \label{eq:2.5.56}
                t_0 = \frac{2}{3H_0\Omega_{\Lambda}^{1/2}} \ln \left[\frac{1 + \Omega_{\Lambda}^{1/2}}{(1-\Omega_{\Lambda})^{1/2}} \right].
            \end{equation}
            
        Para valores del los parámetros $\Omega_0 = 0.1$ y $\Omega_{\Lambda}=0.9$, la edad del Universo es de $1.28 H_0^{-1}$; para valores ideales de $\Omega_{\Lambda} = 0.7$ y $\Omega_0=0.3$, se tiene un Universo con $0.964 H_0^{-1}$. 
                    
    \end{itemize} 
    
 \subsection{Problema de la Llanura {\textit{Flatness}}}
 
    Existen algunos fenómenos que los modelos de mundo no son capaces de explicar, por lo que ciertos parámetros deben ser introducidos en estos modelos a priori. La incorporación de lo que se conoce como el escenario inflacionario a dado respuestas a varios de los problemas que el modelo estándar no termina de resolver por sí solo. Significa que, durante una época muy temprana del Universo, la expansión procedió de manera exponencial, siendo posiblemente dominado por la constante cosmológica $\Lambda$, como es el caso del Universo De Sitter, sec. (\ref{sec:2.5.1.2}). 
    
    En la sección (\ref{sec:2.5}) ya se había obtenido el parámetro de Hubble y el parámetro de densidad para el tiempo presente, $H_0, \Omega_0$. En el caso de $H_0$ fue mencionado que esta depende del tiempo cósmico. En general 
    
        \begin{equation} \label{eq:2.5.57}
            H(t) = \frac{\dot{a}}{a}.
        \end{equation} 
        
    En (\ref{eq:2.5.47}) se obtuvo la relación que existe entre la constante de Hubble y el desplazamiento al rojo, donde se usó uno de los resultados más importantes en cosmología, la relación (\ref{eq:2.3.11}): $a = (1 + 1)^{-1}$. Así
    
        \begin{equation} \label{eq:2.5.58}
            H(z) = \frac{\dot{a}}{a} = H_0 [(1+z)^2 (\Omega_0 z +1) - \Omega_{\Lambda}z(z+2)]^{1/2}.
        \end{equation} 
        
    Equivalente, al parámetro de densidad en la actualidad $\Omega_0$ (\ref{eq:2.5.4}), un parámetro puede ser obtenido.  
    
        \begin{equation} \label{eq:2.5.59}
            \Omega = \frac{\varrho}{\varrho_c} = \frac{8 \pi G \varrho}{3 H^2}.
        \end{equation}
        
    Considerando `polvo’, i.e., un fluido sin presión ($p=0$),  $\varrho = \varrho_0 a^{-3} = \varrho_0(1+z)^3 $ y sustituyendo en esta última relación
    
        \begin{align}
            \Omega & = \frac{8 \pi G }{3 H^2}  \varrho_0(1+z)^3, \label{eq:2.5.60} \\
            & = \frac{\Omega_0}{\left[ \frac{\Omega_0 z+1}{1+z} \right]  - {\Omega_{\Lambda} \left[ \frac{1}{(1+z)} - \frac{1}{(1+z)^3} \right]}}. \label{eq:2.5.61}
        \end{align}
        
    Considerando grandes desplazamientos al rojo $z >> 1$, $\Omega_0 z >> 1$, y analizando el denominador de la última expresión, para el primer término domina $\Omega_0$, mientras que el segundo tiende a cero, obteniéndose entonces que el parámetro de densidad en cualquier época y para $z$ grandes  $\Omega \rightarrow{1}$ sin importar el valor que tenga en la época actual. Significa que, a alto desplazamiento al rojo, la dinámica de $\Omega_{\Lambda}$ no es importante. Hay marcadas diferencias entre la dependencia que puede existir en las densidades de energía oscura y de materia sobre $z$. Si $\Omega_{\Lambda} = 0$
    
        \begin{equation} \label{eq:2.5.62}
            \left( 1 - \frac{1}{\Omega} \right) = (1+z)^{-1} \left( 1 - \frac{1}{\Omega_0} \right).
        \end{equation}
    
    Si, regresamos a la sección (\ref{sec:2.5.1}) y utilizamos las soluciones de (\ref{eq:2.5.9}), obtenemos
    
        \begin{equation} \label{eq:2.5.63}
            a = \Omega_0^{1/3} \left( \frac{3 H_0 t}{2} \right)^{2/3},
        \end{equation}
    
    lo que reafirma la conclusión expuesta al final de la sección; {\bf{la dinámica de todos los modelos de mundo, con $\Omega_{\Lambda} =0$, tienden a esos del modelo crítico en sus escenarios tempranos}}. Así, si el valor de $\Omega_{0}$ fuera diferente de $1$ en el pasado lejano, también lo fuera en el presente, como se puede apreciar en (\ref{eq:2.5.62}). Para un modelo cosmológico considerado, el valor del parámetro $\Omega_{0}$ no viene siendo más que un ajuste fino (\textit{fine tuning}}) que forma parte de las condiciones iniciales de nuestro Universo. 
    
    Hemos definido el parámetro de densidad como la relación de la densidad de materia – energía del Universo $\varrho_0$ a la densidad crítica $3H_0^2/8 \pi G$. Aquí, es importante destacar que, quien realmente tiene un valor cercano a $1$ es un, digamos, parámetro total $\Omega_{T,0}$, que incluye la constante cosmológica: $\Omega_{T,0} = \Omega_{0} + \Omega_{\Lambda} \approx 1$, pero por el momento consideremos solo $\Omega_{0}$ como el único contribuyente. La evidencia observacional sugiere un valor de $\Omega_0 \approx 0.3$, o $\Omega_{T,0} \approx 1$, respaldado por WMAP ({\textit{Wilkinson Microwave Anisotropy Probe}}) , que a su vez afirma que la curvatura del espacio $\kappa \rightarrow{0}$. Específicamente, el tercer año de WMAP, junto con otros datos del {\textit{Supernova Legacy Project}} o el HST {\textit{key project}} indican que  $\Omega_{T,0} \approx 1.015$ (Spergel et al., 2003, ApJS, 148, 175; 2006, astro-ph/0603449). Note que el $\Omega_0 \approx 0.3$ representó un problema cuando se creía que solo este era la principal contribución a $\Omega_{T,0}$. Esto viene siendo el origen de lo que se conoce como el {\textit{Problema de la Llanura (flatness)}}, i.e., el parámetro debe haber sido $\Omega = 1$ en el pasado distante si su valor en el presente es cercano a $\Omega_0$. 
    
   
    La inflación entonces resuelve el problema de la llanura al suponer que en una época muy temprana del Universo, este estuvo dominado por (alguna) $\Lambda$, y durante este período, la dinámica siguió la de Universos de De Sitter, (\ref{eq:2.5.40}),  donde la expansión se acelera exponencialmente, que es lo que también ocurriría si se perturba el modelo estático de Einstein. De esta manera, al expermintar solo la influencia de $\Lambda$, la ec. (\ref{eq:2.5.62}) muestra que $\Omega \rightarrow{1}$ a medida que $z$ decrece. 
    
    
   \large{\textbf{Era Inflacionaria}}


        Esta es una descripción más cualitativa de lo que fue la {\textit{inflación}}. Actualmente, se cree que la evolución del Universo en sus primeras etapas consistió de transiciones hacia estados de menor simetría, partiendo de un estado de máxima simetría. Las cuatro fuerzas fundamentales de la naturaleza estaban unificadas y en el primer rompimiento de simetrías, fueron separadas en dos fuerzas: la fuerza gravitatoria y la fuerza {\textit{hiper débil}}, de igual intensidad en un principio. Poca o nada de distinción existe entre la materia y la antimateria, entre quarks y leptones. 
        
        En el primer segundo, se estima que el Universo tuvo una densidad de $10^6 \mathrm{\,g \, cm^{-3}}$ y la temperatura de $10^{10} \, \mathrm{K}$. En lo que se conoce como la {\textit{época de Planck}}, $10^{-44} \, \mathrm{s}$, la densidad cósmica era del orden de $10^{94} \, \mathrm{g} \, \mathrm{cm^{-3}}$ y la temperatura del orden de los $10^{32} \, \mathrm{K}$ (equivalente a una energía de partículas de $10^{19} \, \mathrm{GeV}$) {\footnote{$1 \,\mathrm{GeV}$ es un billón de electrón voltios, igual a la energía térmica de una partícula a $10000 \,\mathrm{K}$}}. Además, el espacio -  tiempo consistía de una `espuma' densa de fluctuaciones cuánticas reales en escalas de longitud de $10^{-33} \, \mathrm{cm}$ y escalas de tiempo de $10^{-43} \, \mathrm{s}$. Este Universo temprano extremo fue muy corto; debido a la expansión, la temperatura cayó del valor de Planck y a los $10^{-36} \, \mathrm{s}$ alcanzó un valor crítico de $10^{28} \, \mathrm{K}$ (equivalente a $15 \, \mathrm{GeV}$). A continuación, vino el segundo rompimiento de simetrías de la gran unificación: la fuerza hiper débil a su vez, se dividió en la fuerza electro-débil y la fuerza fuerte de la era quark-lepton. Es con esta última transición que inicia la era de la inflación cósmica, además que establece las diferencias entre materia y antimateria, entre quarks y leptones.
        
        Sydney Coleman propuso la idea de una transición de fase de un estado dominado por la fuerza hiper débil a un estado de menor energía compuesto de leptones y quarks dominados por la fuerza electro débil y la fuerza fuerte (Coleman, S., \& De Luccia, F., 1980) (Coleman, S., \& De Luccia, F. (1980). Gravitational effects on and of vacuum decay. Physical Review D, 21(12), 3305–3315. doi:10.1103/physrevd.21.3305 )
        Esta fase puede ser pensada como la transición del agua a hielo cuando la temperatura cae. Ocurre en el punto de congelación $273 \mathrm{\, K}$ ($0^{\circ}C$). Al caer la temperatura, el agua pura no perturbada se enfría a una temperatura menor que la del punto de congelación antes de transformarse en hielo. Similarmente, en el caso del Universo temprano, la temperatura cayó, la transición a quarks y leptones no ocurre en el instante en que la temperatura alcanza un valor crítico de $10^{28} \, \mathrm{K}$. La transformación abrupta en una mezcla de quarks – leptones que se ha súper enfriado fue lo que Coleman denominó {\textit{vacío falso}}; el más bajo estado posible de energía disponible para la fuerza hiper débil. Cuando la transición finalmente  ocurre espontánamente, el vacío falso libera su inmensa energía latente, devolviendo la temperatura casi a su valor inicial de la gran unificación y formando quarks, leptones y gluones (bosones mediadores de la fuerza fuerte). 
        
        En 1980, Alan Guth, quien elaboró la primera teoría del Universo inflacionario en 1970, mientras estudiaba el conocido {\textit{problema del monopolo}} consideró también el vacío falso. Pensaba que este vacío falso existió en un estado extraordinario de presión negativa que causó la expansión del Universo acelerada. Guth argumentaba que la expansión acelerada, lo cual denominó inflación, resolvía el problema del monopolo, de la llanura y del horizonte. Así, cualquier Universo que experimentó en su infancia un período de expansión acelerada es ahora conocido como un {\textit{universo inflacionario}}.
        
        

\section{Redes}

    Una red es un conjunto de objetos conectados. Normalmente se refiere a los objetos como nodos o vértices, y se representan
    generalmente como puntos. Y a las conexiones entre los nodos usualmente se le dicen arcos, estos pueden ser, simples y estar
    representados mediante una linea; pueden ser dirigidos y estar representados mediante una flecha que indica la dirección
    de la conexión y para ambos caso estos pueden tener una ponderación que representa la fuerza de la relación que tiene los
    elementos de la red. Las redes se ven representadas de manera matemática mediante el uso de grafos. Las redes pueden
    representar distintos tipos de sistemas , por ejemplo, la red de Internet, en donde las computadoras serían los nodos y los
    arcos serian la conexión física o inalámbrica entre ellos.
\begin{figure}[H]
    \centering
    \includegraphics[width=0.5\linewidth]{imagenes/grafo2.png}
    \caption{Grafo que representa una red de 7 nodos con sus respectivos arcos}
    \label{fig:redejemplo}
\end{figure}

   Una manera de representar los arcos que conectan los nodos de una red, es por medio de una matriz de adyacencia $W$ que es cuadrada y donde cada uno de sus campos $W_{ij}$ representa la fuerza de la interacción del nodo $i$ sobre el nodo $j$. Para la red representada por la figura \ref{fig:redejemplo}, la matriz de adyacencia es:
\begin{equation}
    W = \bordermatrix{~ & 1 & 2 & 3 & 4 & 5 & 6 & 7 \cr
                       1 & 0 & 3 & 6 & 0 & 0 & 0 & 0 \cr
                       2 & 3 & 0 & 0 & 1 & 0 & 0 & 0 \cr
                       3 & 6 & 2 & 0 & 4 & 2 & 0 & 0 \cr
                       4 & 0 & 1 & 4 & 0 & 6 & 0 & 0 \cr
                       5 & 0 & 0 & 2 & 0 & 0 & 2 & 2 \cr
                       6 & 0 & 0 & 0 & 0 & 2 & 0 & 3 \cr
                      7 & 0 & 0 & 0 & 0 & 2 & 3 & 0 \cr}
\end{equation}

   
\section{Redes de mapas acoplados} \label{sec:RMA}
	Las redes neuronales son sistemas complejos, es decir, sistemas compuestos de múltiples elementos no lineales que interactúan
	entre si. Para estudiar estos sistemas dinámicos no lineales, se puede construir un modelo rico y complejo mediante el
	acoplamiento de un gran numero de sistemas dinámicos de orden bajo. Además, estos pueden simplificarse aun mas considerándolos
	discretos en tiempo y espacio. Las Redes de mapas acoplados ( RMA ) o Coupled Map Lattices ( CML ) están constituidas por un
	conjunto de elementos, los cuales poseen un espectro continuo que evoluciona según un mapa, y viene escrita de manera general
	de la siguiente forma:
    \begin{equation}\label{globalRMA}
        x_{t+1}^i = f(x_{t}^i) + g(V_{t}^{i}) \; ,
    \end{equation}
    donde, $x_{t}^i$ es la variable de estado del nodo o celda $i$ en tiempo discreto $t$, $i = 1,...,N$ es el  índice que
    identifica cada uno de los N elementos de la red, $f(x_{t}^i)$; es una función que representa la dinámica local del nodo
    y el término $g(V_{t}^{i})$, es una función que determina la influencia o la interacción que tiene sobre el elemento $i$-esimo
    el conjunto de sus vecinos $V_{t}^{i}$.
	
	Uno de los criterios usados para determinar si los elementos de la red de mapas acoplados se encuentran sincronizados consiste
	en medir si los elementos de la red exhiben órbitas iguales mientras el tiempo transcurre, esto es:
    \begin{equation}
        |x_{t}^i - x_{t}^i| < \varepsilon \forall i,j, t \rightarrow \infty \; ,
    \end{equation}
    donde $\varepsilon$ es un valor positivo arbitrariamente pequeño, esta definición es válida para órbitas caóticas y periódicas.

\section{Plasticidad del sistema nervioso}
    El término plasticidad corresponde a la capacidad que tiene algún objeto de cambiar de forma y conservarla durante un período
    de tiempo determinado: en cuanto al campo de estudio de este trabajo, se habla de la Neuroplasticidad; que es la propiedad que
    emerge de la naturaleza y funcionamiento de las neuronas cuando éstas establecen una comunicación o sinápsis. Esta dinámica
    deja una huella al tiempo que modifica la eficacia de la transferencia de la información entre las neuronas, es decir, sus
    patrones de conexión sináptico cambian modificando las rutas de interconexión. Este fenómeno está relacionado con la memoria y
    los procesos de aprendizaje.
    
    Una teoría que explora como ocurre la neuroplasticidad es la teoría hebbiana que establece que el valor de una conexión
    sináptica se incrementa si las neuronas de ambos lados de dicha sinápsis se activan repetidas veces de forma simultánea,
    de forma tal que en un futuro no dependerán únicamente de su propia estimulación, sino también, de la activación de las
    neuronas vecinas con la sinápsis incrementada. De esta manera forman una red Hebbiana, para que la plasticidad neuronal sea
    posible también debe existir el fenómeno inverso, es decir, que si una conexión en una red Hebbiana no se usa, debe ir
    perdiendo sus componentes hasta desaparecer, es decir que las neuronas se vayan desconectando unas de otras. Es importante
    destacar, que esta teoría sólo corresponde a una representación de una simplificación del sistema nervioso, por tanto no debe
    tomarse literalmente. Sin embargo este proceso de plasticidad es el que hace al sistema nervioso tan excepcional 
    proporcionándole, su maleabilidad y capacidad de cambio. Esta teoría es comúnmente usada para explicar algunos tipos de
    aprendizaje asociativos, en donde la activación simultanea de las neuronas conduce a un pronunciado aumento de la fuerza
    sináptica. Este aprendizaje se conoce como aprendizaje de Hebb\cite{Grabner2014}.
	
	En los nodos de la red Hebbiana están las neuronas que son células del sistema nervioso. El cerebro humano contiene mas de 50
	billones de neuronas. Del cuerpo de cada neurona se desprenden el axón y las dendritas que son elementos fundamentales en la
	transmisión nerviosa. El axón es una fibra nerviosa única que surge del cuerpo neuronal y transmite información desde el cuerpo 
	de la neurona hasta su extremo donde se producen las neurotransmisiones que provocan las sinápsis que van a las dendritas de
	las neuronas vecinas.
	
    La comunicación de ordenes y mensajes entre las neuronas y las estructuras neuronales como también los órganos del cuerpo, se
    le denomina neurotransmisión, y sólo puede llevarse acabo a través de los neurotransmisores. Los neurotransmisores son
    sustancias  neuro químicas, que actúan como mensajeros, siendo liberada por las terminales del axón neuronal en el espacio
    sinpático. Ellos se unen a los puntos de recepción de las neuronas receptoras acomplándoce a las dendritas del receptor, lo
    cual permite que los iones penetren en la neurona receptora excitándola o inhibiéndola

    La sinapsis es el fenómeno de interacción o comunicación entre dos neuronas que ocurre en una zona determinada (Espacio
    sináptico). Su actividad explica todas las acciones del cerebro, desde las más sencillas; como ordenarle a los músculos su
    contracción, hasta las tareas más complejas, como las que originan y controlan nuestras emociones.

	En el siguiente cápitulo se muestra como están representados matemáticamente todos los elementos anteriormente expuestos, 
	como también su respectiva estructura e implementación de los modelos que analiza este trabajo de grado.

\chapter{Marco teórico}

\section*{Preliminares}

{\textit{Hola}}
  

        
        
\section{Cosmología: Marco Teórico}


  \subsection{Principio Cosmológico} \label{sec:2.2.1}
  
    La isotropía de la Radiación del Fondo Cósmico de Microondas (CMB) nos dice que a primera aproximación el Universo es isotrópico y homogéneo actualmente. Esto último fue asumido en 1917 por Einstein en su modelo estático, que sería el primer modelo del Universo autoconsistente (Einstein, 1917). También, los modelos de Friedman tiempo después se convertirían en los modelos estándar para las dinámicas a gran escala del Universo.
    
    En 1923, Hermann Weyl introdujo su ``Postulado de Weyl'' (Weyl, 1923) que establecía la noción de geodésicas divergentes, las cuales representan las líneas de mundo de galaxias, que no se intersectan excepto en un {\textit{punto singular}} del pasado finito o infinito. Según esto, existe solo una geodésica pasando a través de cada punto en el espacio - tiempo, excepto en el origen.  De aquí es posible asignar un observador a cada línea de mundo. A esto se le conoce como {\textit{Observadores fundamentales}}, donde se supone que cada uno de ellos lleva consigo un reloj estándar y el tiempo medido en este, desde el punto singular, recibe el nombre de {\textit{tiempo cósmico}}.
    
    
    En 1935, Robertson y Walker (Robertson, 1935; Walker, 1936) derivaron la métrica del espacio - tiempo para todos los modelos de Universos isotrópicos, homogéneos y en expansión uniforme, la cual era independiente de la suposición de que las dinámicas a gran escala eran descritas por la Teoría de la Relativdad General, i.e., no importaba cual fuera la física de la expansión,  la forma de la métrica era la de Robertson - Walker, esto partiendo del supuesto de isotropía y homogeneidad del Universo. 
    
    Otra consideración para construir un modelo cosmológico es el conocido {\bf{principio cosmológico}} con el cual nos atrevemos a afirmar que: ``no estamos ubicados en ningún lugar especial en el Universo'', no hay preferencia, por lo que cualquier observador fundamental ubicado en cualquier lugar del Universo pero en la misma época cósmica observa las mismas características a grande escala que nosotros, i.e., la misma expansión de Hubble de la distribución de galaxias, la misma radiación del CMB isotrópica, la misma estructura esponjosa a gran escala de la distribución de galaxias y vacíos (red cósmica), y así sucesivamente. 
    En un sistema de galaxias que se expanden, cada observador en cada galaxia individual observa el mismo flujo de Hubble en la misma época, así, todos tienen derecho a creer que son el centro de un Universo en expansión.
    
     
        
    En cuanto a la geometría, a finales del siglo XVIII se estaban considerando los espacios no euclidianos. Los padres de la geomtría no euclidiana fueron Nikolai Ivanovich Lobachevsky en Rusia y János Bolyai en Transilvania (Lobachevsky, 1829; Bolyai, 1830). Lobachevsky resolvió el problema de la existencia de geometrías no euclidianas, así la misma se puso sobre una base teórica sólida por los estudios de Bernhard Riemann.
    
    \newpage
    
    \begin{wrapfigure}[16]{r}{0.4\linewidth}                                     \includegraphics[width=0.45\textwidth]{superficie_esferica_pp152.png}
                \caption{\footnotesize \footnotesize{Suma de los ángulos de un triángulo en la superficie de una esfera}}
                \label{fig:2.1} 
        \end{wrapfigure}
    
    En la figura (\ref{fig:2.1}) se tiene el caso más simple de la geometría curvada en 2 dimensiones, la superficie de una esfera: un triángulo con dos líneas (con $90^{\circ}$ entre ellas) que parten del polo norte y llegan hasta el ecuador, y una tercera línea que se dibuja sobre el ecuador. Estas líneas son la distancia más corta entre las tres esquinas del triángulo. En geometría curva son geodésicas. Si el radio de una esfera es $R_c$, el área superficial del triángulo ABC es $A= \theta R_c$. Variando el ángulo se obtiene para $\theta = 90^{\circ} \rightarrow{A = \pi R_c^2/2}$ donde la suma de los ángulos del triángulo es $270º$; y si $\theta=0º$ entonces el área también es cero y la suma de los ángulos $180º$. Así, la diferencia de la suma de los ángulos de $180^{\circ}$ es proporcional al área del triángulo:
        
        \vspace{0.4cm}
        
            $$(\textrm{suma de los ángulos del triángulo} - 180^{\circ} \propto \text{área del triángulo},$$
            
        la cual es una propiedad general de los espacios curvos isotrópicos. 
        
 \subsection{La métrica Espacio - Tiempo para Espacios Isotrópicos Curvos} \label{sec:2.1.2} 

    La distancia entre dos puntos separados por $dx, dy, dz$ en un espacio plano viene representado por 
    
        \begin{equation} \label{eq:2.1.1} 
            dl^2 = dx^2 + dy^2 + dz^2. 
        \end{equation}
    
    Siendo el caso más simple de un espacio curvo bidimensional isotrópico la superficie de una esfera, es conveniente utilizar un sistema coordenado polar para describir posiciones en la superficie (ver fig. \ref{fig:2.1}) que para el caso, las coordenadas ortogonales son $\theta$ y $\phi$ y el aumento de la distancia $dl$ entre dos puntos en dicha superficie es:
    
        \begin{equation} \label{eq:2.1.2} 
            dl^2 = R_c^2 d\theta^2 + R_c^2 \sin^2 \theta d\phi^2,
        \end{equation}
        
    donde $R_c$ es el radio de curvatura de la superficie del espacio bidimensional, i.e., el radio de la esfera. La ec. (\ref{eq:2.1.2}) es la {\textit{métrica de la superficie bidimensional}} y puede generalizarse como:
    
        \begin{equation} \label{eq:2.1.3}
            dl^2 = g_{\mu \nu} dx^{\mu} dx^{\nu}, 
        \end{equation}
    
    siendo $g_{\mu \nu}$ el {\textit{tensor métrico}} contenedor de toda la información acerca de la geometría intrínseca del espacio. Otros sistemas de coordenadas pueden definir las coordenadas de un punto en cualquier superficie bidimensional. Para un plano euclidiano:
    
        \begin{equation} \label{eq:2.1.4}
            dl^2 = dx^2 +  dy^2,
        \end{equation}

    en polares

        \begin{equation} \label{eq:2.1.5}
            dl^2 = dr^2 + r^2 d\phi^2.
        \end{equation}
        
    Gauss mostró que con el tensor métrico $g_{\mu \nu}$ es posible determinar la curvatura intrínseca del espacio. Para tensores que pueden ser reducidos a la forma diagonal (ecs. \ref{eq:2.1.2}, \ref{eq:2.1.4}, \ref{eq:2.1.5}) la curvatura intrínseca viene dada por:
    
         \begin{align*}
        \kappa & = \frac{1}{2 g_{11} g_{22}} \left\{ - \frac{\partial^2g_{11} }{\partial x_2^2} - \frac{\partial^2g_{22} }{\partial x_1^2} + \frac{1}{2 g_{11}} \left[ \frac{\partial g_{11} }{\partial x_1} \frac{\partial g_{22} }{\partial x_1} +   \left( \frac{\partial g_{11} }{\partial x_2} \right)^2 \right]
        + \frac{1}{2 g_{22}} \left[ \frac{\partial g_{11} }{\partial x_2} \frac{\partial g_{22} }{\partial x_2} + \left( \frac{\partial g_{22} }{\partial x_1} \right)^2 \right] \right\}. 
        \end{align*}
    
    $\kappa$ se conoce como la {\textit{curvatura gaussiana}} del bi-espacio. La extención a tres espacios isotrópicos es sencilla si mantenemos que cualquier sección bidimensional a través de un espacio tridimensional isotrópico debe ser un bi-espacio isotrópico y ya el tensor métrico par este caso es conocido. 
    
    \newpage
    
    
        \begin{wrapfigure}[16]{r}{0.4\linewidth}                                             \includegraphics[width=0.5\textwidth]{superficie_esferica_pp156_angulos.png}
            \caption{\footnotesize \footnotesize{$\varrho$ es la distancia radial alrededor de la esfera desde el polo y el ángulo $\phi$ mide los desplazamientos angulares en el polo.}}
            \label{fig:2.2} 
        \end{wrapfigure}
    
    Fue mencionado que para un bi-espacio isotrópico, las coordenadas adecuadas son las polares esféricas. De la figura (\ref{fig:2.5}), la distancia alrededor del arco de un gran círculo desde el punto $O$ hasta $P$ es $R_c \theta$, por lo que la métrica se escribe:
    
    
        \begin{equation} \label{eq:2.1.6} 
            dl^2 = d\varrho^2 + R_c^2 \sin^2\left(\frac{\varrho}{R_c} \right) d\phi^2,
        \end{equation}

    donde $\varrho$ es la distancia más corta entre $0-P$  en la superficie de una esfera ya que es parte de un gran  círculo. De esta manera, estamos hablando de una {\textit{distancia geodésica}} en el espacio curvo isotrópico. Recordemos que en espacios curvos las geodésicas juegan el rol de las líneas rectas. 
    
    Ahora, (\ref{eq:2.1.6}) puede ser reescrita como: 

        \begin{equation} \label{eq:2.1.7} 
                x = R_c \sin \left(\frac{\varrho}{R_c} \right).
        \end{equation}
    
    Hallando $dx^2$ con $$d\varrho^2 = \frac{dx^2}{1-\kappa x^2},$$
    la métrica puede ser escrita como: 
    
        \begin{equation} \label{eq:2.1.8} 
            dl^2 = \frac{dx^2}{1-\kappa x^2} + x^2d\phi^2.
        \end{equation}
    
    Recordando que $\kappa = 1/R_c^2$, tres casos deben ser considerados:
    
        \begin{equation}
            \label{eq:2.1.9}
            \kappa  = \left\{
	            \begin{array}{ll}
		            > 0      & {\text{espacio esférico}},, \\
		            = 0 & R_c \rightarrow{\infty}, \hspace{0.4cm}  {\text{espacio plano}}, \\
		            < 0     & {\text{espacio hiperbólico}},
	            \end{array}
	       \right.
        \end{equation}
    
        donde $\kappa = 0$ recupera el espacio euclidiano. Para el incremento espacial en un espacio tridimensional curvo recordemos que cualquier sección bidimensional a través de un tri-espacio isotrópico debe ser un espacio isotrópico de dos dimensiones donde la métrica puede ser ec. (\ref{eq:2.1.6}) o (\ref{eq:2.1.8}). En coordenadas polares esféricas, el desplazamiento angular general perpendicular a la dirección radial es 
        
            \begin{equation} \label{eq:2.1.10}
                d\phi^2 = d\theta^2 +  \sin^2 \theta d\phi^2.
            \end{equation}
            
        (Con $\theta$ y $\phi$ diferentes a los usados en la fig. \ref{fig:2.2}). De esta manera, se puede escribir el incremento espacial de (\ref{eq:2.1.6}) o (\ref{eq:2.1.8}) como sigue
        
            \begin{align}
                dl^2 & = d\varrho^2 + R_c^2 \sin^2\left(\frac{\varrho}{R_c} \right) [d\theta^2 +  \sin^2 \theta d\phi^2], \label{eq:2.1.11} \\
                dl^2 &  =  \frac{dx^2}{1-\kappa x^2} + x^2[d\theta^2 +  \sin^2 \theta d\phi^2] \label{eq:2.1.12} .
            \end{align}
            
        De esta manera, la {\textit{métrica de Minkowski}} para un tri-espacio isotrópico viene escrita como: 

            \begin{equation} \label{eq:2.1.13} 
                \boxed{ds^2 = dt^2 - \frac{1}{c^2} dl^2,}
            \end{equation}

        con $dl$ el de las expresiones (\ref{eq:2.1.11} o \ref{eq:2.1.12}). Aunque $x$ y $\varrho$ son medidas de distancias equivalente, su significado físico es bastante distinto.  Es ahora posible derivar la {\textit{métrica de Robertson -  Walker}}.
        
 \subsection{Métrica de Robertson - Walker} \label{sec:2.1.3}
 
    Queremos aplicar la métrica de Minkowski, ec. (\ref{eq:2.1.13}), a modelos de mundo homogéneos e isotrópicos. Para eso, debemos recurrir a:

        \begin{enumerate}
            \item Principio cosmológico: A primera aproximación, el Universo es isotrópico y homogéneo en la época actual.
            \item Observadores fundamentales: quienes se mueven de modo tal que el Universo siempre parece isotrópico para ellos. 
            \item Tiempo cósmico: cada uno de estos observadores lleva un reloj y el tiempo propio medido por ese reloj es lo que se conoce como {\textit{tiempo cósmico}}.
        \end{enumerate}
        
    Recordemos que según el postulado de Weyl, las geodésicas de todos los observadores se reunen en un punto en el pasado y el tiempo cósmico puede ser medido desde esa época de referencia. 
    
    Ahora bien, considerando las ecs. (\ref{eq:2.1.11}) y (\ref{eq:2.1.13}), podemos escribir la métrica como:
    
        \begin{equation} \label{eq:2.1.14}
            \boxed{ds^2 = dt^2 - \frac{1}{c^2}( d\varrho^2 + R_c^2 \sin^2\left(\frac{\varrho}{R_c} \right) (d\theta^2 +  \sin^2 \theta d\phi^2)],}
        \end{equation}

    siendo $t$ el tiempo cósmico y $d\varrho$ un incremento de la distancia propia \footnote (la cual se ve afectada por la expansión del Universo) en la dirección radial. 
    
    Para un Universo en expansión, esta métrica presenta problemas. Ya que la luz viaja una velocidad finita, observamos todos los objetos astronómicos a lo largo de un cono de luz anterior centrado en la Tierra en la época actual $t_0$ (ver fig. \ref{fig:2.6b}). Así, cuando observamos objetos distantes, no los observamos en la época actual sino en un tiempo anterior $t_1$, cuando el Universo todavía era homogéneo e isotrópico, pero las distancias entre los observadores fundamentales eran menores y la curvatura espacial diferente. Significa entonces que la métrica (\ref{eq:2.1.14}) solo puede ser aplicada a un espacio curvo, isotrópico definido en una única época. 
    
    \vspace{0.6cm}
    
    {\large{\textbf{Cono de Luz}}}
    
    Como una digresión, es importante aclarar acerca del cono de luz y su importancia dentro del marco de la relatividad especial y del Universo. 
    
    El espacio - tiempo consiste de tres elementos que son representados en un diagrama conocido como el {\textit{cono de luz}} (ver fig. \ref{fig:2.3a}). Estos son: 
    
        \begin{itemize}
            \item Puntos que se encuentran en el espacio - tiempo y los cuales representan los {\textit{eventos}}, i.e., algo que ocurre en algún lugar, en algún momento. 
            \item Líneas, que representan las {\textit{líneas de mundo}}, los observadores que se extienden desde el pasado hacia el futuro. Un observador recibe señales de luz que vienen del pasado y transmite señales de luz que salen al futuro. 
            \item Geodésicas nulas que no son otra cosa que los {\textit{rayos de luz}}, que describen líneas a $45^{\circ}$ de la vertical. Para ellas, $ds^2 =0$. 
        \end{itemize}
        
        Las líneas de mundo con un ángulo menor a $45^{\circ}$ de la vertical se conocen como {\textit{líneas temporaloides}} ({\textit{time-like}}), y los eventos están causalmente conectados. Así mismo, las que tienen un ángulo mayor a $45^{\circ}$ con respecto a la vertical son llamadas {\textit{líneas espacialoides}} ({\textit{space-like}}). Eventos sobre estas líneas están  causalmente desconectados; no se pueden afectar mutuamente en sus respectivos conos de luz pasado y futuro ya que están separadas por una distancia mayor que la que la luz puede recorrer en el tiempo que los separa.  
        
        Para medir una distancia propia adecuada que pueda ser incluida en la métrica (\ref{eq:2.1.14})  se pueden alinear un conjunto de observadores fundamentales que estén entre la Tierra y la galaxia. Cada uno de estos observadores va a medir la distancia $d \varrho$ a su próximo observador en un tiempo cósmico particular. Al sumar todos estos $d \varrho$ se puede obtener una distancia adecuada que es medida en una sola época y  que sería posible incluir en la métrica (\ref{eq:2.1.14}). Lo que nosotros observamos son galaxias distantes en una época pasada, i.e., cómo eran en una época anterior, y no se sabe cómo proyectar sus posiciones relativas a nosotros en la época actual hasta que conozcamos la cinemática del Universo en expansión. Es por esto que la distancia depende de la elección del modelo cosmológico que se considere. 
        
        En una expansión uniforme la distancia entre dos observadores fundamentales, $i,j$ en dos épocas $t_1$ y $t_2$ cambia de manera que 

    \begin{align} \label{eq:2.1.15}
        \frac{\varrho_i(t_1)}{\varrho_j(t_1)} & = \frac{\varrho_i(t_2)}{\varrho_j(t_2)} = {\text{constante}} \notag \\
        \frac{\varrho_i(t_1)}{\varrho_i(t_2)} & = \frac{\varrho_j(t_2)}{\varrho_j(t_2)} = ... =  {\text{constante}} = \frac{a(t_1)}{a(t_2)}.
    \end{align}
    
	    
	\begin{figure}[h]
		\centering
    	\begin{subfigure}{\textwidth}
    		\begin{subfigure}{0.4\textwidth}
    		    \includegraphics[width=\textwidth]{light_cone_harrison_pp208.png}
        		\refstepcounter{subfigure}\label{fig:2.3a}
    		\end{subfigure}
    		\begin{subfigure}{0.4\textwidth}
    		    \includegraphics[width=\textwidth]{cono_luz_pp159.png}
        		\refstepcounter{subfigure}\label{fig:2.3b}
    		\end{subfigure}    		
    	\end{subfigure}
		\caption{\footnotesize Izquierda: Espacio de la relatividad especial contiene puntos (eventos), líneas de mundo (cadenas de eventos y cada evento tiene conos de luz), y rayos de luz. Derecha: Diagrama de espacio - tiempo que ilustra la definición de la distancia de la coordenada radial comovil.}
	\end{figure}

    En modelos isotrópicos, $a(t)$ representa una función universal conocida como {\bf{factor de escala}}, la cual explica cómo la distancia entre cualquiera de los dos observadores fundamentales cambia con el tiempo cósmico $t$. Si tomamos $a(t) = 1$ para la época presente, $t_0$, y renombramos el valor de $\varrho$ en el presente como $r$, (\ref{eq:2.1.15}) se escribe como

        \begin{equation} \label{eq:2.1.16} 
            \varrho (t) = a(t)r,
        \end{equation}

    donde $r$ lleva la etiqueta de distancia, la cual está unida a una galaxia u observador fundamental durante todo el tiempo y recibe el nombre de {\bf{coordenada de distancia radial comóvil}}, donde el término ``comóvil'' indica que no se ve afectada por la expansión del Universo. La variación en la distancia propia en el Universo en expansión es $a(t)$. Las distancias propias perpendicular a la línea de visión también pueden cambiar por un factor $a$ entre $t_0$ y $t$, a causa de la isotropía y homogeniedad del modelo,
    
        \begin{equation} \label{eq:2.1.17} 
            \frac{\Delta l(t)}{\Delta l(t_0)} = a(t).
        \end{equation}

    De la métrica (\ref{eq:2.1.14}) 

    \begin{align}
        a(t) & = \frac{R_c(t) \sin[\varrho/R_c(t)] d \theta}{R_c(t_0) \sin[r/R_c(t_0)] d \theta}, \label{eq:2.1.18} \\
        \frac{R_c(t)}{a(t)} \sin \left[\frac{a(t)r}{R_c(t)} \right] & = R_c(t_0) \sin \left[\frac{r}        {R_c(t_0)} \right] \label{eq:2.1.19},
    \end{align}

    válido siempre que 
    
        \begin{equation} \label{eq:2.1.20} 
            R_c (t) = a(t) R_c (t_0),
        \end{equation} 

    de aquí que, para conservar la isotropia y homegeindad, {\bf{la curvatura del espacio cambia a medida que el Universo se expande}} como $k = R_c^{-2} \propto a^{-2}$.
    
    Si ahora el radio de curvatura de la geometría del espacio $R_c$ en la época presente lo etiquetamos como R'
    
        \begin{equation} \label{eq:2.1.21} 
            R_c (t) = a(t) R'.
        \end{equation}
        
    Sustituyendo (\ref{eq:2.1.16}) y (\ref{eq:2.1.20}) en la métrica (\ref{eq:2.1.14})

    \begin{equation} \label{eq:2.1.22} 
        \boxed{ds^2 = dt^2 - \frac{a^2(t)}{c^2} [dr^2 + R'^2 \sin^2(r/R')(d \theta^2 + \sin^2 \theta d\phi^2)],}
    \end{equation}

    obtenemos la {\bf{métrica de Robertson - Walker}}, la cual contiene dos incognitas: 

        \begin{enumerate}
            \item La función desconocida pero etiquetada como el factor de escala $a(t)$, que describe la dinámica del Universo,
            \item La constante, también desconocida, $R'$, la cual describe la curvatura espacial del Universo en la época presente. 
        \end{enumerate} 

    Si se usa la {\bf{distancia de diámetro angular comóvil}}: $r_1 = R' \sin(r/R')$, la métrica puede ser escrita como:
    
        \begin{equation} \label{eq:2.1.23} 
            ds^2 = dt^2 - \frac{a^2(t)}{c^2} \left[ \frac{dr_1^2}{1 - \kappa r_1^2} + r_1^2 (d\theta^2 + \sin^2 \theta d\phi^2) \right],
        \end{equation}
        
    siendo ahora $\kappa = 1/R'^2$. Además, por un reescalamiento adecuado de la coordenada $r_1$: $\kappa r_1^2=r_2^2$, la métrica se escribe:

        \begin{equation} \label{eq:2.1.24} 
            ds^2 = dt^2 - \frac{R_1^2(t)}{c^2} \left[ \frac{dr_2^2}{1 - \kappa r_2^2} + r_2^2 (d\theta^2 + \sin^2 \theta d\phi^2) \right],
        \end{equation}
    
    donde $\kappa = +1, 0 -1$ para universos con geometría esférica, plana o hiperbólica. En este reescalamiento $R_1(t)= R_c(t_0) a= R'a$ por lo que $R_1(t)$ en la época presente es $R'$ más bien que la unidad. 
    Las métricas (\ref{eq:2.1.22}), (\ref{eq:2.1.23}) y (\ref{eq:2.1.24}) pueden definir el intervalo invariante $ds^2$ entre eventos en cualquier época o lugar en el Universo en expansión. Note además que, más adelante, será igualmente válido considerar unidades de $c=1$. 
    
    Finalmente, vamos a repasar algunos aspectos importantes:

        \begin{enumerate}

            \item El {\bf{tiempo cósmico $t$}} es el tiempo medido por un reloj llevado por el observador fundamental,
            \item {\bf{$r$ es la coordenada de distancia radial comóvil}} que está fija a una galaxia para siempre y que es la distancia propia que tendría la galaxia si sus líneas de mundo fueran proyectadas hacia adelante a la época $t_0$ y sus distancias medidas en ese tiempo. 
            \item $a(t)dr$ es el {\bf{elemento de distancia propia (geodésica) en la dirección radial en la época}} $t$. 
            \item $a(t) [R'\sin(r/R')] d\theta = a(t) r_1 d\theta$ es el {\bf{elemento de distancia propia perpendicular a la dirección radial subtendida por el ángulo}} $d\theta$ en el origen. 
            \item $a(t) [R'\sin(r/R')] \sin \theta d\phi = a(t) r_1 \sin \theta d\phi$ es el {\bf{elemento de distancia propia en la dirección}} $\phi$.
        
    \end{enumerate} 
    
      No sabemos nada acerca de la física que gobierna la tasa de expansión de Universo; sin embargo, $a(t)$ absorbe este fenómeno todavía desconocido. 
      

\section{Observaciones en Cosmología}

  \subsection{Corrimiento al Rojo Cosmológico} \label{sec:2.2.1} 

    Cuando hablamos del {\textit{corrimiento al rojo cosmológico}}, nos referimos al desplazamiento de las líneas espectrales hacia longitudes de onda más grandes asociadas a la expansión isotrópica del sistema de galaxias. Así, el corrimiento al rojo $z$ es definido como

        \begin{equation} \label{eq:2.2.1}
            z = \frac{\lambda_0 - \lambda_e}{\lambda_e},
        \end{equation}

    siendo $\lambda_0$ la longitud de onda observada y $\lambda_e$ la longitud de onda de la línea emitida. Si $z$ fuera interpretada como una velocidad de recesión $v$ de una galaxia, $z$ y $v$ serían relacionadas por el desplzamiento Doppler Newtoniano

        \begin{equation} \label{eq:2.2.2} 
            v= cz,
        \end{equation}

    que fue utilizada por Hubble para derivar la relación velocidad-distancia, $v=H_0 r$.
    
    En cosmología, el corrimiento al rojo tiene un significado más profundo. Por ejemplo, si consideramos un paquete de ondas de frecuencia $\nu_1$ emitido entre los tiempos cósmicos $t_1$ y $t_1 +  \Delta t_1$ de una galaxia distante, y es recibido por un observador en la época presente $t_0$ y $t_0 + \Delta t_0$, la señal se propaga a lo largo de {\bf{conos nulos}}, $ds^2=0$, y si además $d\theta = d\phi = 0$, la métrica (\ref{eq:2.1.22}) puede escribir como:
    
        \begin{equation} \label{eq:2.2.3}
            dt = - \frac{a(t)}{c} dr \hspace{0.4cm} \frac{c dt}{a(t)} = -dr,
        \end{equation} 
        
    con $a(t) dr$ el {\bf{intervalo de distancia propia en el tiempo cósmico}} $t$. Para el borde delantero del paquete de ondas 

        \begin{equation} \label{eq:2.2.4} 
            \int_{t_1}^{t_0} \frac{cdt}{a(t)} = - \int_r^0{dr}.
        \end{equation}
        
    El final del paquete de ondas debe viajar la misma distancia en unidades de la coordenada de distancia comovil ya que $r$ es fija a la galaxia para siempre. Así:

        \begin{align} 
            \int_{t_1 + \Delta t_1}^{t_0 + \Delta t_0}{ \frac{cdt}{a(t)}} & = - \int_r^0{dr}, \label{eq:2.2.5}  \\
            \int_{t_1}^{t_0}{ \frac{cdt}{a(t)} +  \frac{c \Delta t_0}{a(t_0)} -\frac{c \Delta t_1}{a(t_1) }}   & =  \int_{t_1}^{t_0}{ \frac{cdt}{a(t)}}. \label{eq:2.2.6} 
        \end{align}
     
     
    Y ya que $a(t_0) = 1$

        \begin{equation} \label{eq:2.2.7} 
            \boxed{\Delta t_0 = \frac{\Delta t_1}{a(t_1)}.}
        \end{equation}


    Esta es la expresión para el fenómeno de la {\bf{dilatación del tiempo}} (equivalente al fenómeno de la relatividad especial). Galaxias distantes son observadas en algún tiempo cósmico anterior $t_1 < 1$, por lo que se observa que los fenómenos tardan más en nuestro marco de referencia que en el de la fuente. Si $\Delta t_1 = \nu_1^{-1}$ es el periodo de las ondas emitidas y  $\Delta t_0 = \nu_0^{-1}$ el período observado, (\ref{eq:2.2.6}) puede ser escrita como
    
        \begin{equation} \label{eq:2.2.8} 
            \nu_0 = \nu_1(t_1) a(t_1),
        \end{equation}
        
    y haciendo uso de (\ref{eq:2.2.1}) y (\ref{eq:2.2.8})
    
        \begin{align}
            z & = \frac{\lambda_0 - \lambda_e}{\lambda_e} = \frac{\lambda_0 }{\lambda_e} - 1 = \frac{\nu_1}{\nu_0} - 1, \label{eq:2.2.9}  \\
            z + 1 & = \frac{\nu_1}{\nu_0} = \frac{1}{a(t_1)} \label{eq:2.2.10}, 
        \end{align}
        
    finalmente 
    
        \begin{equation} \label{eq:2.2.11}
            \boxed{a(t_1)  = \frac{1}{z+1}.} 
        \end{equation}    
        
    
    Esta última expresión representa un resultado muy importante es cosmología y es otra manera de expresar el corrimiento al rojo; una {\bf{medida del factor de escala del Universo cuando la fuente emitió la radiación}}. 
    Cuando observamos una galaxia con desplazamiento al rojo $z = 1$, el factor de escala del Universo cuando se emitió la luz fue $a (t) = 0.5$, i.e., las distancias entre observadores fundamentales (o galaxias) eran la mitad de sus valores actuales. 
    
    De (\ref{eq:2.2.4}) podemos obtener una expresión para la {\bf{coordenada de distancia radial comovil}} $r$:
    
        \begin{subequations} 

            \begin{equation} \label{eq:2.2.12a}
                r = \int_{t_1}^{t_0}{\frac{c dt }{a (t)}}.
            \end{equation} 

    $r$ {\bf{es una distancia artificial que depende de cómo el Universo se ha expandido entre la emisión y recepción de la radiación}}. En el caso de la {\textit{distancia física}}, esta es la distancia real entre dos puntos con coordenada cómovil $r_1$ y $r_2$  y la cual cambia con la expasión del Universo. Es dada por: 
    
    
    
        \begin{equation} \label{eq:2.2.12b}
            D = \int_{r_1}^{r_2}{dl},    
        \end{equation} 
            
        
     con $dl = a(t)/\sqrt{1-\kappa r^2}$ \, de (\ref{eq:2.1.23}), y siendo cero la parte angular. Sustituyendo $dl$ e integrando: 
     
        \begin{equation} \label{eq:2.2.12c}
            D(t) = a(t) \Delta r.
        \end{equation} 
            
     
    Para dos tiempos $t_a$ y $t_b$, si consideramos que la distancia comóvil no cambia con la expansión del Universo, la distancia física $D(t_a)$ puede ser calculada si además de conocer el factor de escala en los dos tiempo, $a(t_a)$ y $a(t_b)$, $D(t_b)$ es igualmente conocida. 
    
        \begin{equation} \label{eq:2.2.12d} 
            D(t_a) = \frac{a(t_a)}{a(t_b)} D(t_b).
        \end{equation}
    
    
    \end{subequations} 
    
    
    La dilatación del tiempo, expresión (\ref{eq:2.2.7}), está relacionada con las supernovas tipo 1a (SN 1a). La estrecha dispersión en sus magnitudes absolutas que tienen exactamente la misma curva de luz  (la variación en el tiempo en sus luminosidades durante la explosión de la SN) ha hecho que estos objetos sean herramientas útiles en cosmología. Las grandes luminosidades de las SN tipo 1a hace que estas puedas ser observadas a altos corrimiento al rojo. 
    
    \subsection{Ley de Hubble} 

    La ley de Hubble puede escribirse como: 


        \begin{equation} \label{eq:2.2.13} 
            \frac{\varrho}{dt} = H\varrho,
        \end{equation}

    con $\varrho$ la distancia propia, $\varrho = a(t) r$. Usamos $H$ en vez de $H_0$ porque vamos a considerar la {\bf{``constante de Hubble''}} en cualquier época y no en el presente. Sustituyendo $\varrho = a(t) r$,
    
        \begin{align}
            r\frac{da(t)}{dt} & = Ha(t)r,  \label{eq:2.2.14} \\
            \boxed{H = \frac{\dot{a}}{a}.} \label{eq:2.2.15} 
        \end{align}

    Si consideramos la constante de Hubble en el tiempo presente, $t= t_0$, $a=1$, entonces 
    
        \begin{equation} \label{eq:2.2.16}
            \boxed{H_0 = (\dot{a})_{t_0}.}
        \end{equation}
        
    La {\bf{constante de Hubble define la tasa de expansión del Universo en el presente}}. Para cualquier época podemos definir: 

        \begin{equation} \label{eq:2.2.17}
            \boxed{H(t) = \frac{\dot{a}}{a}.}
        \end{equation}
    
 \subsection{Diámetros angulares}

    En la sección anterior, (\ref{sec:2.2.1}) hemos considerado únicamente la parte radial. Siguiendo con la métrica de Roberton - Walker 

        $$ds^2 = dt^2 - \frac{a^2(t)}{c^2} [dr^2 + R'^2 \sin^2(r/R')(d \theta^2 + \sin^2 \theta d\phi^2)],$$
    
    consideraremos ahora el componente espacial relevante $d\theta$. La longitud propia $d$ de un objeto en el desplazamiento al rojo $z$, correspondiente al factor de escala $a(t)$ viene dado por el incremento de la longitud propia perpendicular a la dirección radial:
  
        \begin{align}
            d & = a(t) R' \sin \left(\frac{r}{R'} \right) \Delta \theta = a(t) D \Delta \theta = \frac{D \Delta \theta}{1+z}, \notag \\
            \Delta \theta & = \frac{d(1+z)}{D},\label{eq:2.2.18}
        \end{align}
        
    donde distinguimos una nueva {\textit{medida de distancia}} $D = R' \sin(r/R')$. Para pequeños $z$, $z << 1, r <<R'$ la ec. (\ref{eq:2.2.18}) se reduce a la relación euclidiana $d = r \Delta \theta$. Entonces también puede ser escrita como:

        \begin{equation} \label{eq:2.2.19} 
            \Delta \theta = \frac{d}{D_A},
        \end{equation}
        
    donde se ha definido una nueva medida $D_A$ conocida como {\bf{distancia de diámetro angular}}.
    
    Otro cálculo importante es el del {\bf{diámetro angular de un objeto que continúa participando en la expansión}} como es el caso de una perturbación infinitesimal en la expansión del Universo. Un ejemplo es el diámetro angular que estructuras a gran escala presentes en el Universo actual habrían subtendido en una época anterior, e.g., la recombinación, si simplemente se hubiesen expandido con el Universo. Este se usa para calcular tamaños físicos correspondientes a las escalas angulares de las fluctuaciones observadas en la radiación del CMB. Si el tamaño de un objeto es actualmente de $d(t_0)$ y se expandió con el Universo, su tamaño físico en el desplazamiento al rojo fue de

        $$d(t_0) a(t)=  \frac{d(t_0)}{(1+z)}$$,
    
    por lo tanto, ese objeto subtendió un ángulo

    \begin{equation} \label{eq:2.2.20}
        \Delta \theta = \frac{d(t_0)}{D}.
    \end{equation}
    
\subsection{Intensidades Aparente}

    Se tiene una fuente con luminosidad $L(\nu_1), (\mathrm{W Hz^{-1}})$, y con un corrimiento al rojo $z$. $L$ es la {\textit{energía total emitida sobre $4 \pi$ estereorradián por unidad de tiempo por unidad de intervalo de frecuencia}}. Siendo $\nu_0$ la frecuencia en la época presente $t_0$, queremos calcular la densidad de flujo $S(\nu_0)$ de la fuente en la frecuencia $\nu_0$ en que esta se observa. Específicamente, ¿cuál es la energía recibida por unidad de tiempo, por unidad de área, por unidad de ancho de banda, en unidades de $\mathrm{W m^{-2} Hz^{-1}}$. De ecs. (\ref{eq:2.2.8}) y (\ref{eq:2.2.11})
    
         $$\nu_0 = a(t_1) \nu_1 = \frac{1}{z+1}.$$
         
    Podemos suponer que la fuente emite $N(\nu_1)$ fotones con energía $h \nu_1$, en el ancho de banda $\nu_1, \nu_1 + \Delta \nu_1$ en el intervalo de tiempo propio $\Delta t_1$. De esta manera $L(\nu_1)$ es

        \begin{equation} \label{eq:2.2.21}
            L(\nu_1) = \frac{N(\nu_1) h \nu_1 }{\Delta \nu_1 \Delta t_1}.
        \end{equation}
        
    Imagine que la fuente, en un tiempo $t_1$, es el centro de una ``esfera'' y que los fotones que ha emitido están distribuidos sobre una ``capa'' que rodea a la esfera. Cuando esta capa de fotones llega al observador, en una época $t_0$, cierta cantidad de ellos son interceptados por un telescopio. En $t_0$, los fotones observados tienen frecuencia $\nu_0 = a(t_1) \nu_1$ (ec. \ref{eq:2.2.8}), un intervalo de tiempo propio $ \Delta t_0 = \Delta t_1/a(t_1)$ (ec. \ref{eq:2.2.37}) y en el ancho de banda $ \Delta \nu_0 = a(t_1) \nu_1$.
    
    Para conocer cuántos fotones están sobre esta ``capa'' que rodea la esfera entre la época $t_1$ y $t_0$ es necesario relacionar el diámetro del telescopio $\Delta l$  al diámetro angular $\Delta \theta$ que subtiende en la fuente en la época $t_1$. De ec. (\ref{eq:2.2.18}), si consideramos que para un tiempo presente, $t_0$, $a(t_0) = 1$
    
        \begin{equation} \label{eq:2.2.22}
            \Delta l = D \Delta \theta,
        \end{equation}
    
    siendo $\Delta \theta$ el ángulo medido por un observador fundamental, el cual se encuentra en la fuente. Si ahora introducimos la noción de ángulo sólido $d\Omega$, podemos considerar cómo los fotones que la fuente ha emitido se extienden sobre $d \Omega$, como se observa desde la fuente en la geometría curva. Si se supone que el Universo no se expande, el área superficial sobre la cual se observan los fotones en un tiempo $t$ después de su emisión se escribe como: 

        \begin{equation} \label{eq:2.2.23} 
            dA = R_c^2 \sin^2 \frac{x}{R_c} d\Omega,
        \end{equation}

     siendo $x = ct$. Por otro lado, si el Universo se expande, el radio de curvatura $R_c$ cambia como el Universo lo haga, y en lugar de $x/R_c$, tendríamos 
     
        \begin{equation} \label{eq:2.2.24} 
            \frac{1}{R'} \int_{t_1}^{t_0}{\frac{c dt}{a}} =     \frac{r}{R'},
        \end{equation}
     
     con $r$ la coordenada de distancial radial comóvil. Sustituyendo (\ref{eq:2.2.24}) en $dA$

    \begin{equation} \label{eq:2.2.25} 
        dA = R'^2 \sin^2 \frac{r}{R'} d\Omega.
    \end{equation} 
    
    Consideraremos el diámetro del telescopio como lo vemos desde la fuente como $\Delta l = D \Delta \theta$. Recordemos que las distancia comóviles toman en cuanta la expansión del Universo, a diferencia de las distancias propias. Por lo tanto, el área superficial del telescpio es $\pi \Delta l^2 / 4$ y el ángulo sólido subtendido por esta área superficial en la fuente es $d \Omega = \pi \Delta \theta^2/4$. El número de fotones incidentes sobre el telescopio en un tiempo $\Delta t_0$ es 
    
        \begin{equation} \label{eq:2.2.26} 
            N (\nu_1) \frac{\Delta \Omega}{4 \pi},
        \end{equation}
     
     
    donde ahora ellos son observados con frecuencia $\nu_0$. De esta manera, podemos conocer la expresión para la densidad de flujo de la fuente, i.e., la cantidad de energía recibida por unidad de tiempo, por unidad de área, por unidad de ancho de banda. Esto es
    
        \begin{equation} \label{eq:2.2.27} 
            S(\nu_0) = \frac{N (\nu_1) h \nu_0 \Delta \Omega} {4\pi \Delta t_0 \Delta \nu_0 (\pi/4)\Delta l^2 }.
        \end{equation}
    
    Haciendo uso, nuevamente, de las ecs. (\ref{eq:2.2.7}) y (\ref{eq:2.2.8}), además de las relaciones de la luminosidad, (\ref{eq:2.2.21}) , y el diámetro del telescopio, (\ref{eq:2.2.22}), podemos expresar (\ref{eq:2.2.28}) como:
    
        \begin{align}
            S( \nu_0) & = \frac{[N(\nu_1) h \nu_1 (t_1) a(t_1)] (\pi \Delta \theta^2/4)}{4 \pi [\Delta t_1/a(t_1)] [\nu_1 a(t_1)] (\pi/4)(D^2 \Delta \theta^2)}, \notag \\ 
            \\
            S(\nu_0) & = \frac{L(\nu_1) a(t_1)}{4\pi D^2} =\frac{L(\nu_1)}{4\pi D^2(1+z)}. \label{eq:2.2.29}
        \end{align}
    
    (\ref{eq:2.2.29}) relaciona la intensidad observada $S(\nu_0)$ a la luminosidad intrínseca de la fuente $L(\nu_1)$. 
    Para luminosidades y densidades de flujo {\bf{bolométricas}} se considera entonces la energía total emitida en un ancho de banda finito $\Delta \nu_1$ y que es recibida en el ancho de banda $\Delta \nu_0$. Se tiene
    
        \begin{align}
            L_{bol}= L (\nu_1) \nu_1 & = 4 \pi D^2 S(\nu_0)(1+z) \times \Delta \nu_0 (1+z), \label{eq:2.2.30} \\
            & = 4 \pi D^2 (1+z)^2 S_{bol}, \label{eq:2.2.31}
        \end{align}
        
    de donde se desprende que 
    
        \begin{align}
            S_{bol} & = S(\nu_0) \nu_0, \label{eq:2.2.32} \\
            S_{bol}  & = \frac{L_{bol}}{4 \pi D^2 (1+z)^2} =  \frac{L_{bol}}{4     \pi D_L^2}. \label{eq:2.2.33} 
        \end{align}
    
    Note la nueva cantidad $D_L =  D (1+z) $, la cual es conocida como la {\bf{distancia de luminosidad }} de la fuente. Esta cantidad $D_L$ que relaciona a $L_{bol}$ con $S_{bol}$ hace que la misma luzca como una ley del  del cuadrado inverso. 
    
    Es posible medir la densidad de flujo bolométrica en la presente época, $t_0$, si integramos la luminosidad bolométrica en cualquier ancho de banda adecuado siempre que se utilice el $z$  correspondiente. Así, 
    
        \begin{equation} \label{eq:2.2.34}
            \sum_{\nu_0} S(\nu_0) \Delta \nu_0 = \frac{ \sum_{\nu_1} L(\nu_1) \Delta \nu_1 }{4 \pi D^2 (1+z)^2} =   \frac{\sum_{\nu_1} L(\nu_1) \Delta \nu_1}{4 \pi D_L^2}
        \end{equation}
        
    Si se conoce el espectro de la fuente $L(\nu)$, esta última relación puede escribirse en términos de la luminosidad de la fuente en la frecuencia observada $\nu_0$. Esto es:
    
        \begin{equation} \label{eq:2.2.35} 
            S(\nu_0) = \frac{L(\nu_0)}{4\pi D_L^2} \left[ \frac{L(\nu_1)}{L(\nu_0)}(1+z) \right]. 
        \end{equation}
        
    El término entre corchetes se conoce como {\bf{Corrección-$K$}} y es utilizada para ``corregir'' las magnitudes aparentes de galaxias distantes por los efectos de desplazamiento al rojo de sus espectros cuando las observaciones son hechas a través de filtros estándar con una frecuencia de observación fija media $\nu_0$ (Sandage, 1961b). Así, la relación (\ref{eq:2.2.35}) puede escribirse en función de la magnitud absoluta, 
    
        $$M = cte - 2,5 \log_{10} L(\nu_0),$$ 
        
    y la aparente, 
    
        $$m = cte - 2,5 \log_{10} S(\nu_0),$$ 
    
    encontrándose que
    
        \begin{align}
            M & = m - 5 \log_{10} (D_L) - K(z) - 2.5 \log_{10} (4\pi), \hspace{0.9cm} \label{eq:2.2.36} \textrm{donde} \\
            K & = -2.5 \log_{10} \left[ \frac{L(\nu_1)}{L(\nu_0)}(1+z) \right]. \label{eq:2.2.37} 
        \end{align}
    
     Esta corrección-$K$ es correcta para densidades de flujo y luminosidades {\textit{monocromáticas}}.
     
\subsection{Densidades Numéricas}
    
     Se quiere conocer la cantidad de objetos en un intervalo de corrimiento al rojo $z, \, z + dz$, y varias cosas sabemos: existe una relación uno a uno entre $r$ y $z$, siendo $r$ la {\bf{coordenada de distancia propia radial definida en la presente época}}, resultados que ya fueron dados en sec. (\ref{sec:2.2.2}), por lo que ya se conoce el número de objetos en el intervalo de distancia de coordenadas radial cómovil $r, \, r+dr$. El diagrama de espacio - tiempo (cono de luz, fig. \ref{fig:2.6b}) muestra cómo podemos conocer el número de objetos por trabajar en base a los volúmenes comóviles para la presente época. Para este caso, el radio de curvatura de la geometría curva es $R'$, así que el volumen de la concha esférica de grosor $dr$ en la coordenada de distancia comóvil $r$ es
    
        \begin{equation} \label{eq:2.2.38} 
            dV = 4\pi R'^2 \sin^2 (r/R') dr = 4\pi D^2 dr,
        \end{equation}
    
      donde identificamos a $R'^2 \sin^2 (r/R') = D^2$. Si $N_0$ es la densidad de objetos en el espacio actual y su número es conservado como el Universo se expande
      
        \begin{equation} \label{eq:2.2.39}
            dN = N(z) dz = 4\pi N_0 D^2 dr. 
        \end{equation}

    Podemos afirmar que la ec. (\ref{eq:2.2.39}) da la densidad numérica de los objetos en el intervalo $z, z +dz$, asumiendo que la densidad numérica de objetos no cambia con la época cósmica. En el caso en que la densidad numérica de objetos sí cambiara con la época cósmica, e.g., existe una función que depende de $z$, $f(z)$, con $f(z=0) =1$ entonces la cantidad de objetos esperados en el intervalo $dz$ sería: 

    \begin{equation} \label{eq:2.2.40} 
        dN= n(z) dz = 4\pi N_0 F(z) D^2 dr
    \end{equation}
    
\subsection{Edad del Universo}

     Por último, para determinar la edad del Universo, etiquetada como $T_0$, vamos a recordar la ec. (\ref{eq:2.2.3}). Tenemos:
     
        \begin{equation} \label{eq:2.2.41}
            dt = - \frac{a(t)}{c} dr \hspace{0.4cm} \frac{c dt}{a(t)} = -dr,
        \end{equation} 

    y de esta manera, 

        \begin{equation} \label{eq:2.2.42} 
            T_0 = \int_0^{t_0}{dt } = \int_0^{r_{max}}{\frac{a(t) dr}{c}},
        \end{equation}
    
    donde $r_{{\text{máx}}}$ es la coordenada de la distancia comóvil correspondiente a $a=0$ y $z = \infty$.
    
%%%%%%%%%%%%%%%%%%%%%%%%%%%%%%%%%%%%%%%%%%%%%%%%%%%%%%%%%%%%%%%%%%%%%%%%%
    
\section{Modelos de Mundo de Friedmann}
    
    
  \subsection{Ecuaciones de Campo de Einstein}

     Recapitulemos el {\textit{Principio Cosmológico}} y el {\textit{Postulado de Weyl}} (\ref{sec:2.2.1}), además de la {\textit{Teoría de la Relatividad General}}.
   
        \begin{itemize}
            \item El principio cosmológico, el cual establece un Universo isotrópico, homogéneo y que además se expande uniformente a gran escala. ESto nos conduce también a la {\textit{métrica de Walker -  Robertson}}, (\ref{eq:2.1.22}). 
            \item Weyl postula que existe una única línea de mundo que pasa a través de cada punto en el espacio - tiempo. Líneas que provienen de una singularidad en el pasado finito o infinito. Podemos entonces hablar de un fluido que se mueve a lo largo de estas líneas al ritmo en el que el Universo se expande, entonces su comportamiento es como el de un fluido perfecto, donde el {\textit{tensor de energía - momento}} es 
            
                 \begin{equation} \label{eq:2.3.1} 
                    T_{\alpha \beta} = (\varrho_0 + p) u^{\alpha} u^{\beta} - p g^{\alpha \beta},
                \end{equation}
            
            siendo $g^{\alpha \beta}$ el tensor métrica, $\varrho_0$ la densidad de masa propia del polvo,  $u^{\alpha}, u^{\beta}$ los cuadrivectores de la velocidad y $p$ el trimomento del polvo. 
       
            \item Por último, la relatividad general que relaciona el tensor energía - momento a las propiedades geométricas del espacio - tiempo a través de: 
            
            
                \begin{empheq}[box=\fbox]{align} 
                    R_{\mu \nu } - \frac{1}{2} g_{\mu \nu} R & =  8 \pi G   T_{\mu \nu},  \label{eq:2.3.2} \\ 
                    \notag  \\ 
                     R_{\mu \nu } - \frac{1}{2} g_{\mu \nu} R + \Lambda  g_{\mu \nu} & =  8 \pi G   T_{\mu \nu}, \label{eq:2.3.3}
                \end{empheq}
                
            donde  $R_{\mu \nu }$ es el {\textit{tensor de Ricci}}, $R$ el {\textit{escalar de Ricci o curvatura escalar}}, $G$ la constante de Gravitación Universal, $c$ la velocidad de la luz y $\Lambda$ la famosa {\textit{constante cosmológica}}, introducida por Einstein en 1917 con la intención de crear un Universo estático con geometría cerrada y que permitiera la incorporación del {\textit{Principio de Mach}} a la relatividad general (Einstein, 1917). Ernest Mach fue uno de los más influyentes críticos de los conceptos de espacio y tiempo absolutos Newtonianos. Este principio establecía que todas las fuerzas inerciales son debido a las distribuciones de materia en el Universo, el cual publicó luego en su {\textit{``Mechanics''}} y posteriormente Einstein llamó Principio de Mach (para una explicación más amplia consulte: Harrison E., "The Science of the Universe", pp: 236 - 239).
            
        \end{itemize} 

    Estos tres ingredientes son fundamentales para la construcción de modelos cosmológicos estándar. La relatividad general le permitió a Einstein construir modelos coherentes del Universo.
    
 \subsubsection{Ecuaciones de Friedmann}
    
    La isotropía y homogeneidad del Universo implicaron grandes simplificaciones a las ecuaciones de campo de Einstein, ecs. (\ref{eq:2.3.2}) y (\ref{eq:2.3.3}) reduciéndose entonces a este par de ecuaciones:
    
         \begin{empheq}[box=\fbox]{align}
            \ddot{a} = - \frac{4 \pi G}{3} a \left( \varrho + 3p \right) + \frac{1}{3} \Lambda a, \label{eq:2.3.4} \\
            \notag \\ 
            \dot{a}^2 =  \frac{8 \pi G \varrho}{3} a^2 - \frac{1}{R'^2} + \frac{1}{3} \Lambda a^2, \label{eq:2.3.5}
        \end{empheq}
        
    con $a$ el factor de escala normalizado a la época actual $t_0$, $\varrho$ la densidad de masa inercial total del contenido de materia y radiación en el Universo y $p$ está asociado con  la presión. Recordando, además, que $R'$ es la curvatura de la geometría del modelo de mundo en la época presente $t_0$, por lo tanto, $1/R'^2$ es una constante de integración. 
    
    Las ecuaciones de Friedmann describen la dinámica de un Universo isotrópico y homogéneo. Fueron derivada primero por la relatividad general, pero, ¿por qué no con la teoría de Newton? El problema está en que la mecánica clásica es una teoría global que incluye un potencial gravitatorio que diverge en un Universo isotrópico y homegéneo, a diferencia de la relatividad general que es una teoría local.
    
 \subsubsection{Derivación Newtoniana de la Ecuaciones de Friedmann}
 
     En mecánicla clásica es importante recordar que la noción de conservación de la masa y de la energía son independientes. En la cosmología Newtoniana para referirnos a distancias podemos considerar un fluido cósmico pero teniendo en cuenta que no estamos en un marco de referencia inercial. Así, al momento de aplicar la segunda ley de Newton tenemos que considerar que el punto de referencia puede cambiar. De esta manera, la distancia relativa viene dada por $R(t) = a(t) R$.
     
     La masa  encerrada $M_{int}$ dentro de una esfera de radio $R(t)$ es: 
    
        \begin{equation} \label{eq:2.3.6}
            M_{int} = \int_M{dm} = \frac{4\pi}{3} \rho(t) a^3(t) R^3.
        \end{equation}
    
     Imponiendo que $dM/dt=0$
    
        \[
            \frac{dM}{dt}   = 0 =  \frac{4\pi}{3} \left( \rho(t)3 a^2(t) \dot{a}(t)  R^3 + a^3(t) R^3 \dot{\rho} \right), 
        \]
    
        \begin{equation} \label{eq:2.3.7} 
            \boxed{\dot{\rho} = -3 \rho(t) \frac{\dot{a}}{a}.}
        \end{equation}
        
    Nos referimos a ec. (\ref{eq:2.3.7}) como la {\textit{ecuación de Friedmann Newtoniana de la conservación de masa}}. Este resultado tiene sentido ya que en mecánica clásica el campo no transporta energía por lo que el campo gravitatorio no ejerce presión. 
    
    
    Por el teorema de Gauss, el potencial gravitatorio es: 
    
        \begin{equation} \label{eq:2.3.8} 
            V_{grav} = -4 \pi G \int_0^{R(t)}{\frac{\rho(t)r^2}{r}dr} = -4 \pi G \rho(t) \frac{R^2(t)}{2},
        \end{equation}
        
    donde $R(t)$ es la distancia relativa entre dos puntos del Universo.
    
    Por otro lado, la fuerza gravitatoria que un punto de masa $m$ está experimentando es: 
    
        \begin{equation} \label{eq:2.3.9}
            \vec{F} = -m \vec{\nabla} V_{grav} = - G \frac{M_{int}}{R^2(t)} m \hat{r}.
        \end{equation}
        
     De la segunda ley de Newton $\vec{F} = m \vec{a}$, sustituyendo ecs. (\ref{eq:2.3.6}) y (\ref{eq:2.3.9}),
    
        \begin{equation} \label{eq:2.3.10}
            m \frac{d^2 a(t)R}{dt^2} = - Gm \frac{M_{int}}{a^2(t)R^2} = -Gm \frac{4\pi}{3}\frac{a^3(t)R^3}{a^2(t)R^2}\rho(t).
        \end{equation}
    
     Finalmente, 
    
        \begin{equation} \label{eq:2.3.11}
            \boxed{\frac{\ddot{a}(t)}{a(t)} = - \frac{4\pi G}{3}  \rho(t).}
        \end{equation}
    
    
    De esta manera, se ha obtenido la {\textit{ecuación de aceleración de Friedmann Newtoniana}} que implica un Universo no estático. 
    
    Las ecuaciones de conservación de la masa y aceleración, (\ref{eq:2.3.7} y (\ref{eq:2.3.11}), son independientes por lo que tenemos solo dos ecuaciones de Friedmann independientes lineales.
    
    {\large{\bf{Conservación de la Energía}}}
    
    Estas ecuaciones tienen oculto el principio de conservación de la energía. Si tomamos (\ref{eq:2.3.11}) y la integramos, considerando que $R(t) = a(t) R$: 
    
        \begin{equation} \label{eq:2.3.12} 
            \ddot{R}(t) = - \frac{4\pi G}{3} \rho(t) R(t)   = - G \frac{M_{int} }{R^2(t)}.
        \end{equation}
    
    Multiplicando por $\dot{R}(t)$ e integrando 
    
        \[
            \dot{R}(t) \left( \ddot{R}(t) = - G \frac{M_{int} }{R^2(t)} \right),
        \]
        
        \begin{equation} \label{eq:2.3.13} 
            \frac{1}{2}  \dot{R}^2(t)  = G \frac{M_{int} }{R(t)} + U,
        \end{equation}
    
    la cual tiene la forma de la ecuación de la conservación de la energía. Manipulándola, llegamos a 
    
        \begin{equation} \label{eq:2.3.14} 
            \boxed{\left(\frac{\dot{a}(t)}{a(t)} \right)^2 = \frac{8\pi G}{3} \rho(t) + \frac{2U}{R^2 a^2(t)}.}
        \end{equation}
    
    {\large{\bf{Con Constante Cosmológica $\Lambda$}}}
    
    En la teoría clásica, $\Lambda$ no aparece de manera natural, pero gracias a las observaciones actuales sabemos que esta actúa como una fuerza repulsiva proporcional a la distancia radial que sufre la masa $m$. Agredando esto a la segunda ley de Newton, (\ref{eq:2.3.10}), se tiene:
    
        \begin{equation} \label{eq:2.3.15}
            m \frac{d^2 a(t)R}{dt^2} = -Gm \frac{4\pi}{3}\frac{a^3(t)R^3}{a^2(t)R^2}\rho(t) + \frac{\Lambda}{3} a(t) mR,
        \end{equation}
    
    donde el $1/3$ es arbitrario en $\Lambda$. Luego de algo de álgebra, obtenemos la ecuación de aceleración: 
    
        \begin{equation} \label{eq:2.3.16}
            \boxed{\frac{\ddot{a}(t)}{a(t)} = - \frac{4\pi G}{3}  \rho(t) + \frac{\Lambda}{3}.} 
        \end{equation}
    
    La constante de integración $U$ obtenida en (\ref{eq:2.3.13}) seguirá siendo la energía mecánica. 
        
     
    
    \begin{equation} \label{eq:2.3.17}
        \boxed{\left(\frac{\dot{a}(t)}{a(t)} \right)^2 = \frac{8\pi G}{3} \rho(t) + \frac{\Lambda}{3} +  \frac{2U}{R^2 a^2(t)}.}
    \end{equation}
    
     \subsubsection{Derivación Relativista de la Ecuaciones de Friedmann}
     
     
     Retomando el conjunto de ecs. de Friedmann, usando la métrica de Walker - Robertson en su forma (\ref{eq:2.1.23}), la cual considera el principio cosmológico y la expansión del Universo, podemos calcular todos los elementos contenidos en las ecs. de Einstein, (\ref{eq:2.1.6}). En el caso del tensor de momento -  energía se considera un fluido perfecto, un fluido isotrópico, i.e., luce igual en cada dirección en la que nos movemos, y el cual tiene la forma $T_{\nu}^{\mu} = {\textrm{Diag}}(- \rho, p, p, p)$, donde $\rho$ y $p$ vienen siendo la densidad de energía y densidad de presión del fluido, respectivamente.  
     
     Ya que el tensor $T_{\nu}^{\mu}$ es conservado por virtud de las identidades de Bianchi, una tercera ecuación independiente, i.e., derivable de las otras dos (\ref{eq:2.3.4}) y (\ref{eq:2.3.5}), puede ser obtenida:

        $$\dot{\rho} + 3H(\varrho + p) =0$$,

    la cual es una ecuación de continuidad pero con la masa de flujo $\mathbf{j} = (\rho +  P)\mathbf{v}$ (Copeland, Sami, \& Tsujikawa (2006). Esta ecuación puede ser escrita también como una declaración de la conservación de la energía. De la primera ley de la Termodinámica
    
        \begin{equation} \label{eq:2.3.18}
            dQ = 0 = dU + pdV,
        \end{equation}
        
    entonces el cambio en la energía (negativo) depende solo del trabajo hecho por el sistema. $U$ es la energía interna y es igual a la suma de todas las energías que contribuyen a la energía total (energía cinética, energía térmica, etc.), $ \varepsilon_{tot} = \sum_i \varepsilon_i,$ en un fluido en el marco relativista. Con $U = \varepsilon_{tot} V$ la energía contenida en el volumen físico $V$ (reescalado como $a^3$). 
    
    
        \begin{equation} \label{eq:2.3.19}
            dU = - pdV \rightarrow{\frac{d}{dt} (\rho a^3) = -p \frac{d}{dt} a^3 }.
        \end{equation} 
        
    Siendo $\dot{a} = da/dt$ y $d\varrho/da = - 3 (\varrho + p)/a$, si derivamos (\ref{eq:2.3.5}) y dividimos por $\dot{a}$, entonces recobramos la (\ref{eq:2.3.4}), la cual tiene la forma de una ecuación de fuerza que contiene implicitamente la Primera Ley de la Termodinámica, que es posible derivar de la mecánica Newtoniana como ya vimos en (\ref{eq:2.3.16}), solo que el término de presión $p/c^2$ no está contenido en ella. Esta presión puede ser interpretada como una ``corrección'' a la densidad de masa inercial, que es diferente a la fuerza de presión que, e.g., sostiene a las estrellas. Por otro lado, el término $\varrho + 3p/c^2$ puede leerse como la {\textit{densidad de masa gravitatoria activa}}.
    
    Friedmann obtuvo la solución general de (\ref{eq:2.3.5}) para modelos de mundo en expansión (Fridmann A. A., 1922, 1924), donde asumió que la constante cosmológica era diferente de cero ($\Lambda \neq 0$). De esta manera, se le llama {\textit{Modelos de Mundo de Friedmann}} ya que estos son capaces de incluir o no el término $\Lambda$. Friedmann murió en 1925 y nunca supo que los modelos de mundo llevarían su nombre. Cuando George Lema$\hat{i}$tre redescubrió sus soluciones en 1927 logró resaltar las grandes contribuciones de Friendmann y la atención de astrónomos y cosmólogos en 1930 (Lema$\hat{i}$tre, 1927). 
    
 
 
 \section{Modelos de Mundo de Friedmann Estándar con $\Lambda = 0$} \label{sec:2.4} 
 
    Cuando en cosmología se habla de {\textit{polvo}} se refiere a un fluido sin presión y por lo tanto $p=0$ en las ecuaciones de Friedmann. Comencemos por estudiar el caso en el que $\Lambda=0$. Por conveniencia, se tomará el valor de la densidad del fluido en la época actual $\varrho_0$ y por conservación de la masa $\varrho=\varrho_0 a^{-3}$. Bajo estas suposiciones, las ecuaciones (\ref{eq:2.3.4} y (\ref{eq:2.3.5}) se reducen a: 
 
        \begin{empheq}[box=\fbox]{align}
            \ddot{a} & = - \frac{4\pi G \varrho_0  }{3 a^2}, \label{eq:2.4.1} \\
            \notag \\
            \dot{a}^2 & = \frac{8\pi G \varrho_0  }{3 a} - \frac{1}{R'^2}. \label{eq:2.4.2}
        \end{empheq}
 
     Por conveniencia, se va a definir una {\textit{densidad crítica $\varrho_c$}} para la densidad de los modelos de mundo. Así:
   
        \begin{equation} \label{eq:2.4.3} 
            \varrho_c = \frac{3H_0^2}{8\pi G} = 1.88 \times 10^{-26} \mathrm{h^2 kg m^{-3}},
        \end{equation}
        
    siendo la constante de Hubble $H_0$ con un valor de $100 \, \mathrm{h \, km\, s^{-1} Mpc^{-1}}$ {\footnote{${\mathrm{h}}= H_0/(100 \, km \, s^{-1}\, Mpc^{-1})$}. En la época presente, la densidad del modelo $\varrho_0$ puede referirse al valor de la densidad crítica a través de lo que se conoce como {\textit{parámetro de densidad} $\Omega_0 = \varrho_0/\varrho_c$, donde este parámetro ha sido definido como
    
        \begin{equation} \label{eq:2.4.4} 
            \Omega_0 = \frac{\varrho_0}{\varrho_c} = \frac{8 \pi G \varrho_0}{3 H_0^2}.
        \end{equation}
        
    Así mismo, el parámetro de densidad bariónico es $\Omega_B$, el de la materia ordinario, o luminosa, $\Omega_{matt}$, a de la materia oscura (DM) como $\Omega_{DM}$. Estos parámetros representan las contribuciones a $\Omega_0$.  De (\ref{eq:2.4.3}) y (\ref{eq:2.4.4}), podemos reescribir las ecs. (\ref{eq:2.4.1}) y (\ref{eq:2.4.2}) como: 
    
         \begin{empheq}[box=\fbox]{align}
            \ddot{a} & = - \frac{\Omega_0 H_0^2 }{ 2a^2}, \label{eq:2.4.5}\\
            \notag \\
            \dot{a}^2 & = \frac{\Omega_0 H_0^2  }{a} - \frac{1}{R'^2}. \label{eq:2.4.6}
        \end{empheq}
        
    Con respecto a esta última ecuación, si establecemos los valores en la época actual $t=t_0$, entonces $a=1$ y $\dot{a} = H_0$, encontramos qué: 
    
        \begin{equation} \label{eq:2.4.7} 
            \boxed{ R' = \frac{1/H_0}{(\Omega_0 - 1)^{1/2}},}
        \end{equation}
        
    y recordando de secciones anteriores el radio de curvatura, ec. (\ref{eq:2.1.21}): $R_c = a R'$, además de la curvatura del espacio $\kappa = R_c^{-2}$, obtenemos:
    
    \begin{equation} \label{eq:2.4.8} 
        \boxed{\kappa = \frac{\Omega_0 - 1}{(1/H_0)^2}.}
    \end{equation}
    
    Existe una relación uno a uno entre la densidad del Universo $\Omega_0$ y la curvatura del espacio $\kappa$; uno de los grandes resultados de los Modelos de Mundo de Friedmann cuando la constante cosmológica $\Lambda=0$.
        
 
 \subsection{Dinámica de los Modelos de Friedmann con $\Lambda = 0$} \label{sec:2.4.1}
 
    Si incluimos la relación (\ref{eq:2.4.7}) en (\ref{eq:2.4.6}) 
    
        \begin{equation} \label{eq:2.4.9}
            \boxed{\dot{a}^2 = H_o^2 \left[ \Omega_0 \left( \frac{1}{a} - 1 \right) + 1 \right].}
        \end{equation}
    
    En el caso de $a >> 1$, $\dot{a}^2$ se escribe como
    
        \begin{equation} \label{eq:2.4.10}
            \boxed{ \dot{a}^2  = H_o^2 (1- \Omega_0).}
        \end{equation}
    
    Y de esta últiam ecuación, se desprende tres casos.
    
   
 
        \begin{itemize}
            \item[i.] Modelos con un parámetro de densidad $\Omega_0 < 1$, los cuales tienen una geometría hiperbólica abierta y se expanden a $a=\infty$. Donde 
            
                $$\dot{a}  = H_o (1- \Omega_0)^{1/2}.$$
        
            \item[ii.] También, modelos con un parámetro $\Omega_0 >1$. Aquellos con geometría esférica cerrada. Dejan de expandirse en algún valor finito de $a$, entonces $a=a_{max}$ (en el infinito tienen ``tasas de expansión imaginarias''), el cual alcanzan en un tiempo máximo 
            
                $$t_{max} = \frac{\pi \Omega_0}{2 H_0 (\Omega_0 - 1)^{3/2}}.$$
        
            \item[iii.] Por último, tenemos el caso cuando el parámetro de densidad $\Omega_0 = 1$, cuyo modelo tiene una velocidad de expansión que tiende a cero cuando $a \rightarrow{\infty}$. 
 
            Se conoce como {\textit{Modelo Einstein- de Sitter}} o {\textit{modelo crítico}}. Plantea un Universo que no colapsa ni tampoco se expande por siempre. El valor del factor de escala varía con el tiempo cósmico como:

                \begin{equation} \label{eq:2.4.11} 
                    a(t) ) \left( \frac{t}{t_0} \right)^{2/3},
                \end{equation} 

            donde $\kappa = 0$ y en la época presente                   
                $$t_0 = (2/3) H_0^{2/3}$$.
    
        \end{itemize}
        
        \newpage
        
        
        \begin{wrapfigure}[21]{r}{0.5\linewidth}        \includegraphics[width=0.5\textwidth]{grafica_scale_factor_timepp219Malcolm.png}
            \caption{\footnotesize \small{$\Omega = \varrho_0/\varrho_c$. Para  $\Omega_0 > 1$, el Universo colapsa a $a=0$. Para $\Omega_0 <1$, el Universo se expande hasta el infinito y la velocidad de expansión es infinita como $a \rightarrow{\infty}$ . Por último, $\Omega_=1$. Para el del tiempo, este viene en términos del tiempo adimensional $H_0 t$. Siendo $t_0$ es el tiempo actual, cuando $\Omega_0 = 0$, $H_0 t_0=1$; $\Omega_0 = 1$, $ H_0 t_0 = 2/3$ y para $\Omega_0 = 2$, $H_0 t_0 = 0.57$. Las tres curvas tienen la misma pendiente $1$ para $a(1)$.}}
            \label{fig:2.4}
    \end{wrapfigure}
    
    En la figura (\ref{fig:2.4}) se muestran algunas soluciones a la ec. (\ref{eq:2.4.9}) el cual ilustra la relación entre la dinámica y geometría de los modelos de Friedmann sin constante cosmológica. El eje $x$ viene en unidades de $H_0^{-1}$. Nótese la línea de la presente época $a=1$.  Las pendientes en ese punto son siempre $1$. Así mismo, la edad actual del Universo para cada parámetro es dada por la intersección de cada curva con la línea en $a=1$. Se aprecia el caso en que $\Omega_0 = 0$, un modelo de mundo vacío también conocido como {\textit{Modelo de Milne}} donde $a(t) = H_0 t$ y $\kappa = - (H_0/c)^2$. Milne construyó su propia teoría conocida como {\textit{relatividad cinemática}}, donde la gravedad no es un elemento principal, y basado en el principio cosmológico y la relatividad especial crea una visión que explica la naturaleza de la gravedad y otras leyes físicas.  Su Universo consiste de una nube esférica de partículas (galaxias) las cuales se expanden dentro de un espacio plano, que es infinito y por lo tanto vacío.  Las partículas colisionan a velocidades próximas a la de la luz. Lo mismo ocurre con la expansión del borde del Universo (Milne, E.A, 1935: E.A. Milne Relativity Gravitation and World Structure Oxford University Press 1935). Este Universo obedece la métrica de Walker -  Robertson (ec. \ref{eq:2.1.22}), tiene geometría global hiperbólica y los tiempos cósmicos medidos en diferentes marcos de referencia están relacionados por las transformación de Lorentz estándar $t’ = \gamma(t – rv/c^2)$, siendo $\gamma = (1-v^2/c^2)^{-1/2}$.
    
    
\subsection{Modelos de Friedmann con Constante Cosmológica}   
    
  \subsubsection{Constante Cosmológica y la Densidad de Energía del Vacío}
    
    Einstein asumía que la estructura a gran escala del Universo era estática, entonces introdujo, de manera más bien {\textit{ad hoc}},  la constante cosmológica para reconciliar esta visión con su teoría de la relatividad general. $\Lambda$ aparecía en sus ecuaciones de campo como una constante. McVittie (1956) y otros, consideraban que $\Lambda$ era una constante de integración. McVittie (1956) y otros, consideraban que $\Lambda$ era una constante de integración, y por ende no podía tener cualquier valor ((McVittie, G. C., 1956, General Relativity and Cosmology (Chapman and Hall, London). En 1933, Lema\^itre sugiere que $\Lambda$ puede ser interpretada como una {\textit{densidad de energía del vacío}} (Lema\^itre, 1933), idea que desde el punto de vista de la física de partículas surge naturalmente. Además, la escala de energía de $\Lambda$  debería ser mucho mayor que la de la constante de Hubble en el presente $H_0$ siempre que, se origine a partir de la densidad de energía del vacío. Esto es conocido como el ``problema de la constante cosmológica'' (S. Weinberg, Rev. Mod. Phys. 61, 1 (1989)). 
    
    Por supuesto que desde diferentes enfoques de la física se ha intentado buscar una ``solución''. Esto incluye gravedad cambiante (e.g., Van der Bij, 1982; W. Buchmuller, 1988), gravedad cuántica (Baum. E, 1983; Hawking S., 1984; Coleman S. R., 1988), teoría de cuerdas (Kachru S. et al., 2003), aproximación de la espuma espacio - tiempo (Garattini R., 2002) y fluctuaciones de vacío de la densidad de energía (Padmanabhan T., 2005; Gurzadyan V. G. \& Xue S. S., 2003).  

    El lado derecho en ec. (\ref{eq:2.3.2}) es también definido como el tensor de Einstein $G_{\mu \nu}$ el cual, junto con el tensor $T^{\mu \nu}$ satisfacen las identidades de Biachi $\nabla_{\nu} G^{\mu \nu} = 0$ y la conservación de la energía $\nabla_{\nu} T^{\mu \nu}=0$. Debido a que la métrica $g^{\mu \nu}$ es constante con respecto a derivadas covariantes ( $\nabla_{\alpha} g^{\mu \nu} = 0$), existe libertad de agregar un término $\Lambda g_{\mu \nu} $ en las ecs. de Einstein, obteniéndose (\ref{eq:2.3.3}) (para un mejor desarrollo, ver Dadhich N., 2004). Tomando traza sobre ella se obtiene $- R +4 \Lambda = 8 \pi G T$. Combinando este resultado de nuevo con (\ref{eq:2.3.3}) se obtiene:
   
        \begin{equation} \label{eq:2.4.12} 
            R_{\mu \nu } - \Lambda  g_{\mu \nu} =  8 \pi G \left(   T_{\mu \nu} - \frac{1}{2} T g_{\mu \nu} \right).
        \end{equation}
 
    Considerando gravedad en el marco Newtoniano con $g_{\mu \nu} = \eta_{\mu \nu} + h_{\mu \nu }$, siendo $h_{\mu \nu }$ la perturbación alrededor de la métrica de Minkowski $\eta_{\mu \nu}$. Calculando la componente $R_{00}$ en (\ref{eq:2.4.12}), la cual puede ser reescrita como un potencial gravitatorio $\Phi$, se obtiene  
    
        \begin{equation}  \label{eq:2.4.13} 
            \Delta \Phi = 4 \pi G \rho - \Lambda. 
        \end{equation}
    
      Para reproducir la ecuación de Poisson Newtoniana, la constante cosmologica debe ser cero o tener un valor muy pequeño en comparación con $4 \pi G \rho$ para poder ser despreciada.  $\Lambda$ tiene unidades de $[\textrm{longitud}]^{-2}]$ por lo que la escala correspondiente tiene que ser mucho mayor que la escala de objetos estelares en la cual la gravedad Newtoniana funcione correctamente. Esto significa que $\Lambda$ se convierte en importante a escalas muy grandes. 
      
      Las ecuaciones de Einstein modificadas entonces en el contexto de la métrica de Walker - Robertson son:
      
        \begin{empheq}[box=\fbox]{align} 
            \ddot{a} = - \frac{4 \pi G}{3} a \left( \varrho + 3p \right) + \frac{1}{3} \Lambda a, \label{eq:2.4.14}  \\
            \notag \\ 
            \dot{a}^2 =  \frac{8 \pi G \varrho}{3} a^2 - \frac{1}{R'^2} + \frac{1}{3} \Lambda a^2, \label{eq:2.4.15} 
        \end{empheq}
 
    
    Esto demuestra que la constante cosmológica contribuye negativamente a la presión por lo que exhibe un efecto repulsivo. 
    Si consideramos un Universo lleno de polvo, $3p = 0$, simplificándose (\ref{eq:2.4.14}) a:
    
        \begin{equation} \label{eq:2.4.16} 
            \ddot{a} = - \frac{4 \pi G a \varrho  }{3}+ \frac{1}{3} \Lambda a =- \frac{4 \pi G\varrho_0 }{3a^2}+ \frac{1}{3} \Lambda a.
        \end{equation} 
 
     Así se considere un Universo vacío ($\varrho=0$), hay una fuerza neta que actúa sobre una partícula de prueba. Si además de vacío es estático, como el que Einstein propuso, 
    
        \begin{equation} \label{eq:2.4.17} 
            \varrho = \frac{\Lambda}{4 \pi G}, \hspace{0.8cm} \Lambda = \frac{1}{R' a^2}.
        \end{equation}
    
    La idea de un Universo estático fue abandonada con el descubrimiento del corrimiento al rojo de estrellas distantes. 
    
    Si la constante cosmológica surge de la idea de una densidad de energía del vacío, entonces existe un grave problema de {\textit{ajuste fino}}. De las observaciones, $\Lambda$ tiene un valor del orden del valor actual deL parámetro de Hubble, $H_0$
    
        \begin{equation} \label{eq:2.4.18} 
            \Lambda \approx H_0^2 = (2.13h \times 10^{-42} \, \mathrm{GeV})^2,
        \end{equation}
 
    
    que corresponde a una densidad crítica $\varrho_{\Lambda}$ 
    
        \begin{equation} \label{eq:2.4.19} 
            \varrho_{\Lambda} = \frac{\Lambda m_{pl}^2}{8\pi} \approx 10^{-47} \, \mathrm{GeV^4},
        \end{equation}
 
    con $m_{pl} = 1.22 \times 10^{19} \, \mathrm{GeV} $ la masa de Planck. Desde el punto de vista de la teoría cuántica de campos, se puede hacer un análisis para estimar la densidad de energía de los campos vacíos.  La {\textit{energía del punto cero}} de campos constribuye con la densidad de la energía oscura. Así, la densidad de energía del vacío evaluada por la suma de de las energía del punto cero de campos cuánticos con masa $m$ es dada por 
    
        \begin{align}
            \varrho_{vac} & = \frac{1}{2}\int_0^{\infty}{\frac{d^3{\mathbf{k} }}{(2\pi)^3} \sqrt{k^2 + m^2}}, \\
            & = \frac{1}{4 \pi^2}\int_0^{\infty}{dk \, k^2 \sqrt{k^2 + m^2}}. \label{eq:2.4.21}
        \end{align}
 
    Esta integral muestra una divergencia ultravioleta: $\varrho_{vac} \propto k^4$. La suma sobre las energías de modo de punto cero debe ser limitada a una frecuencia alta (o distancia corta) hasta la cual el modelo físico tenga sentido. Así, la integral de la energía de punto cero de modos normales (de número de onda $k$) hasta un número de onda máximo $k_{max}$
 
        \begin{equation} \label{eq:2.4.22}
             \varrho_{vac} =  \hbar \frac{k^4_{max}}{16 \pi^2}.
        \end{equation}
 
    Para el caso de la relatividad general, el valor esperado es justo por debado de la escala de la masa de Planck. Por lo tanto, si $k_{max} = m_{pl}$, el valor de la densidad de energía del vacío es del orden de 
    
        \begin{equation} \label{eq:2.4.23}
            \varrho \approx 10^{74} \, \mathrm{GeV^4},
        \end{equation}
    
    valor que es alrededor de $10^{121}$ órdenes más grande que el valor observado (\ref{eq:2.4.19}). 
    
    Con el desarrollo de la idea del rompimiento de simetría en el modelo estándar de partículas, se planteó una relación de la expansión y enfriamiento del Universo con una secuencia de transiciones de fase que acompañan al rompimiento de simetría. Cada transición de fase de primer orden tiene como una contribución a una constante cosmológica dependiente del tiempo $\Lambda(t)$ o densidad de energía oscura. La disminución del valor de esta densidad de energía oscura en cada transición de fase es mayor que un valor presente. Una posible respuesta es que la energía oscura ahora es despreciada, lo cual parece descabellado pero condujo al escenario de la {\textit{inflación}} que tomó lugar en el Universo muy temprano ($\sim 10^{-36} - 10^{-33} s$). Desde esta visión, este es exactamente el tipo de campo que condujo la expansión inflacionaria. Se debe encontrar una explicación al hecho de que $\varrho_v$ decrece por un factor $10^{120}$ al final de la era {\textit{inflacionaria}}. Un valor de $10^{-120}$ es un valor tan pequeño que podría considerarse que $\Lambda=0$ en el modelo de Friedmann. 
    
    Es posible relacionar una densidad de masa $\varrho_m$ con la densidad de energía del vacío o la energía oscura en la presente época. Consideremos entonces un parámetro de densidad relacionada con la energía oscura $\Omega_{\Lambda}$. Reescribiendo (\ref{eq:2.4.14}):
    
        \begin{equation} \label{eq:2.4.24}
            \ddot{a} = - \frac{4 \pi G a}{3}  \left( \varrho_m -2\varrho_v \right).
        \end{equation}
        
    donde estamos considerando la densidad de `polvo' $\varrho_{m}$, la densidad del vacío $\varrho_v$ y la presión $p_v = - \varrho_v c^2$. En este caso, $\varrho_m = \varrho_0/a^3$, mientras $\varrho_v = cte$.Y sustituyendo
    
        \begin{align} 
            \ddot{a} & = - \frac{4 \pi G \varrho_0}{3 a^2} + \frac{8 \pi G \varrho_v a}{3}, \label{eq:2.4.25} 
            \hspace{0.5cm}  \\
            & \textrm{o para un tiempo presente, ($a=1$):} \notag \\
            \ddot{a}(t_0) & = - \frac{4 \pi G \varrho_0}{3} + \frac{8 \pi G \varrho_v }{3}.  \label{eq:2.4.26} 
        \end{align}
    
    Comparando los términos de esta última ecuación con la (\ref{eq:2.4.14}) concluimos que
    
        \begin{equation} \label{eq:2.4.27}
            \boxed{\Lambda = 8\pi G\varrho_v.}
        \end{equation}
    
    Un parámetro a la densidad del vacío y la constante cosmológica asociado también es derivable: 
    
        \begin{align}
             \Omega_{\Lambda} &  = \frac{\varrho_v}{\varrho_c} = \frac{8 \pi G \varrho_v}{3 H_0^2}, \label{eq:2.4.28} \hspace{0.8cm} \textrm{por lo que} \\
             \Lambda &= 3 H_0^2 \Omega_{\Lambda}. \label{eq:2.4.29}
        \end{align}
    
    De esta manera, sustituyendo (\ref{eq:2.4.4}) además de las dos últimas relaciones, (\ref{eq:2.4.14}) y (\ref{eq:2.4.15}) pueden también escribirse como: 
    
        \begin{empheq}[box=\fbox]{align}
            \ddot{a} & = - \frac{\Omega_0 H_0^2 }{ 2a^2} +  H_0^2 \Omega_{\Lambda} a, \label{eq:2.4.30} \\ 
            \notag \\
            \dot{a}^2 & = \frac{\Omega_0 H_0^2  }{a} - \frac{1}{R'^2} + H_0^2 \Omega_{\Lambda} a^2. \label{eq:2.4.31}
        \end{empheq}
    
    O, para la época actual siendo $\dot{a} = H_0$ y $a=1$ (\ref{eq:2.4.31}) es
    
        \begin{align}
            \frac{c^2}{R'^2} & = H_0^2 [(\Omega_0 + \Omega_{\Lambda}) - 1] \hspace{1cm} \text{o} \label{eq:2.4.32} \\
            \notag \\
            \kappa & = \frac{1}{R'^2} = \frac{[(\Omega_0 + \Omega_{\Lambda})]}{c^2/H_0^2}. \label{eq:2.4.33}
        \end{align}
    
    De aquí, la condición de que las secciones espaciales son espacios planos Euclideos se escribe 
   
        \begin{equation} \label{eq:2.4.34}
            (\Omega_0 + \Omega_{\Lambda}) = 1. 
        \end{equation}
        
    Ya que $R_c$ cambia con el factor de escala, $R_c=aR'$, si la curvatura del espacio es cero en este momento, también debió serlo en todos los tiempos del pasado. 
    
 \subsubsection{Consideraciones Generales de la Dinámica de Modelos de Mundo con $\Lambda \neq 0$} \label{sec:2.4.2.2}
 
    La importancia de estos modelos es que ayudan a determinar y mejoras valores de los parámetros cosmológicos. Algunas consideraciones son: 
    
    \begin{itemize}
        \item $\Lambda < 0$: 
        
            Existe una fuerza adicional a la gravedad la cual ralentiza la expansión del Universo. Si revisamos la relación (\ref{eq:2.4.16}), podemos extraer que independientemente de cuán pequeños sean los valores de $\Omega_{\Lambda}$ y  $\Omega_{0}$ , la expansión del Universo se va a revertir.  
            
        \item $\Lambda, \Omega_{\Lambda} > 0$:
        
            Los modelos con constante cosmológica positiva conducen a una ``fuerza repulsiva'' contraria a la fuerza de la gravedad. Tienen una tasa de expansión mínima. Si $\ddot{a} = 0$ en ec. (\ref{eq:2.4.30}), obtenemos el factor de escala mínimo, $a_{min}$ que, insertado en (\ref{eq:2.4.31}), nos permite conocer la tasa mínima de expansión, $\dot{a}_{min}$
            
                \begin{empheq}[box=\fbox]{align}
                    a_{min} & = \left( \frac{\Omega_0}{ 2\Omega_{\Lambda}} \right)^{1/3}, \label{eq:2.4.35} \\
                    \notag \\
                    \dot{a}_{min}^2 & = \frac{3 H_0^2 }{2} (2\Omega_{\Lambda} \Omega_0^2 )^{1/3} - \frac{c^2}{R'^2}. \label{eq:2.4.36}
                \end{empheq}
    
    \end{itemize} 
    
    Analicemos la ec. (\ref{eq:2.4.35}). 
     
     \begin{itemize}
         \item Cuando el lado derecho de la igualdad es mayor que cero, su comportamiento es reflejado en la fig. (\ref{fig:2.5}a). Para valores grandes de $a$, la dinámica sigue  a la  de Universos de Sitter
         
            \begin{equation} \label{eq:2.4.37}
                a(t) \propto \exp \left[\left(\frac{\Lambda}{3} \right)^{1/2} t \right] = \exp (\Omega^{1/2}_{\Lambda} H_ot)
            \end{equation}
        
       \item Si el lado derecho es menor que cero, la función $a(t)$ tiene dos ramas (fig. \ref{fig:2.5}b). Existe un rango de factores de escala para los cuales no hay solución. 
       
            \begin{itemize}
                \item [Rama A:] La dinámica está dominada  $\Lambda$; la fuerza repulsiva es tan fuerte que el Universo nunca se contrajo a una escala a la que la fuerza atractiva de la gravedad pudiera contrarrestar o vencer su efecto. 
        
                \item [Rama B:] En este caso, el Universo nunca se expandió alcanzando valores suficientemente grades de $a$ como para que la fuerza repulsiva transportada a través de $\Lambda$ pudiera evitar el colapso del Universo. No presenta una singularidad inicial, por lo que el Universo ``rebotó''  bajo la influencia de la fuerza repulsiva. 
            \end{itemize}
            
        En el caso límite, en el que el parámetro de densidad $\Omega_0 = 0$ la dinámica del modelo y su solución vienen dadas por
            
            \begin{align} 
                \dot{a}^2 & = H_0^2 [\Omega_{\Lambda} a^2 - (\Omega_{\Lambda}-1)], \hspace{1cm} \textrm{con solución} \label{eq:2.4.38} \\
                a & = \left(\frac{\Omega_{\Lambda} - 1}{\Omega_{\Lambda}} \right)^{1/2} \cosh \Omega_{\Lambda}^{1/2} H_0 \tau, \label{eq:2.4.39}
            \end{align}
        
        siendo $\tau = t - t_{min}$, el cual es medido desde el momento en el que el Universo `rebotó', i.e., cuando $a= a_{min}$. El comportamiento asintótico es debido al colapso exponencial que experimenta y son soluciones exponenciales de Sitter
        
            \begin{equation} \label{eq:2.4.40}
                a = \left(\frac{\Omega_{\Lambda} - 1}{\Omega_{\Lambda}} \right)^{1/2} \exp (\pm \Omega_{\Lambda}^{1/2} H_0 \tau).
            \end{equation}
            
        En estos Universos `rebotantes', el valor más pequeño de $a$, $a_{min}$, es aquel con el desplazamieto al rojo más grande que un objeto pueda tener. 
        
        \item Cuando $\dot{a}_{min} \approx 0$. La figura (\ref{fig:2.5}c) ilustra lo que se conoce como el {\textit{modelo de Eddington - Lema\^itre}} donde $\dot{a_{min}} = 0$. La interpretación de A, B y C es:
        
            \begin{itemize}
                \item[A:] El Universo se expandió desde un punto en el origen en algún tiempo pasado finito y eventualmente alcanzará un estado estacionario en el futuro infinito. 
                \item[B:] El Universo se está expandiendo lejos de una solución estacionaria en el pasado infinito. 
                \item[C:] Es un estado estacionario y además inestable ya que, de ser perturbado, el Universo va a moverse hacia el estado que colapsa A, o hacia B.
                
                Para el caso del Universo estático de Einstein, el estado estacionario sería alcanzado en el presente día. Para estos modelos con $\dot{a} = 0$, el valor de la constante cosmologica puede ser obtenido de la ec. (\ref{eq:2.4.35}):
                
                
                    \begin{align}
                        \Lambda & = \frac{3}{2} \Omega_0 H_0^2 (1 + z_c)^3 \hspace{0.4cm} \text{o equivalentemente} \label{eq:2.4.41} \\
                        \Omega_{\Lambda} & = \frac{\Omega_0}{2} (1 + z_c)^3, \label{eq:2.4.42}
                    \end{align}
           
                siendo $z_c$ el desplazamiento al rojo de objetos estacionarios. 
            \end{itemize} 
            
         Ya que los Universos estáticos de Eddington - Lema\^itre no presentan dinámica, $\dot{a}=0$, sustituyendo (\ref{eq:2.4.41}) (\ref{eq:2.4.36}) se encuentra 
    
            \begin{equation} \label{eq:2.4.43}
                \Omega_0 = \frac{2}{ (1 + z_c)^3 - 3  (1 + z_c) + 2} = \frac{2}{ z_c^2(z_c + 3)},
            \end{equation}
            
        encontrándose una relación uno a uno entre la densidad media de materia en el Universo $\Omega_0$ y el desplazamiento al rojo en el estado estacionario $z_c$.
        
        Los modelos con constante cosmológica positiva pueden tener edades mayores a $H_0^{-1}$. En casos límites como los modelos de Eddington - Lema\^itre con $\dot{a}_{min}=0$ en el pasado infinito, el Universo fuera infinitamente viejo. 
        
       Los conocidos {\textit{modelos de Lema\^itre}} son otro posible caso de Universos con edades mayores a $H_0^{-1}$, donde el valor de $\Omega_{\Lambda}$ es tal que $\dot{a}_{min} >0$. Como se mencionó arriba, un ejemplo de estos modelos es el de la figura (\ref{fig:2.5}d).
       
    \end{itemize}  
    
    
        
        \begin{figure}[H]  
            \centering
                \includegraphics[width=0.8\textwidth]{graficos73_pp226Mlalcon.png}
                \caption{\footnotesize \footnotesize{Modelos con $\Lambda \neq 0$ }. En el gráfico {\bf{a}} los parámetros cosmológicos son $\Omega_0=0.3$ y $\Omega_{\Lambda}= 0.7$, los cuales son valores que se ajustan bastante bien a las estimaciones actuales. {\bf{b}} ilustra el Universo `rebotante'con $\Omega_0 = 0.05$ y $\Omega_{\Lambda}=5$. El cero en el tiempo cósmico se ha establecido para cuando $\dot{a}=0$. {\bf{c}} muestra un modelo de Eddington - Lema\^itre que tiene un corrimiento al rojo estacionario $z_c = 3$, lo que es equivalente a un factor de escala $a=0.25$. El modelo {\bf{d}} tiene parámetros $\Omega_0 = 0.01$ y $\Omega_{\Lambda}= 0.99$, donde la edad del Universo puede exceder por mucho a $H_0^{-1}$. Por último, {\bf{a}} y {\bf{d}} son conocidos como los modelos de Lema\^itre. (Bondi, 1960)}
                \label{fig:2.5}
        \end{figure}
        
       
    Existe una fuerte evidencia de que Universos con geometría plana tiene un valor de $\Omega_{\Lambda} + \Omega_0$ cercano a la unidad. La dinámica de estos modelos planos con diferentes valores de $\Omega_0$ y $\Omega_{\Lambda}$ se muestran en la figura 8. Dichos modelos ilustran cómo la edad del Universo puede ser más grande que $H_0^{-1}$ para valores de $\Omega_{\Lambda}$ suficientemente grandes.
    
    \newpage
        
        \begin{wrapfigure}[15]{r}{0.55\linewidth}       
            \centering
                \includegraphics[width=0.6\textwidth]{flat_models_pp229_Malcolm.png}
                \caption{\footnotesize \footnotesize{Dinámica de modelos con geometría plana, $\Omega_{\Lambda} + \Omega_0 = 1$. El eje $x$ tiene unidades de  $H_0^{-1}$. Este gráfico puede ser comparado con la figura (\ref{fig:2.7}), el cual representa el caso donde $\Omega_{\Lambda}=0$ .}}
                \label{fig:2.6}
        \end{wrapfigure}
    
  \subsection{Cosmología Observacional}
  
      La cosmología actual es dominada por modelos de mundo con un valor finito de la constante cosmológica, sin embargo, modelos con $\Omega_{\Lambda}=0$ son igualmente considerados para hacer comparaciones y así obtener mejores estimaciones sobre parámetros cosmológicos.
      
       La constante de Hubble $H_0$ ha sido introducida, y sabemos que mide la tasa de expansión del Universo en el tiempo actual, i.e., $a=1$. Así mismo, también podemos considerar el {\textit{parámetro de desaceleración}} $q_0$ el cual es adimensional y en la presente época viene definido como
       
         \begin{align} 
            q_0 & = - \left( \frac{a\ddot{a}}{\dot{a}^2}\right)_{t_0}, \hspace{1.0cm} \textrm{o para la época presente:} \label{eq:2.4.44} \\
            q_0 & = - \frac{\ddot{a}}{H_0^2}. \label{eq:2.4.45}
        \end{align}
        
    siendo  $a(t_0)=1$ y $\dot{a}=H_0$. Considerado la ecuación dinámica (\ref{eq:2.4.31} 
       
       \begin{equation} \label{eq:2.4.46} 
	        q_0 = \frac{\Omega_0}{2} - \Omega_{\Lambda}.
        \end{equation} 
        
    Usando los valores preferenciales de $\Omega_{\Lambda} = 0.7$ y $\Omega_0 = 0.3$, nosotros obtenemos un valor del parámetro de desaceleración $q_0 = -0.55$, lo cual indica que efectivamente, el Universo se está desacelerando en la presente época producto de que la energía oscura domina sobre la gravedad. 
    
  \subsubsection{Relación Tiempo Cósmico – Corrimiento al Rojo}
  
    Combinando (\ref{eq:2.4.30}) y (\ref{eq:2.4.31}) se obtiene
    
        \begin{align}
	        \dot{a}^2 &= \frac{\Omega_0 H_0}{a} – H_0^2 [(\Omega_0 + \Omega_{\Lambda} -1] +  \Omega_{\Lambda}H_0^2 a^2, \notag \\
	        \dot{a} & = H_0 \left[ \Omega_0 \left( \frac{1}{a} - 1 \right) + \Omega_{\Lambda}(a^2 - 1) + 1 \right]^{1/2}, \label{eq:2.4.47} 
        \end{align}
    
    y considerando que $a = (1+z)^{-1}$,

        \begin{equation} \label{eq:2.4.48} 
	        \frac{dz}{dt} = - H_0(1+z) [ (1+z)^2 (\Omega_0 z + 1) - \Omega_{\Lambda} z (z+2) ]^{1/2}.
        \end{equation}
        
    El tiempo cósmico, desde el {\textit{Big Bang}} hasta cualquier $z$ puede ser medido por integrar (\ref{eq:2.4.43}), con límites desde $z = \infty$ hasta $z$. 

        \begin{equation} \label{eq:2.4.49}
	        t = \int_0^t{dt} = - \frac{1}{H_0} \int_{\infty}^z{\frac{dz}{  (1+z)[(1+z)^2 (\Omega_0 z + 1) - \Omega_{\Lambda} z (z+2) ]^{1/2}}.} 
	    \end{equation}
    
         \begin{itemize}
         
	        \item Modelos con $\Omega_{\Lambda}=0$. 

		    Para un parámetro de densidad $\Omega_0 >1$ y $\Omega_0 < 1$, la relación corrimiento al rojo -  tiempo cósmico, respectivamente es 
		
		        \begin{equation} \label{eq:2.4.50}
			        t(z) = \frac{\Omega_0}{H_0( \Omega_0 - 1)^{3/2}}[\sin^{-1} x^{1/2} - x^{1/2} (1-x)^{1/2}].
		        \end{equation}
		  
		    y
		  
		        \begin{equation} \label{eq:2.4.51}
	                t(z) = \frac{\Omega_0}{H_0(1 - \Omega_0)^{3/2}}[ y^{1/2} (1+y)^{1/2} + \sinh^{-1} y^{1/2}].
                \end{equation}
		
		    Para $z>>1$ y $\Omega_0 z >>1$ (\ref{eq:2.4.50}) y (\ref{eq:2.4.51}) se reducen a 
		 
		        \begin{equation} \label{eq:2.4.52}
                    t(z) = \frac{2}{3H_0 \Omega_0^{1/2}} z^{-3/2},
                \end{equation}
                
            y entonces hallar la edad del Universo para diferentes valores de $\Omega_0$ es posible. Por ejemplo, para el caso $\Omega_0=1$:
            
                \begin{equation} \label{eq:2.4.53}
                    t_0 = \frac{2}{3H_0}.
                \end{equation}
            
            Los casos más útiles y simples son estos con $\Omega=1$, el caso de un Universo vacío tipo Milne, $\Omega_0 =0$ con una edad de $H_0^{-1}$; y el caso con $\Omega_0=2$, cuya edad calculada es $0.571 H_0^{-1}$. 
            
            \item Modelo con $\Omega_{\Lambda} \neq 0$:
            
            Para que se cumpla la condición de que la curvatura del espacio es cero $(\Omega_{\Lambda}  + \Omega_0) =1$, $R'\rightarrow{\infty}$. De (\ref{eq:2.4.49})
                
                \begin{equation} \label{eq:2.4.54}
                    t = \int_0^t{dt} = - \frac{1}{H_0}\int_{\infty}^z{\frac{dz}{(1+z)[\Omega_0(1+z)^3+\Omega_{\Lambda}]^{1/2}}},
                \end{equation}
            con solución 
        
                \begin{align}
                    t &= \frac{2}{3H_0\Omega_{\Lambda}^{1/2}} \ln \left(\frac{1+\cos \theta }{\sin \theta} \right)  \label{eq:2.4.55}  \hspace{0.5cm} \text{donde:} \notag \\
                    \tan \theta &= \left(\frac{\Omega_0}{\Omega_{\Lambda}} \right)^{1/2} (1+z)^{3/2}
                \end{align} 
                
        Para la época presente, $z=0$. Así: 
        
            \begin{equation} \label{eq:2.4.56}
                t_0 = \frac{2}{3H_0\Omega_{\Lambda}^{1/2}} \ln \left[\frac{1 + \Omega_{\Lambda}^{1/2}}{(1-\Omega_{\Lambda})^{1/2}} \right].
            \end{equation}
            
        Para valores del los parámetros $\Omega_0 = 0.1$ y $\Omega_{\Lambda}=0.9$, la edad del Universo es de $1.28 H_0^{-1}$; para valores ideales de $\Omega_{\Lambda} = 0.7$ y $\Omega_0=0.3$, se tiene un Universo con $0.964 H_0^{-1}$. 
                    
    \end{itemize} 
    
 \subsection{Problema de la Llanura {\textit{Flatness}}}
 
    Existen algunos fenómenos que los modelos de mundo no son capaces de explicar, por lo que ciertos parámetros deben ser introducidos en estos modelos a priori. La incorporación de lo que se conoce como el escenario inflacionario a dado respuestas a varios de los problemas que el modelo estándar no termina de resolver por sí solo. Significa que, durante una época muy temprana del Universo, la expansión procedió de manera exponencial, siendo posiblemente dominado por la constante cosmológica $\Lambda$, como es el caso del Universo De Sitter, sec. (\ref{sec:2.5.1.2}). 
    
    En la sección (\ref{sec:2.4}) ya se había obtenido el parámetro de Hubble y el parámetro de densidad para el tiempo presente, $H_0, \Omega_0$. En el caso de $H_0$ fue mencionado que esta depende del tiempo cósmico. En general 
    
        \begin{equation} \label{eq:2.4.57}
            H(t) = \frac{\dot{a}}{a}.
        \end{equation} 
        
    En (\ref{eq:2.4.47}) se obtuvo la relación que existe entre la constante de Hubble y el desplazamiento al rojo, donde se usó uno de los resultados más importantes en cosmología, la relación (\ref{eq:2.2.11}): $a = (1 + 1)^{-1}$. Así
    
        \begin{equation} \label{eq:2.4.58}
            H(z) = \frac{\dot{a}}{a} = H_0 [(1+z)^2 (\Omega_0 z +1) - \Omega_{\Lambda}z(z+2)]^{1/2}.
        \end{equation} 
        
    Equivalente, al parámetro de densidad en la actualidad $\Omega_0$ (\ref{eq:2.4.4}), un parámetro puede ser obtenido.  
    
        \begin{equation} \label{eq:2.4.59}
            \Omega = \frac{\varrho}{\varrho_c} = \frac{8 \pi G \varrho}{3 H^2}.
        \end{equation}
        
    Considerando `polvo’, i.e., un fluido sin presión ($p=0$),  $\varrho = \varrho_0 a^{-3} = \varrho_0(1+z)^3 $ y sustituyendo en esta última relación
    
        \begin{align}
            \Omega & = \frac{8 \pi G }{3 H^2}  \varrho_0(1+z)^3, \label{eq:2.4.60} \\
            & = \frac{\Omega_0}{\left[ \frac{\Omega_0 z+1}{1+z} \right]  - {\Omega_{\Lambda} \left[ \frac{1}{(1+z)} - \frac{1}{(1+z)^3} \right]}}. \label{eq:2.4.61}
        \end{align}
        
    Considerando grandes desplazamientos al rojo $z >> 1$, $\Omega_0 z >> 1$, y analizando el denominador de la última expresión, para el primer término domina $\Omega_0$, mientras que el segundo tiende a cero, obteniéndose entonces que el parámetro de densidad en cualquier época y para $z$ grandes  $\Omega \rightarrow{1}$ sin importar el valor que tenga en la época actual. Significa que, a alto desplazamiento al rojo, la dinámica de $\Omega_{\Lambda}$ no es importante. Hay marcadas diferencias entre la dependencia que puede existir en las densidades de energía oscura y de materia sobre $z$. Si $\Omega_{\Lambda} = 0$
    
        \begin{equation} \label{eq:2.4.62}
            \left( 1 - \frac{1}{\Omega} \right) = (1+z)^{-1} \left( 1 - \frac{1}{\Omega_0} \right).
        \end{equation}
    
    Si, regresamos a la sección (\ref{sec:2.5.1}) y utilizamos las soluciones de (\ref{eq:2.4.9}), obtenemos
    
        \begin{equation} \label{eq:2.4.63}
            a = \Omega_0^{1/3} \left( \frac{3 H_0 t}{2} \right)^{2/3},
        \end{equation}
    
    lo que reafirma la conclusión expuesta al final de la sección; {\bf{la dinámica de todos los modelos de mundo, con $\Omega_{\Lambda} =0$, tienden a esos del modelo crítico en sus escenarios tempranos}}. Así, si el valor de $\Omega_{0}$ fuera diferente de $1$ en el pasado lejano, también lo fuera en el presente, como se puede apreciar en (\ref{eq:2.4.62}). Para un modelo cosmológico considerado, el valor del parámetro $\Omega_{0}$ no viene siendo más que un ajuste fino (\textit{fine tuning}}) que forma parte de las condiciones iniciales de nuestro Universo. 
    
    Hemos definido el parámetro de densidad como la relación de la densidad de materia – energía del Universo $\varrho_0$ a la densidad crítica $3H_0^2/8 \pi G$. Aquí, es importante destacar que, quien realmente tiene un valor cercano a $1$ es un, digamos, parámetro total $\Omega_{T,0}$, que incluye la constante cosmológica: $\Omega_{T,0} = \Omega_{0} + \Omega_{\Lambda} \approx 1$, pero por el momento consideremos solo $\Omega_{0}$ como el único contribuyente. La evidencia observacional sugiere un valor de $\Omega_0 \approx 0.3$, o $\Omega_{T,0} \approx 1$, respaldado por WMAP ({\textit{Wilkinson Microwave Anisotropy Probe}}) , que a su vez afirma que la curvatura del espacio $\kappa \rightarrow{0}$. Específicamente, el tercer año de WMAP, junto con otros datos del {\textit{Supernova Legacy Project}} o el HST {\textit{key project}} indican que  $\Omega_{T,0} \approx 1.015$ (Spergel et al., 2003, ApJS, 148, 175; 2006, astro-ph/0603449). Note que el $\Omega_0 \approx 0.3$ representó un problema cuando se creía que solo este era la principal contribución a $\Omega_{T,0}$. Esto viene siendo el origen de lo que se conoce como el {\textit{Problema de la Llanura (flatness)}}, i.e., el parámetro debe haber sido $\Omega = 1$ en el pasado distante si su valor en el presente es cercano a $\Omega_0$. 
    
   
    La inflación entonces resuelve el problema de la llanura al suponer que en una época muy temprana del Universo, este estuvo dominado por (alguna) $\Lambda$, y durante este período, la dinámica siguió la de Universos de De Sitter, (\ref{eq:2.4.40}),  donde la expansión se acelera exponencialmente, que es lo que también ocurriría si se perturba el modelo estático de Einstein. De esta manera, al expermintar solo la influencia de $\Lambda$, la ec. (\ref{eq:2.4.62}) muestra que $\Omega \rightarrow{1}$ a medida que $z$ decrece. 
    
    
   {\large{\textbf{Era Inflacionaria}}}


        Esta es una descripción más cualitativa de lo que fue la {\textit{inflación}}. Actualmente, se cree que la evolución del Universo en sus primeras etapas consistió de transiciones hacia estados de menor simetría, partiendo de un estado de máxima simetría. Las cuatro fuerzas fundamentales de la naturaleza estaban unificadas y en el primer rompimiento de simetrías, fueron separadas en dos fuerzas: la fuerza gravitatoria y la fuerza {\textit{hiper débil}}, de igual intensidad en un principio. Poca o nada de distinción existe entre la materia y la antimateria, entre quarks y leptones. 
        
        En el primer segundo, se estima que el Universo tuvo una densidad de $10^6 \mathrm{\,g \, cm^{-3}}$ y la temperatura de $10^{10} \, \mathrm{K}$. En lo que se conoce como la {\textit{época de Planck}}, $10^{-44} \, \mathrm{s}$, la densidad cósmica era del orden de $10^{94} \, \mathrm{g} \, \mathrm{cm^{-3}}$ y la temperatura del orden de los $10^{32} \, \mathrm{K}$ (equivalente a una energía de partículas de $10^{19} \, \mathrm{GeV}$) {\footnote{$1 \,\mathrm{GeV}$ es un billón de electrón voltios, igual a la energía térmica de una partícula a $10000 \,\mathrm{K}$}}. Además, el espacio -  tiempo consistía de una `espuma' densa de fluctuaciones cuánticas reales en escalas de longitud de $10^{-33} \, \mathrm{cm}$ y escalas de tiempo de $10^{-43} \, \mathrm{s}$. Este Universo temprano extremo fue muy corto; debido a la expansión, la temperatura cayó del valor de Planck y a los $10^{-36} \, \mathrm{s}$ alcanzó un valor crítico de $10^{28} \, \mathrm{K}$ (equivalente a $15 \, \mathrm{GeV}$). A continuación, vino el segundo rompimiento de simetrías de la gran unificación: la fuerza hiper débil a su vez, se dividió en la fuerza electro-débil y la fuerza fuerte de la era quark-lepton. Es con esta última transición que inicia la era de la inflación cósmica, además que establece las diferencias entre materia y antimateria, entre quarks y leptones.
        
        Sydney Coleman propuso la idea de una transición de fase de un estado dominado por la fuerza hiper débil a un estado de menor energía compuesto de leptones y quarks dominados por la fuerza electro débil y la fuerza fuerte (Coleman, S., \& De Luccia, F., 1980) (Coleman, S., \& De Luccia, F. (1980). Gravitational effects on and of vacuum decay. Physical Review D, 21(12), 3305–3315. doi:10.1103/physrevd.21.3305 )
        Esta fase puede ser pensada como la transición del agua a hielo cuando la temperatura cae. Ocurre en el punto de congelación $273 \mathrm{\, K}$ ($0^{\circ}C$). Al caer la temperatura, el agua pura no perturbada se enfría a una temperatura menor que la del punto de congelación antes de transformarse en hielo. Similarmente, en el caso del Universo temprano, la temperatura cayó, la transición a quarks y leptones no ocurre en el instante en que la temperatura alcanza un valor crítico de $10^{28} \, \mathrm{K}$. La transformación abrupta en una mezcla de quarks – leptones que se ha súper enfriado fue lo que Coleman denominó {\textit{vacío falso}}; el más bajo estado posible de energía disponible para la fuerza hiper débil. Cuando la transición finalmente  ocurre espontánamente, el vacío falso libera su inmensa energía latente, devolviendo la temperatura casi a su valor inicial de la gran unificación y formando quarks, leptones y gluones (bosones mediadores de la fuerza fuerte). 
        
        En 1980, Alan Guth, quien elaboró la primera teoría del Universo inflacionario en 1970, mientras estudiaba el conocido {\textit{problema del monopolo}} consideró también el vacío falso. Pensaba que este vacío falso existió en un estado extraordinario de presión negativa que causó la expansión del Universo acelerada. Guth argumentaba que la expansión acelerada, lo cual denominó inflación, resolvía el problema del monopolo, de la llanura y del horizonte. Así, cualquier Universo que experimentó en su infancia un período de expansión acelerada es ahora conocido como un {\textit{universo inflacionario}}.
        
        

\section{Modelo $\Lambda$CDM}

    La biografía acerca de nuestro Universo ha sido creada, la cual está apoyada en observaciones de décadas atrás hasta nuestros días y describe la historia y estructura del Universo. Nos referimos al ``modelo cosmológico estándar'' que actualmente se conoce como {\textit{Modelo de Materia Oscura Fría} con constante cosmológica $\Lambda$, $\Lambda$CDM ({\textit{Cold Dark Matter}}). Este modelo está basado en la teoría de la Relatividad General donde la métrica es la de Robertson - Walker (\ref{eq:2.1.22}) estudiada en la la sec. (\ref{sec:2.1.3}). $\Lambda$CDM está compuesto de materia ordinaria además de materia no relativista, i.e., materia oscura fría detectada únicamente por sus efectos gravitatorios. A su vez, la constante cosmológica representa la expansión acelerada cuya ecuación de estado es 
    
        \begin{equation} \label{eq:2.4.1} 
            \rho = \rho_i \left( \frac{a}{a_i} \right)^{-3(w + 1)},
        \end{equation}
    
    siendo $w = -1$, $\rho_{\Lambda} = \Lambda$. Ya que, en general, $p = w\rho $, $p_{\Lambda} = \rho_{\Lambda} = \Lambda$.  Para un Universo dominado por radiación $w = 1/3$, mientras que para uno dominado por polvo $w = 0$, teniéndose entonces $\rho \propto a^{-4} $ y $\rho \propto a^{-3} $, respectivamente. Ambos casos son basados en la expansión desacelerada del Universo. 
    
    El modelo está basado en la interpretación hecha por el corrimiento al rojo de galaxias distantes como un resultado de la expansión del espacio. Extrapolando hacia atrás en el tiempo da predicción de que el Universo se expandió desde un {\textit{ extremadamente denso y caliente estado}}, el Big Bang, hace unos $13.80 \, \mathrm{Gyr}$. $\Lambda$CDM explica la Ley de Hubble, Fondo Cósmico de Microondas (CMB), la abundancia de elementos ligeros y la estructura a gran escala. 
    
    El modelo estádar describe un conjunto de escenarios evolutivos experimentados por el Universo, los cuales son brevemente explicados a continuación inspeccionando la línea de tiempo de la ilustración, figura. (\ref{fig:2.7}).  
    
    El Universo temprano extremo comenzó con una violenta expansión desde un estado casi inconseviblemente caliente y denso. Una mezcla casi homogénea de fotones y materia, estrechamente acoplados en forma de plasma. 
            
        \begin{enumerate}
            \item Entre los primeros $10^{-36} - 10^{-34} \, \mathrm{s}$ después del Big Bang, el Universo experimentó la llamada {\textbf{era inflacionaria}}. Las fluctuaciones de densidad en el plasma primordial se siembran por fluctuaciones cuánticas en el campo que impulsan la inflación. Estas fluctuaciones de densidad potencialmente gravitatorias, más tarde sembraron la formación de estructuras. Además, la amplitud es casi la misma en todas las escalas espaciales (e.g., Tsujikawa S., 2003; Baumann D., 2009).
            
            \item Pequeñas perturbaciones se propagan a través del plasma, como una onda de sonido, produciendo regiones de baja y sobre densidad con cambios simultáneamente en la densidad de materia y radiación. 
            
            \item El Universo temprano era una sopa de materia y energía, donde los pares partículas/antipartículas se producían y se aniquilaban constatemente. A medida que el Universo se enfriaba pudo producir ciertos tipos de partículas. Por ejemplo, la creación de un par de protones/antiprotones se detuvo por debajo de unos pocos trillones de Kelvin, mientras que la creación de un par electrón/positrón continuó hasta unos pocos miles de millones de Kelvin. Cuando este proceso ocurre para un tipo de partículas se dice que la partícula se ha {\textit{congelado (frozen out)}}. La razón por la que en el Universo hay materia es por un fenómeno poco conocido denominado {\textit{bariogénesis}} causado por una asimetría en la física de la materia/antimateria. 
            
            \item Debido a las condiciones físicas que reinan en el plasma, las partículas creadas durante la bariogénesis, electrónes y bariones (neutrones y protones), se recombinan de manera estable para crear los primeros átomos, principalmente en forma de Hidrógeno neutro, así como también otros elementos ligeros como el Deuterio (un isótopo pesado del Hidrógeno), Helio-3, Helio-7 y Litio-7. A esta formación de elementos ligeros se le conoce como la {\textit{Nucleosíntesis}} del Big Bang (BBN). También, los fotones se desacoplan de los bariones (la radiación escapa) cuando el plasma se vuelve neutro y las perturbaciones ya no se propagan en forma de ondas acústicas: el patrón de densidad existente se congela. Esta ``instantánea'' de las fluctuaciones de densidad se conserva en las anisotropías de CMB y la huella de las {\textit{oscilaciones acústicas bariónicas}} (BAO) observada hoy en la estructura a gran escala del Universo (Eisenstein \& Hu 1998).
            
            \item Como consecuencia de la recombinación se produce un Universo casi neutro que no puede ser observado en la mayor parte del espectro electromagnético. Este lapso de tiempo se conoce también como {\textit{era oscura}}. Inicia el colapso gravitatorio de la CDM en las regiones con sobredensidades. Prosigue entonces el colapso de la materia bariónica (ordinaria) en estos halos de CDM y una especie de {\textit{amanecer cósmico}} toma lugar con la formación de las primeras fuentes de radiación, como por ejemplo, estrellas. Esta radiación permite la reonización del medio intergaláctico (IGM)
            
            \item Continúa la formación de estrcuturas en el Universo bajo la influencia de la gravedad, formando entonces una gran red cósmica de densidad de materia oscura. La formación de galaxias se convierte en trazadores de la materia que compone el Universo. Cúmulos de galaxias son las estructuras más grandes atadas gravitatoriamente.
            
            \item Con la expsnsión del Universo la presión negativa asociada a la constante cosmológica, que viene representada por la energía oscura en el modelo $\Lambda$CDM se hace cada vez más intensa contrarrestando y dominando la atractiva gravedad, y así la expansión va acelerándose. 
        \end{enumerate}
        
        \begin{figure}[H]
            \centering
            \includegraphics[width=0.9\linewidth]{time_line_UNiverse.png}
            \caption{\footnotesize Línea de tiempo de los escenarios evolutivos del Universo en los aproximadamente $13.8$ Gyr. Fuente: \url{https://lambda.gsfc.nasa.gov/education/graphic_history/univ_evol.cfm}}
            \label{fig:2.7}
        \end{figure}
        
{ \large{\textbf{Actual Composición del Universo}}}
 
    Observaciones cosmológicas estiman que el Universo actualmente está comprendido por (ver fig. \ref{fig:2.8})
    
        \begin{itemize}
            \item $\sim 4.9 \, \%$ Materia Bariónica, i.e., la materia ordinaria o luminosa compuesta por partículas del modelo estándar de partículas; protones, neutrones y electrones. Es toda la materia que nos rodea, desde estrellas, galaxias hasta personas.  
            \item $\sim 26.8 \, \%$ Materia Oscura Fría:  También llamada ``materia faltante'' en el Universo. Comprende los halos de materia oscura que rodean a las galaxias y a los cúmulos de galaxias y que propiciaron la formación de estructuras. Las partículas que componen la materia oscura actualmente sigue siendo un misterio. Las candidatas deben cumplir con ser estables a escalas cosmológicas, interccionar débilmente con la radiación electromagnética y ser ``frías'' (no relativistas). Una de las partículas potencialmente candidata proviene de lo que se conoce como {\textit{SÚpersimetría}} (SUSY), una extensión del modelo estándar de partículas actual. Es una nueva partícula elemental referida como una {\textbf{partícula masiva de interacción débil}} (WIMP). 
            \item $\sim 68.3 \, \%$ Energía Oscura: La expansión del Universo acelerada fue probada por dos grupos independientes, gracias a observaciones de supernovas tipo 1a (Perlmutter S. et al., 1999; Riess A. G., et al., 1998). Este fenomeno requiere entonces de ``algo'' opuesto a la gravedad, una ``antigravedad'' que puede ser generada, e.g., por algún nuevo campo de energías que permea el Universo, posiblemente la constante cosmológica. 
        \end{itemize}
 
        
        \begin{figure}[H]
            \centering
            \includegraphics[width=0.9\linewidth]{dark_matter_baryons.png}
            \caption{\footnotesize Estimaciones del WMAP y PLACK para las densidades de materia oscura, materia ordinaria y energía oscura. (Fuente: Masterclass.
            CEVALE2, \url{http://www.cevale2ve.org/es/inicio/}}
            \label{fig:2.8}
        \end{figure}

    
  
    $\Lambda$CDM es ampliamente exitoso debido a la concordancia y coherencia de resultados ``experimentales'' que se basan en una variedad de efectos físicos y observaciones en diversas épocas cósmicas.  Por ejemplo, el método para determinar la expansión del Universo, caracterizado por el parámetro de Hubble $H$ es midiendo distancias de objetos distantes. En 1980, estudios realizados por Perlmutter (2003) hallaron que el brillo máximo en la curva de luz de supernovas 1a proporcionaban una candela confiabe para explorar el comportamiento de la expansión a bajo $z$. Posteriormente, de observaciones del brillo aparente de supernovas a $z<1$, se obtuvieron evidencias directas de que el Universo se expande de manera acelerada (Perlmutter et al., 1999; Riess et a., 1998). En el marco de la cosmología estándar de Friedmann, solo un agente puede generar la expansión acelerada, esta es la constante cosmológica o su posible generalización, la energía oscura. También, el modelo de CDM da respuesta a la cinemática observada en galaxias y sus curvas de rotación,  que no es esperado si se considerara solo la materia bariónica. También, para que el Universo sea compatible con el escenario predicho de la inflación, un Universo plano, es necesario que haya una contribución adicional a la densidad de materia $\Omega_m = \Omega_{dm} + \Omega_b \approx 0.3$ (con el índice $b$ referido a la materia bariónica y el $dm$ a la materia oscura), esto es $\Omega_m + \Omega_{\Lambda} \approx 1$. En 1980, dos sondeos, el sondeo fotográfico APM ({\textit{automated photographic Measuring}}, Efstathiou et al., 1990) y el sondeo al rojo QDOT de galaxias infrarrojas (Saunders et al., 1991) mostraron que el espector de potencia de de la distribución de galaxias, si rastrea el de la materia a gran escala, puede ajustarse bien con un modelo de CDM, solo si la densidad de materia es baja, $\Omega_m \approx 0.3$. Otra razón para doptar el modelo $\Omega_{\Lambda}$CDM. ESte modelo también predice la generación de elementos ligeros (Deuterio, Helio, Berilio y Litio) en el Universo temprano como consecuencia de la era de recombinación. Sin embargo, algunos tópicos quedan sin resolver. La posibilidad de la presencia de un aparente campo de energía oscura que aporta $\sim 70 \, \%$ del contenido del Universo y que durante los últimos 5 mil millones de años  ha impulsado la expansión cósmica acelerada. El escenario más simple atibuiría una energía del vacío del orden de los $10^{19}$GeV donde la gravedad debería unificarse con las otras tres fuerzas fundamentales. Esto es $120$ órdenes de magnitud mayor que el valor esperado por la cosmología. Por otro lado, si se plantea una conexión a la escala de la cromodinámica cuántica aún dejaría una discrepancia de $40$ órdenes de magnitud. Un campo de energía oscura que es tan poco natural cuando se comprara con estas escala de la física de partículas es realmente un profundo misterio aún sin una respuesta clara. 
    
 {\large{\textbf{Parámetros Cosmológicos}}}
 
    Son seis los parmámetros independientes suficientes para describir la cosmología con el modelo estándar $\Lambda$CDM. Dichos parámetros claves son elegidos con la finalidad de evitar degeneraciones y así lograr la convergencia del modelo y ajustarlos a los datos (Kosowsky et al., 2002). Así, tenemos: el índice de ley de potencia escalar $n_S$, la edad del Universo $t_0$, la profundidad ópticaq para reonización $\tau$, la constante de Hubble $H_0$ en el tiempo presente además de cuatro parámetros que guardan relacionon con la densidad de materia del Universo. Se tiene: la densidad de bariones $\Omega_b \, h^2$, la densidad de CDM $\Omega_{dm} \,h^2$, con $h$ la constante de HUbble reducida cuyo valor definida como $H_0/100$ la densidad de materia $\Omega_m$ y la amplitud de fluctuación $\sigma_8$. Teniendo en cuenta que los parámetros de densidad se definen como la relación de densidad media de materia en el presente $\varrho_0$ de cada componente a la densidad crítica $\chi$ , siendo $\sum_{\chi} \chi \approx 1 $. De esta manera, se tiene $\Omega_m + \Omega_{\Lambda} \approx 1$, donde $\Omega_m \approx \Omega_b + \Omega_{dm}$.
    
    En $\Lambda$CDM, el {\textbf{índice de ley de potencia escalar $n_S$}} representa un parámetro crítico para caracterizar la intensidad de las anisotropías presentes en el CMB en todas las escalas angulares. La inflación calcula un valor ligeramente menor que 1 para este índice que se debe a una lenta variación del potencial inflacionario y al tamaño del horizonte sobre el período que abarcó la inflación (Kinney, 2003; Tsujikawa, 2003). Las misiones PLANCK y WMAP estiman un valor de $n_S = 0.96-0.97 \pm 0.0080$. Son varios los métodos para determinar la {\textbf{edad del Universo $t_0$}}, pero ningunos de estos son una medida directa. Varios de los resultados se basan en la teoría de evolucióne stelar enfocándose en las estrellas más viejas, e.g., cúmulos globulares pobres en metalicidad (Spergel et al., 2003). Bajo este método, el valor de $t_0$ es obtenido por medir la la edad del cúmulo globular NGC7652 con datos del HST, $t_0 = 14.5 \pm 1.5$Gyr. Usando modelos cosmológicos representados por resultados CMB de los nueve años de WMAP y PLANCK 2015 arrojan valores de $t_0 = 13.772 \pm 0.059$ y $t_0 = 13.799 \pm 0.021$Gyr, respectivamente. La {\textbf{profundidad óptica para reonización $\tau$}} es una cantidad adimensional que proporciona una medida de la opacidad de los electrones libres en la línea de visión a la radiación CMB. Un límite superior detectado de las anisotropías de la temperatura en el fondo cósmico de microondas indican un valor $\tau <1$ ( (Griffiths et al., 1999). De la misma manera, un límite inferior es estimado de $\tau > 0.03$ (Pryke et al. 2002, Fan et al. 2005). El WMAP 2013 estima un valor de $0.081 \pm 0.012$. Para el caso de la {\textbf{densidad de materia $\Omega_m$}}, son muchas las técnicas empleadas para estimar su valor, $\sim 30 \, \%$, el cual representa la fracción de densidad de energía en todas las formas posibles de materia, que incluye la bariónica y la de materia oscura. Por ejemplo, da-\^Angela et al. ( 2008), a través de una muestra de {\textit{objetos casi estelares}} 
    (QSO) obtuvo un valor de $02.25 \pm 0.08$, donde usó el parámetro de sesgo ({\textit{bias}} $b$, y la distorsión del espacio de corrimiento al rojo $\beta$. Un resultado que coincidió con  Verde et al. (2002), quien trabajó fue con galaxias. Seguidamente, el {\textbf{parámetro de densidad bariónica $\Omega_b \, h^2$}} tiene dos métodos para ser determinada, una de ellas es con medidas de la abundancia de Deuterio primordial (D/H) junto con la BBN analizando nubes de Ly$\alpha$ en QSO o sistemas Ly$\alpha$  amortiguados, obteniéndose $\Omega_b \, h^2 = 2.5 \pm 0.1, 2.156 \pm 0.020$ y $2.260 \pm 0.034$  (e.g., Pettini \& Bowen, 2001;  Cooke et al. 2016). Recordemos que el modelo ya asume el contenido de CDM de manera natural, así, estimaciones para el parámetro de {\textbf{materia oscura no bariónica}} son limitadas por mediciones de $\Lambda$CDM para el CMB. Por ejemplo, $\Omega_{dm} \, h^2 = 0.1188 \pm 0.0010$ con datos del PLANCK 2015, y $0.1153 \pm 0.0019$ de WMAP 2013. La interpetación de $\Omega_m$ es especificar la densidad de materia media en el presente mientras que $\sigma_8$ determina cómo está distribuida la materia (agrupada) en un escala de longitud fija. Especçificamente, $\sigma_8$ es definida como el $rms$ de las perturbaciones de densidad $z=0$  en escalas de $8 \, h^{-1}$Mpc (e.g., Planck XX, 2014). Una manera de determinar este parámetro es basándose en campos de velocidades de las galaxias los cuales son fuertemente degenerados con $\Omega_m$. El trabajo de  Spergel et al. (2003) combinó resultados  de la medición de velocidades de las galaxias IRAS (Willick \& Strauss, 1998) con la amplitud de fluctuación IRAS (f Fisher et al., 1994)para obtener un valor de $\Omega_ m 
    + \sigma_8$, con $\sigma_8 = 0.73 \pm 0.1$. Otros valores han sido obtenidos con el espectro de potencia del CMB por WMAP 2013 $\sigma_8 = 0.820 \pm 0.014$ y Planck 2015 $0.8159 \pm 0.0086$. Anterimente, la {\textbf{constante de Hubble}} fue estudiada. Recordemos que en la
   cosmología Friedmann - Lema\^itre - Walker - Robertson, la dependencia entre $H$ y el corrimiento al rojo viene expresada por la ec. (\ref{eq:2.4.58}). La estimación de $H_0$ es amplia y tiene una larga historia. Podemos referenciar a  Freedman \& Madore (2010). En la actualidad, los valores están cercanos a $H_0 = 70$km/s/Mpc. Recientemente, se ha obtenido un valor proveniente del estudio de las ondas gravitatorias, que es diferente de la escalera de distancias cósmicas. El primer valor fue de un grupo de investigadores de LIGO/Virgo, del cual se obtuvo $H_0 = 70.0^{+12.0}_{-8.0}$. 
   
   Una tabla con valores actuales de todos los parámetros necesarios en $\Lambda$CDM ha sido extraía de Planck Collaborations 2018, mostrada a continuación en la figura (\ref{fig:2.9}). 
   
        \begin{figure}[H]
            \centering
            \includegraphics[width=1.0\linewidth]{table_cosmological_parameters.png}
            \caption{\footnotesize Parámetros cosmológicos para $\Lambda$CDM. Resultados usando probabilidad {\textit{Plik}} son desplegados en las dos primeras columnas. La probabilidad {\textit{CamSpec}} da una idea de las incertidumbres restantes en la polarización l-alta. La columna ``combinado'' es un promedio de los resultados de $\mathsf{Plik}$ y  $\mathsf{CamSpec}$ asumiendo igual peso.} 
            \label{fig:2.9}
        \end{figure}
    
    
 \section{El Universo a Gran Escala}
 
    Las galaxias no están distribuidas uniformemente en el espacio, sin embargo, a grandes escalas, el Universo muestra una estructura coherente, con galaxias que residen en grupos y cúmulos en escalas comprendidas entre $1 - 3 \, h^{-1}$Mpc, las cuales descansan en las intersecciones de extensos filamentos de galaxias con longitudes mayores a $10 \, h^{-1}$Mpc. También, existen vastas regiones relativamente vacías, conocidas como {\textit{vacíos}} las cuales contienen pocas galaxias y abarcan el volumen entre estas estructuras. Esta estructura filamentosa que une galaxias, grupos y cúmulos de galaxias, a gran escala se ve como una {\textit{red cósmica}}.
    
    En un Universo $\Lambda$CDM, grupos de materia oscura en cuasi-equilibrio, o ``halos'', crecen por el colapso y agregación de estructuras cada vez más masivas. Es un modelo fenomenológico explicado muy bien por Press y Schechter (1974). La formación de galaxias inicia en el centro de estos halos oscuros dentro del cual toda la materia está gravitatoriamente unida y ocurre el enfriamiento y condensación del gas que se fragmenta en estrellas una vez que se vuelve lo suficientemente denso para que el colapso proceda (White, S. D. \& Reess, M.J., 1978). Así mismo, cúmulos y grupos de galaxias se forman a medida que los halos de DM se combinan formando sistemas cada vez más masivos. Las galaxias visibles se forman porque la materia bariónica ordinaria puede irradiar su energía cinética y caer hacia los centros de los halos de DM. La red cósmica, el patrón que se observa a gran escala, es una ``agudización'' gravitatoria no lineal del patrón ya presente en el campo aleatorio gaussiano de las fluctuaciones iniciales (Bond, J. R. et al., 1996). Los primeros objetos observados, posiblemente, fueron estrellas masivas que colapsaron en en pequeños halos a $z > 50$ o mayores (Reed et al., 2005). A un $z \approx 15$ estas estrellas pudieron haber sido lo suficientemente numerosas para que su radiación reionizara todo el gas presente en el Universo (Ciardi et al., 2003).  También, a $z<6$ las predicciones para la estructura a mediana escala, estado de ionización y el contenido de elementos pesados pueden ser detectados a través del espectro de cuásares (QSO) distantes. Para corrimientos al rojo todavía menores, se pueden obtener medidas de la estructura a gran escala (LSS) a partir de las distorsiones débiles y coherentes de las imágenes de galaxias tenues inducidas por lentes gravitacionales a medida que viajan por la red cósmica intermedia. Por ejemplo, la figura (\ref{fig:2.10}) muestra el espectro en alta resolución de un QSO a $z= 3.26$. A $\lambda$ menores que la de línea de emisión Lyman$\alpha$ existe un ``bosque'' de líneas de absorción de distinta intensidad. Este bosque surge de la absorción  Lyman$\alpha$  por la distribución uniformemente variable del hidrógeno en el medio intergaláctico en primer plano, en efecto de los filamentos y halos en la estructura cósmica. El aumento en la calidad y cantidad de datos disponibles ha hecho posible medir una variedad de estadísticas para el bosque Lyman$\alpha$ como una función del desplazamiento al rojo $z$ a alta precisión.
    
    
    La evolución temporal de las distribuciones de masa y galaxias en una subregión se muestra en la figura (\ref{fig:2.11}). Se observa la red cósmica donde se detallan los filamentos y paredes que rodean una espuma de de vacíos. Una característica sorprendente es el hecho de que, si bien el crecimiento de la LSS es notorio en la distribución de masas, la distribución de galaxias aparece fuertement agrupada en todo momento. 
    
        \begin{figure}[H]
            \centering
            \includegraphics[width=0.7\linewidth]{lyman_forest.png}
            \caption{\footnotesize El ``bosque'' Lyman-$\alpha$ como una prueba de la LSS. El panel de arriba muestra el espectro de un QSO a $z=3.26$. En el marco en reposo, la línea de emisión Lyman-$\alpha$ está a $\lambda = 1216 (1+z)$\AA (en la figura se muestra la $\lambda$ observada), y hacia la izquierda se observa el ``bosque'' de líneas de absorción de diferentes intensidades producidas por el gas de hidrógeno neutro intergaláctico a lo largo de la línea de visión del QSO hasta la Tierra. El panel inferior muestra el espectro simulado, correspondiente a la LSS en $z \approx 3$. El boceto en el panel central muestra un ejemplo de la distribución de gas en una simulación  en un modelo $\Lambda$CDM. (Fuente: Springel V. et al., 2006)} 
            \label{fig:2.10}
        \end{figure}
    
         \begin{figure}[H]
            \centering
            \includegraphics[width=0.8\linewidth]{galaxies_DM_simulations_Springel_pp5.pdf}
            \caption{\footnotesize \footnotesize Evolución temporal de la LSS en DM y galaxias. obtenidas de simulaciones del modelo $\Lambda$CDM. A la izquierda se muestra la distribución de DM para diferentes $z$ de la simulación de N-cuerpos {\textit{Millennium}} de formación de estructuras (Springel, V. et al.,2005). El lado derecho muestra la distribución predicha por de galaxias en la misma región que el panel izquierdo y para una misma época obtenida por aplicar técnicas semi-analíticas para simular la formación de galaxias. La DM evoluciona de una distribución suave y casi uniforme a un estado altamente agrupado, a diferencia de las galaxias, que están fuertemente agrupadas desde el principio. (Fuente: Springel V. 2006)}
            \label{fig:2.11}
        \end{figure}
        
        
        \begin{figure}[H]
            \centering
            \includegraphics[width=0.8\linewidth]{LSS_millenniumII.png}
            \caption{\footnotesize \footnotesize LSS simulada en   {\textit{Millennium II}}. Una simulación de la evolución de DM usando los parámetros dados por el modelo $\Lambda$CDM. (Fuente: Boylan-Kolchin, 2009)}
            \label{fig:2.12}
        \end{figure}
    
 \subsection{Teoría de Perturbación y Formación de Estructuras}


    Existió una época y una escala, a la cual los cálculos lineales realizados a partir del espectro de las fluctuaciones del $\Lambda$CDM hace predicciones precisas, e.g., la época del CMB, hace aproximadamente unos 400 000 años después del Big Bang, o en escalas muy grandes en épocas posteriores. Sin embargo, a escalas menores, donde se dio la formación de estructuras, las fluctuaciones crecieron lo suficiente como para que sean fuertemente no lineales, y entonces las simulaciones juegan un rol importante. Debido a la gravedad, las regiones con una densidad un poco más alta que el promedio se expanden más lentamente que el promedio, así como las que tienen una densidad ligeramente más baja, se expanden un poco más rápido. 
    
    El proceso que genera la formación no lineal en el Universo se conoce como {\textit{colapso gravitatorio}} y es un tanto engañoso el nombre debido a que lo que realmente ocurre es que cuando las fluctuaciones positivas han crecido lo suficiente para que sean aproximadamente el doble de densas que las regiones típicas de su tamaño, dejan de expandirse mientras el Universo circundante sigue expandiéndose a su alrededor.
    Como una consecuencia, las regiones que colapsan primero son más densas que aquellas que colapsan más tarde, i.e., los halos de materia oscura de la galaxia son más densos que los halos de las agrupaciones. 

    A continuación, para considerar el régimen lineal con estructuras cósmicas de $ \sim <
    50 \, h^{-1}$Mpc, un enfoque Newtoniano podrá ser considerado. En este caso, el camino libre medio de partículas es mucho más pequeño que las escalas consideradas. Son incluido el gas bariónico, el polvo sin presión (DM sin colisiones). También, las sobredensidades del IGM estudiadas a través del Bosque de Lyman-$\alpha$ son incluidas. El caso no lineal será analizado a través de la teoría de perturbaciones de orden superior, la cual incluye estructuras de escalas superiores, como es el caso del agrupamiento de galaxias. 

    Vamos a considerar un conjunto de tres ecuaciones: la {\textit{ecuación de continuidad}} que guarda relación con la conservación de la masa; la {\textit{ecuación de Euler}}, referida a la ecuación de movimiento, y la {\textit{ecuación de Poisson}} que describe el campo gravitatorio:
    
        \begin{empheq}[box=\fbox]{align}
            \frac{\partial \rho}{\partial t} + \vec{\nabla}_r.(\rho \vec{v}) & = 0, \label{eq:2.6.1} \\
            \frac{\partial \vec{v}}{\partial t} + (\vec{v}.\vec{\nabla}_r) \, \vec{v} & = - \frac{\vec{\nabla}_r p}{\rho} + \vec{\nabla_r} \phi, \label{eq:2.6.2} \\
            \nabla^2 \phi & = 4 \pi G \rho, \label{eq:2.6.3}
        \end{empheq}

    las cuales describen la evolución de un fluido, siendo $\rho$ la densidad, $\vec{v}$ el campo de velocidades de un fluido en el potencial gravitatorio $\phi$ y $r$ la coordenada propia. 
    
    Descomponiendo la densidad $\rho$ y el campo de velocidades $\vec{v}$ como una perturbación, se tiene:
    
        \begin{align}
            \rho = \rho_0 + \delta \rho,  \label{eq:2.6.4} \\
            \vec{v} = \vec{v_0} + \delta \vec{v}. \label{eq:2.6.5}
        \end{align}

    Convirtiendo la coordenada propia a la coordenada comóvil $\vec{x}$, de modo que podamos introducir la evolución temporal de la perturbación en un Universo en expansión con métrica del tipo Robertson - Walker, ec. (\ref{eq:2.1.22}), se sigue
    
        \begin{equation} \label{eq:2.6.6}
            \vec{r} = \vec{x} a(t), 
        \end{equation}
        
    y la velocidad propia
    
        \begin{equation} \label{eq:2.6.7}
            \vec{v} = \dot{\vec{r}} =  \dot{\vec{x}} a(t) +  \vec{x} \dot{a}(t) = H \vec{r} + \vec{x} \dot{a}(t) = \vec{v_0} + \delta \vec{v}, 
        \end{equation}
        
    donde identificamos la {\textit{velocidad de Hubble}} $\vec{v_0}= H \vec{r}$ y $\delta \vec{v} = \vec{x} \dot{a}(t)$ como la {\textit{velocidad peculiar}} que describe el movimiento de un fluido relativo al observador comóvil con el fondo (observador fundamental) en $\vec{x}$, donde definimos $u \equiv \delta \vec{v}/a$. El gradiente y derivada temporal son reemplazadas por: 
    
        \begin{align}
            \vec{\nabla}_x = \frac{1}{a}\vec{\nabla}_r, \label{eq:2.6.8} \\
            \frac{\partial}{\partial t} + H \vec{x} \, . \, \vec{\nabla}_x \rightarrow{\frac{\partial}{\partial t}}. \label{eq:2.6.9}
        \end{align}

    Insertando un {\textit{ansatz}} perturbativo en la ecuación de continuidad, linealizando e introduciendo un contraste de densidad $\delta \equiv \delta \rho/ \rho_0$, la {\textit{ecuación de continuidad perturbada}} se escribe como:
    
        \begin{equation} \label{eq:2.6.10} 
            \dot{\delta} + \vec{v}_0 \, . \, \vec{\nabla} \delta + \vec{\nabla} \, . \, \delta \vec{v} = 0.
        \end{equation}
    
   De la misma manera para las ecuación de Euler se tiene
   
        \begin{align} \label{eq:2.6.11} 
           \frac{\partial \delta \vec{v}}{\partial t} + H \delta \vec{v} +  (\vec{v}_0 \,.\vec{\nabla}_x) \, \delta \vec{v} & = - \frac{\vec{\nabla}_x \delta p}{\rho_0} + \vec{\nabla_x} \delta \phi, 
        \end{align}
    
    y para el ecuación de Poisson
    
        \begin{equation} \label{eq:2.6.12} 
            \nabla^2 \delta \phi = 4\pi G \rho_0 \delta. 
        \end{equation}
    
    Ya que estamos en el caso de perturbaciones lineales, donde $\delta << 1$, términos de más altos ordenes tanto en $\delta$ como en $\vec{v}$ son lo suficientemente pequeños para ser despreciados en (\ref{eq:2.6.10}) y (\ref{eq:2.6.11}). De esta manera, las tres ecuaciones perturbadas se reescriben como: 
     
        \begin{empheq}[box=\fbox]{align}
            \dot{\delta} + \vec{\nabla} \, . \, \vec{u} & = 0, \label{eq:2.6.13}  \\
            \dot{\vec{u}} + H \vec{u} & = - \frac{\vec{\nabla}_x \delta p}{a^2 \rho_0} + \frac{\vec{\nabla}_x \delta \phi }{a^2}, \label{eq:2.6.14} \\
            \nabla^2 \delta \phi & = 4\pi G \rho_0 a^2 \delta, \label{eq:2.6.15}
        \end{empheq}
    
    para $\delta$, $\vec{u}$ y $\delta \phi$. Ahora, es necesario una ecuación de estado que enlace la presión con las fluctuaciones de densidad:
    
        \begin{equation} \label{eq:2.6.16}
            \delta p = \delta p (\delta) = c_s^2 \delta \rho = c_s^2 \rho_0 \delta,
        \end{equation}
    
    siendo $c_s$ la velocidad del sonido, la cual relaciona las perturbaciones de densidad con las perturbaciones de presión. 
    
    Calculando la derivada con respecto al tiempo en la ec. de continuidad (\ref{eq:2.6.13}), combinándola con la ec. de Euler (\ref{eq:2.6.14}), haciendo uso de la ec. de Poisson, (\ref{eq:2.6.15}) y considerando que $\delta$ es descompuesta en ondas planas, $\delta \propto e^{-i \, \vec{k}.\vec{x}}$ con $\vec{k}$ en unidades comóviles, donde toda la dependencia temporal es llevada por la amplitud de la onda. Se obtiene:
    
        \begin{equation} \label{eq:2.6.17}
           \boxed{ \ddot{\delta} + 2H \dot{\delta} = \delta \left(4\pi G\rho_0 - \frac{c_s^2 k^2}{a^2} \right).}
        \end{equation}
    
    En esta última ecuación, $2\dot{\delta} H = 2 \dot{a}/a$ se refiere a  término de {\textit{arrastre de Hubble (Hubble drag)}}, el cual tiende a amortiguar las perturbaciones a medida que el Universo crece; el primer término a la derecha de la igualdad es el gravitatorio, el cual permite el crecimiento de las perturbaciones, y por último el segundo término representa la variaciones espaciales en densidad. Si definimos: 
    
       \begin{equation} \label{eq:2.6.18}
           \omega_0^2 \equiv \frac{c_s^2 k^2}{a^2} - 4 \pi G \rho_0,
       \end{equation}
        
       
    
    con $k = 2\pi/ \lambda$ el número de onda de la oscilación. En un {\textbf{Universo sin expansión ($H=0$)}} (\ref{eq:2.6.17}) tiene solución 
    
        \begin{equation} \label{eq:2.6.19}
            \delta \propto i e^{(i \omega t - i \vec{k}.\vec{x})} \rho_0,
        \end{equation}
        
     Existe un valor del número de onda $k$ para el cual $\omega_0$ es imaginario, el {\textit{número de onda de Jeans}} $k_J$ definido como
     
        \begin{equation} \label{eq:2.6.20}
            k_J \equiv \frac{\sqrt{4 \pi G \rho_0}}{c_s}.
        \end{equation}
    
    Para este valor, la frecuencia de oscilación $\omega_0 = 0$. Tenemos dos escenarios distintos para $\omega_0$:
    
        \begin{itemize}
            \item $\omega_0^2 > 0$ (pertenece a los reales), el término de presión domina sobre la gravedad y la perturbación oscila como una onda de sonido. Lo mismo sucede cuando consideramos escalas pequeñas $k > k_J$. 
            \item $\omega_0^2 < 0$: El término gravitatorio domina sobre el de presión y las perturbaciones pueden crecer, o decrecer, exponencialmente. 
        \end{itemize}
    
    Para grandes escalas, $k < k_J$, $\omega_0$ es imaginaria y las perturbaciones crecen exponencialmente. 
    
    Si ahora analizamos un {\textbf{Universo en expansión, $H \neq 0$}}, compuesto de un fluido no relativista (materia), donde ahora (\ref{eq:2.6.17}) se transforma en
    
    
        \begin{equation} \label{eq:2.6.21}
            \boxed{ \ddot{\delta} + 2H \dot{\delta} = 4\pi G\rho_0 \, \delta. }
        \end{equation}
    
    Esta ecuación es fácilmente resuelta para $\Omega_m = 1$, donde consideramos que $H = \dot{a}/a = 2/3t$ y $a = k t^{2/3}$, así:
    
        \begin{equation} \label{eq:2.6.22}
            4 \pi G \rho_0 = \frac{3}{2} H^2 = \frac{2}{3} \frac{1}{t^2}.
        \end{equation}
    
    Proponemos un ansatz $\delta(t) \propto t^n$, tenemos: 
    
        \begin{align}
            \delta & = \delta_i t^n, \label{eq:2.6.23} \hspace{0.9cm} \text{y derivando} \\
            \dot{\delta} &= \frac{n}{t} \delta, \label{eq:2.6.24} \\
            \ddot{\delta} & = \frac{n(n-1)}{t^2} \, \delta^2. \label{eq:2.6.25}
        \end{align}
    
    Sustituyendo este conjunto de ecuaciones en (\ref{eq:2.6.21}), se tiene
    
        \begin{align}
            \frac{\delta}{t^2} \left[ n(n+1) + \frac{4}{3} - \frac{2}{3} \right] & = 0, \hspace{1.1cm} \textrm{de manera que:} \label{eq:2.6.26} \\ 
            \notag \\
            n^2 + \frac{1}{3}n - \frac{2}{3} & = 0, \hspace{0.8cm} n^2 - 1=0,\label{eq:2.6.27}
        \end{align}
    
    cuyas soluciones son $n = -1$ y $n = 2/3$. De esta manera, para un {\textbf{Universo dominado por materia}} se tiene $\delta \sim t^{2/3} \sim a$, correspondiente a la inestabilidad gravitatoria de las perturbaciones de densidad. Así, la perturbación crece como lo hace el factor de escala $a$. 
    
    Pensando ahora en el {\textbf{Universo temprano dominado por radiación}}, es común referirse a la teoría de perturbaciones de la relatividad general, la mecánica de fluidos de la relatividad especial y la gravedad Newtoniana con un término de fuente relativista (para más detalles consultar la sección 15.2 de Peacock, 1999). Aquí utilizaremos únicamente el resultado de este análisis, donde la ecuación de evolución para $\delta$ tiene un término ``impulsor'', un factor de $8/3$ mayor al de la época de materia. Así, (\ref{eq:2.6.21}) se escribe como: 
    
        \begin{equation} \label{eq:2.6.28}
            \boxed{ \ddot{\delta} + 2H \dot{\delta} = \frac{32}{3}\pi G\rho_0 \, \delta. }
        \end{equation}
    
    Para el caso en que $\Omega_m = 1$, la solución de ley de potencia da 
    
        \begin{equation} \label{eq:2.6.29}
            \delta \propto t  \hspace{0.9cm}  \textrm{o}  \hspace{0.9cm}  t^{-1}.
        \end{equation}
    
    Así, el modo de crecimiento durante la época de radiación $a \propto t^{1/2}$ tiene $\delta  \propto a(t)^2$. 
    
  \vspace{0.4cm} 
  
 % {\textbf{\large{Caso no Lineal}}
  
  
  \section{Oscilaciones Acústicas Bariónicas}
  
    Las oscilaciones acústicas bariónicas (BAO) son las reliquias congeladas del Universo antes de la época de desacoplamiento. Consideremos una región con sobredensidad de plasma primigenio. Tal región atrae gravitatoriamente la materia hacia ella, pero por otro lado, las interacciones materia - radiación en búsqueda del equilibrio térmico,  generan gran cantidad de presión hacia fuera. Esta ``lucha'' entre la fuerza gravitatoria y la presión crearon oscilaciones, un fenómeno similar al de las ondas de sonido creadas por diferentes presiones. 
    
     
     La cosmología requiere de una {\textit{regla estándar (standard ruler}}, i.e., un objeto de tamaño conocido en un único corrimiento al rojo $z$, o una población de objetos en diferentes $z$ cuyo tamaño cambie de una manera bien conocida con el $z$. Las BAO ahora forman parte de una familia de reglas estándar que entraron a la lucha inicialmente como un explicación del aparente agrupamiento alrededor de los $100 \, h^{-1}$ Mpc  (Eisenstein et al., 1998a; Meiksin et al., 1999). Siendo las BAO fluctuaciones periódicas en la densida de materia de los bariones en todo el Universo, puede ser usado para restringir parámetros cosmológicos como el contenido de energía oscura en el Universo. La idea de usar BAO para saber más acerca de los parámetros cosmológicos vino de la mano de Eisenstein et al. (1998b) quienes argumentaban que la detección de ondas acústicas en el espectro de potencia de materia confirmaría la historia y el paradigma de la inestabilidad gravitatoria. 
     
    {\textbf{Ruler Estándar Estadístico (RSS)}}
     
     La idea de los Ruler Estándar Estadísticos (RSS) es la siguiente: imagine que tiene una red tridimensional regular de espacio $L$ conocido, que hay galaxias y todas se ubican en las intersecciones de dicha red. Mediciones sobre las distancias de diámetro angular en función de $z$ y la tasa de expansión como una función también de $z$ podrían ser obtenidas. A continuación, empezamos a insertar galaxias aleatoriamente distribuidas sobre la red. A medida que agregamos más galaxias, el patrón de la cuadrícula se volverá más difícil de detectar a simple vista. A pesar de esto, con la transformada de Fourier este patrón podría ser detectado estadísticamente. 
     
     Para llegar al núcleo de la SSR, construyamos una distribución de galaxias. Insertamos una galaxia al azar, ahora con una probabilidad $p$ insertemos otra galaxia en cualquier dirección pero a una distancia $L$. Ahora usemos esta nueva galaxia como marco de referencia y sigamos agregando galaxias hasta que tengamos la cantidad que deseemos. En este punto ya no hay una red regular de galaxias pero $L$ sigue siendo la escala de longitud preferida en la distribución de galaxias y forma una SSR. BAO proporciona una elegante SSR la cual está oculta entre el resto de los agrupamientos de galaxias y tiene una ventaja adicional: principalmente, es un fenómeno físico lineal, lo que indica que podemos ignorar efectos no lineales (e.g., distorsiones en el espacio del corrimiento al rojo y agrupamiento gravitatorio no lineal, lo cual cambiará la naturaleza esférica de la escala de oscilación). La escala característica de BAO se establece por el horizonte de sonido en el desacoplamiento. 
     Usando CMB nosotros podemos fijar la escala de oscilación. Con sondeos de corrimiento al rojo se puede apreciar la escala de agrupamiento preferida establecida por BAO en diferentes $z$ y así restringir el parámetro de Hubble $H(z)$ y la distancia del diámetro angular. Esta escala es medida a lo largo (paralelo) y transversal (perpendicular) a la línea de visión (ver fig. \ref{fig:2.13}). El pico (o anillo) BAO aparece a una separación angular $\Delta \theta = [r_d/(1+z) DA]$ y a una separación de corrimiento al rojo $\Delta z =r_d/DH(z)$, donde $DA$ y $DH = c/H$ son las distancias angular y de Hubble y $r_d$ el horizonte de sonido en la época de ``arrastre'' (drag). La escala característica, $s_{\parallel}(z)$, a lo largo de la línea de visión nos da una medida del parámetro de Hubble a través de:
     
        \begin{equation} \label{eq:2.7.1} 
            H(z) = \frac{c \Delta z}{s_{\parallel}(z)},
        \end{equation}
     
     y el modo tangensial da una medida de la distancia del diámetro angular 
     
        \begin{equation} \label{eq:2.7.2} 
            d_A (z) = \frac{s_{\perp}}{\Delta \theta (1+z) }.
        \end{equation}
        
     La figura (\ref{fig:2.13a}) es la figura esquemática de lo dicho arriba, donde el eje $x$ viene siendo la ec. ({\ref{eq:2.7.1}) y el eje $y$ ({\ref{eq:2.7.2}). 
        
        %\begin{figure}
            %\centering
            %\includegraphics[width=0.30\linewidth]{bao_scale(2).png}
            %\caption{\footnotesize La longitud radial de un objeto es dado por $c dz/H(z)$ donde $dz$ es la diferencia en el corrimiento al rojo entre la parte de adelante y la trasera de un objeto. El tamaño transversal es $A(z) \theta$, siendo $\theta$ el tamaño angular del objeto. Con BAO uno puede determinar teóricamente el valor del diámetro $\theta$ y así obtener por separado $d A(z)$ y $H(z)$. (Fuente: Bassett \& Hlozek, 2009)}
           %\label{fig:2.13}
        %\end{figure}
        
        \begin{figure}[h]
		\centering
    	\begin{subfigure}{\textwidth}
    		\begin{subfigure}{0.3\textwidth}
    		    \includegraphics[width=\textwidth]{bao_scale(2).png}
        		\refstepcounter{subfigure}\label{fig:2.13a}
    		\end{subfigure}
    		\begin{subfigure}{0.7\textwidth}
    		    \includegraphics[width=\textwidth]{BAO_SR.png}
        		\refstepcounter{subfigure}\label{fig:2.13b}
    		\end{subfigure}    		
    	\end{subfigure}
		\caption{\footnotesize Izquierda: La longitud radial de un objeto es dado por $c dz/H(z)$ donde $dz$ es la diferencia en el corrimiento al rojo entre la parte de adelante y la trasera de un objeto. El tamaño transversal es $A(z) \theta$, siendo $\theta$ el tamaño angular del objeto. Con BAO uno puede determinar teóricamente el valor del diámetro $\theta$ y así obtener por separado $d A(z)$ y $H(z)$. Derecha: Huella de la onda de sonido congelada en el Universo temprano. BAO, un Ruler Estándar similar a Candela Estándar (Fuente de izquierda a derecha: Bassett \& Hlozek, 2009. Padmanabhan N., 2017) }
		\label{fig:nullinitial}
	\end{figure}
        
        \begin{figure}[h]
		\centering
    	\begin{subfigure}{\textwidth}
    		\begin{subfigure}{0.5\textwidth}
    		    \includegraphics[width=\textwidth]{bao_phenomen.png}
        		\refstepcounter{subfigure}\label{fig:2.14a}
    		\end{subfigure}
    		\begin{subfigure}{0.6\textwidth}
    		    \includegraphics[width=\textwidth]{bao_tecnique.png}
        		\refstepcounter{subfigure}\label{fig:2.14b}
    		\end{subfigure}    		
    	\end{subfigure}
		\caption{Izquierda: De fondo el CMB y cómo se traduce eso en el mapa de galaxias a $z$ menores. Derecha: Esquema de la medición de la escala de BAO. Ya qu esta escala es una huella congelada, puede ser usada como RRS. (Fuente de izquierda a derecha: \url{http://galaxies-cosmology-2015.wikidot.com/baryon-acoustic-oscillations} y ) }
		\label{fig:nullinitial}
	\end{figure}
        
     
     {\textbf{Física de LAS BAO}}
     
     Antes de la recombinación y el desacoplamiento, la temperatura disminuyó debido a la expansión. El Universo estaba compuesto de un plasma caliente de fotones y bariones que estaban estrechamente unidos a través de la {\textit{dispersión de Thomson}} y que debido a la disminución de la temperatura, dejaron de interaccionar formando hidrógeno neutro. La competencia entre la presión de radiación y la gravedad configuraron oscilaciones en el fluido de fotones. Si consideramos una sola perturbación de densidad esférica en el plasma fotón - barión estrechamente acoplado, se propagará hacia fuera como una onda acústica con una velocidad $c_s = c/\sqrt{3(1+R)}$, siendo $R = 3 \rho_b/4 \rho_{\gamma} \propto \Omega_b/(1+z)$ (Eisenstein et al., 1998b). En la recombinación, el cosmos se volvió neutro eliminándose así la presión sobre los bariones, las perturbaciones no se propagan ya en forma de ondas acústicas, mientras que los fotones se desacoplan y viajaron libremente formando el CMB. Los bariones y la DM interaccionan vía la gravedad, por lo que la DM también se agrupa en esta escala preferencial. Así, hay mayor probabilidad de que una galaxia se forme en algún lugar de los restos de mayor densidad de la onda bariónica. Este escenario es descrito por las figuras (\ref{fig:2.15}) y (\ref{fig:2.16}). 
     
     
   
        %\begin{figure}[H]
		%\centering
    	%\begin{subfigure}{\textwidth}
    		%\begin{subfigure}{0.55\textwidth}
    		    %\includegraphics[width=\textwidth]{bao_z_mayor.png}
        		%\refstepcounter{subfigure}\label{fig:2.13a}
    		%\end{subfigure}
    		%\begin{subfigure}{0.55\textwidth}
    		    %\includegraphics[width=\textwidth]{bao_z_1081.png}
        		%\refstepcounter{subfigure}\label{fig:2.13b}
    		%\end{subfigure}    		
    	%\end{subfigure}
		%\caption{\footnotesize \footnotesize Izquierda: Instantánea muestra el escenario incial en el proceso en el que BAO fue formado, a un corrimiento al rojo $z = 82507$, cuando el Universo tenías unos $\sin 110$ yrs. Electrones y protones (gas) estaban acoplados en el pozo de potencial gravitatorio de la DM. Derecha: Cuando las condiciones de tempueratura lo permitieron, los fotones se desacoplaron del plasma, dejando una huella de bariones y entonces viajaron libremente por el Universo. Algunos bariones cayeron en el pozo de potencial de la materia oscura y otros se quedaron congelados en la última oscilación. El radio de este patrón congelado se conoce como la {\textit{escala de BAO}}. (Fuente: Eisenstein et al., 2007). }
	%\end{figure}
    
        \begin{figure}[H]
            \centering
            \includegraphics[width=0.8\linewidth]{bao_z_mayor.png}
            \caption{\footnotesize Instantánea muestra el escenario incial en el proceso en el que BAO fue formado, a un corrimiento al rojo $z = 82507$, cuando el Universo tenías unos $\sin 110$ yrs. Electrones y protones (gas) estaban acoplados en el pozo de potencial gravitatorio de la DM  (Fuente: Eisenstein et al., 2007)}
           \label{fig:2.15}
        \end{figure}
    
    \newpage
    
    
    
        \begin{figure}[H]
            \centering
            \includegraphics[width=0.75\linewidth]{bao_DM_baryons_Bassett_pp15.png}
            \caption{\footnotesize \footnotesize Instantáneas de una perturbación de densidad esférica en evolución: perfil de masa radial en función del radio de movimiento para una sobredensidad inicialmente puntual ubicada en el origen. Las perturbaciones en la {\textbf{materia oscura}}, {\color{blue}{bariones}}, {\color{red}{fotones}} y {\color{green}{neutrinos}} y evolcuiona desde los primeros tiempos ($z = 6824$) hasta mucho después del desacoplamiento ($z = 10$). Inicialmente, la perturbación de densidad se propaga a través de los fotones y bariones como un solo pulso (panel superior izquierdo). El arrastre de los fotones y bariones sobre la materia oscura es visible en el panel superior derecho (previo a la recombinación); la materia oscura interacciona solo gravitatoriamente, y por lo tanto, su perturbación queda rezagada del plasma fuertemente acoplado. Durante la recombinación (panel medio izquierdo) los fotones se desacoplan, escapan de la perturbación bariónica. Una vez que se haya completado la recombinación (panel medio derecho), los fotones desaparecen libremente dejando solo una perturbación de densidad de bariones alrededor de los 150 Mpc, y una perturbación de materia oscura cerca del origen. Algunos de los bariones cayeron en el pozo potencial y otros se quedaron congelados en la última oscilación. El radio de este patrón congelado se conoce como la {\textit{escala BAO}}. El los dos páneles inferiores se observa cómo la interacción gravitatoria entre la materia oscura y los bariones afecta el pico: la materia oscura tira a los bariones cerca del radio cero, mientras que los bariones siguen arrastrando a la materia oscura hacia el pico en 150 Mpc (abajo a la izquierda), finalmente produciendo un pico en el perfil de masa radial de la materia oscura en la escala establecida por la distacia que la onda acústica barión - fotón podría haber recorrido en el tiempo antes de reacoplarse. El último panel en $z = 10$ muestra cómo, cuando se terminaron las edades oscuras y se formaron las primeras fuentes de luz (estrellas) (época de reionización), ya se formó la estructura BAO.  (Fuente: Eisenstein et al., 2007)}
            \label{fig:2.16}
        \end{figure}
    
    
    
    
    
    \newpage 
    
 \subsection{Sondeos de la Estructura del Universo a Gran Escala}
 
        \begin{wrapfigure}[17]{r}{0.3\linewidth}                               \includegraphics[width=0.45\textwidth]{mapa_galaxias_coil_pp3.png}
        \caption{\footnotesize \footnotesize{Suma de los ángulos de un triángulo en la superficie de una esfera}}
                \label{fig:2.1} 
        \end{wrapfigure}
    
    La LSS se define como la estructura del Universo en escalas mayores al tamaño de una galaxia. Edwin Hubble fue el primero desarrollar la idea de si las galaxias están distribuidas uniformemente en el espacio. Hubble usó su catálogo de 400 {\textit{nebulosas extragalácticas}} y las clasificó de acuerdo a su morfología, lo que se conoce como {\textit{secuencia de Hubble}} para probar esta idea (Hubble, 1926), encontrando que hay una uniformidad a grandes escalas. En una de estas nebulosas clasificada como ``espiral'' Hubble descubrió estrellas {\textit{cefeidas}} {\footnote{estrellas evolucionadas que ya han salido de la Secuencia Principal hacia la ``banda de inestabilidad'' de las Cefeidas. Además, estas estrellas emiten pulsos radiales con una relación bien definida entre su período y luminosidad, lo que las convierte en ideales para ser utilizadas como indicador de distancias primarias formando parte de las candelas estándar.}} y calculando la distancia que las separaba de nosotros, pudo  que estas estrellas estaban fuera de nuestra Galaxia. Observando estas galaxias a un corrimiento al rojo dado, Hubble descubrió que se están alejando unas de otras siguiendo lo que se conoce actualmente como la {\textit{ley de Hubble - Lema\^itre}} $v = Hr$, siendo $v$ la velocidad de recesión de la galaxia, $r$ la distancia que la separa del observador y $H$ la constante de Hubble. Ya que esto era observador en cualquier dirección, se concluyó que el Universo se está expandiendo. En 1934, Hubble (Hubble, 1934) haciendo uso de muestras estadísticas más grandes, observó que en escalas angulares menores a $10^{\circ}$ había un exceso en el número de conteos de galaxias por encima de lo que sería esperado para una distribución de Poisson aleatoria, aunque la muestra seguía una distribución Gaussiana a escalas mayores. 
    
    
    
    Entre otros sondeos, en 1977 Seldner y colaboradores (Seldner et al., 1977) publicaron mapas de conteos de galaxias en celdas angulares (ver fig. \ref{fig:2.17}) los cuales exhibían un patrón similar al de la espuma con posibles paredes y filamentos largos que contenían cúmulos y galaxias, además de grandes vacíos. Luego incia la era de los sondeos de corrimiento al rojo. Mapas sobre la distribución espacial tridimesional de 238 galaxias alrededor del supercúmulo ``Coma/Abell 1367'' fueron generados descurbiéndose que a menor $z$, existían grandes regiones sin galaxias, a las que se les denomina ``vacíos'' (Gregoy \& Thompson, 1978). 
    
    Otros sondeos fueron el KOS (Kirshner, Oemler, Schechter) el cual midió $z$ de 164 galaxias con un brillo mayor que 15 mag. y su finalidad era la de estudiar la distribución espacial tridimensional de las galaxias el cual arrojó un alto agrupamiento de las galaxias en el espacio de velocidades (Kirshner et al., 1978). También, el CfA ({\textit{Center for Atrophysics}}; Davis et al., 1978) que contenía datos del corrimiento al rojo de 2500 galaxias más brillantes que 14.5 mag y cuyo principal objetivo era el de cuantificar agrupaciones de galaxias en tres dimesiones. Hubo un segundo sondeo CfA entre 1985 y 1995, con espectros de $\sim 5800$ galaxias y donde fue detectado el supercúmulo ``Gran Muralla'' con una extensión de más de $170 \, h^{-1}$ Mpc así como también se observaron grandes vacíos de baja densidad con un $20 \, \%$ de la densidad media. 
    
    En las últimas dos décadas los avances para determinar cómo luce el Universo a gran escala ha dado sus frutos. Así, sondeos como el {\textit{Field Growth Redshift Survey 2 grados}} (2dFGRS) (Colless et al., 2001) y el {\textit{Sloan Digital Sky Survey}} (SDSS) (York et al., 2000) han dado información valiosa sobre el Universo local, considerando la distribución de gas intergaláctico y de DM. Los avances tecnológicos además de un entendimiento más claro del Universo han permitido que, através de diferentes técnicas, el mapeo tridimensional de estructuras usando su posición en el cielo y corrimiento al rojo sea cada vez más preciso y con esto también el cálculo de la constante de Hubble
    
        \begin{equation}
            H = H_0 \left[ \Omega_{r_0} \left( \frac{a}{a_0}\right)^{-4} + \Omega_{m_0} \left(\frac{a}{a_0}\right)^{-3} + \Omega_{\Lambda_0} \left(\frac{a}{a_0}\right)^{-2} - (1 - \Omega_{T_0} \right]^{1/2}.
       \end{equation}
    
    Actualmente, son varios los sondeos cuya misión es la de obtener y reducir datos de objetos a $z$ cada vez más altos para así ser capaces de generar mapas que cubran mayor extensión y profundidad del cielo. Algunos de ellos son el {\textit{Dark Energy Survey}}, el ya mencionado SDSS, y {\textit{Dark Energy Spectroscopic Instrument}} (DESI). 
	
    \subsubsection{SDSS}
    
    El {\textit{Sloan Digital Sky Survey}} (SDSS, York et al., 2000) es un sondeo de corrimiento al rojo que inició sus observaciones en el año 1998 y ya ha completado varias fases diferentes: SDSS-I (2000-2005), SDSS-II (2005-2008) SDSS-III (2008-2014) y SDSS-IV que inició en 2014 hasta el presente. Cada una de estas fases ha involucrado objetivos científicos relacionados. Algunos datos obtenidos incluyen imágenes profundas multicolores que cubren más de un cuarto del cielo y ha creado mapas en tres dimensiones que contienen más de 930 mil galaxias y 120 mil QSO, así como también espectroscopía óptica y del infrarrojo cercano de más de 3.5 millones de estrellas, galaxias y QSO. Todos estos datos han sido obtenidos con el telescopio de 2.5 m: {\textit{Sloan Foundation Telescope}} en {\textit{Apache Point Observatory}} (APO, Nuevo México; Gunn et al., 2006), el cual viene equipado con una cámara CCD de mosaico de gran formato para obtener imágenes en cinco bandas ópticas $u, g, r, i$ y $z$ en noches con buenas condiciones para observar y sin Luna, además de dos espectrógrafos que usan fibra óptica para obtener el espectro de alrededor de 1 millón de galaxias y 100 mil QSO de datos de imágenes. 
    
    El {\textbf{SDSS-I}} comenzó el {\textit{SDSS Legacy Survey}} entre abril del 2000 y junio de 2005, creando imágenes del cielo en las bandas $u, g, r, i$ y $z$ y utilizando la cámara de imágenes SDSS. Observó galaxias y QSO, 129 de ellos usando par de espectrógrafos de fibra óptica. Las galaxias se divierdon en dos grupos: una muestra principal de flujo limitado con un $z$ medio de $z \sim 0,1$  (Strauss et al., 2002) y una muestra seleccinada por color de Galaxias Rojas Luminosas (LRG) que se extendió a un $z \sim 0.5$ (Eisenstein et al. 2001). La muestra de QSO incluyo QSO tanto de exceso ultravioleta hasta $z \sim 2$ y un conjuto de QSO de alto corrimiento al rojo más allá de $z = 5$ (Richards et al. 2002). 
    
    Luego vino el {\textbf{SDSS-II}} entre julio de 2005 hasta junio de 2008, que completó el {\textit{Legacy Survey}} con 1.3 millones de espectros, cubrió una gran área contigua en la {\textit{Northern Galactic Cap}} (NGC) y tres franjas delgadas y y largas en la {\textit{Southerm Galactic Cap}} (SGC). SDSS-II también llevó a cabo dos programas: El {\textit{Sloan Extension for Galactic Understanding and Exploration 1}} (SEGUE-1, Yanny et al., 2009) donde obtuvo imágenes de sobre un gran rango de latitudes galácticas y espectros de 240 mil estrellas sobre un rango de tipo espectral para así estudiar la estructura de nuestra Galaxia. El segundo programa fue el {\textit{Sloan Digital Sky Survey II Supernova Survey}} (Frieman et al. 2008; Sako et al. 2014) que catalogó más de 10 mil fuentes, entre esas 1400 SN1a. Ambos programas utilizaron principalmente el tiempo oscuro. 
    
    El {\textbf{SDSS-III}} junto con el SDSS-IV persiguen propósitos cosmológicos. SDSS-III está compuesto por cuatro sondeos: el {\textit{Sloan Extension for Galactic Understanding and Exploration }} (SEGUE-2), el cual espectroscópicamente observó alrededor de 119 mil estrellas, enfocándose en el halo estelar de la Galaxia con distancia de 10 a 60 kpc. SEGUE-2 se enfoca en estudiar halos más distantes con la finalidad de poder entender entonces el crecimiento de nuestro halo Galáctico en el tiempo. El segundo sondeo es el {\textit{APO Galactic Evolution Experiment }} (APOGEE-1) que usó espectroscopía infrarroja para poder penetrar en el polvo que oscurece extensas partes del disco y bulgo de la Galaxia. La idea es que sondeando unas 100 mil estrellas gigantes rojas dispuestas en el halo, bulgo, barra y disco de la Galaxia, se puedieron precisar velocidades radiales y dando detalles de la abundancia química lo cual generaró una mejor idea de todo lo referente a la historia química y estructura dinámica de la Vía Láctea. El tercer sondeo es el {\textit{Multi-object APO Radial Velocity Exoplanet Large-area Survey }} (MARVELS) destinado ya al estudio de velocidades radiales de 11 mil estrellas para así detectar planetas gigantes gaseosos fuera de nuestro Sistema Solar. MARVELS proporcionó un conjunto de datos para probar modelos teóricos sobre formación migración y evolución dinámica de los gigantes gaseosos. 
    \medskip
    
    {\textbf{BOSS}}
    
    Nosotros vamos a hacer énfasis en el cuarto sondeo, el {\textit{Baryon Oscillation Spectroscopic Survey }} (BOSS) enfocado en el estudio de la energía oscura y la geometría del espacio. Para esto, BOSS ha mapeado la distribución espacial de LRG y QSO para detectar la escala característica (150 Mpc) generada por las BAO y que dejaron su huella impresa en las fluctuaciones del CBM, evolucionando en filamentos y vacíos en la actualidad. Las BAO han sido medidas en la distribución de galaxias. Usando la escala acústica como una regla calibrada físicamente, BOSS ha sido capaz de medir la distacia del diámetro angular con una precisión del $1 \, \%$ y en un $z= 0.3$ y $z=0.5$  usando la distribución de galaxias. Otras medidas realizadas por el sondeo han sido la distribución de líneas de absorción de QSO en $z=2.5$, lo que a su vez condujo a la medición de la distancia del diámetro angular y la tasa de expansión  $H(z)$ ambas en ese mismo corrimiento al rojo. Estas mediciones proporcionan pruebas para el estudio de la energía oscura así como del origen de la aceleración. Por otra parte, BOSS también ha colaborado con muestras de galaxias y QSO ideales para estudiar la formación y evolución de galaxias en el Universo. 
   
    La fase {\textbf{SDSS-IV}}, está compuesta por tres sondeos, el primero de ellos es {\textit{APO Galactic Evolution Experiment-2}} (APOGEE-2) donde, someramente,  su objetivo es el de proporcionar un mejor entendimiento sobre la historia de La Vía Láctea y de la astrofísica de estrellas; segundo tenemos el {\textit{Mapping Nearby Galaxies at APO}} cuyo objetivo es el de tener un mejor entendimiento sobre la historia evolutiva de galaxias y lo que regula la formación de estas. 
    \medskip
    
    {\textbf{eBOSS}}
    
    El {\textit{Extended Baryon Oscillation Spectroscopic Survey }} (eBOSS; Dawson et al., 2016), es una extesión del BOSS, con objetivos como el mejor entendimiento de la DM, energía oscura, propiedades de los neutrinos y la inflación. Precisamente medirá la historia de expansión del Universo temprano, desde que tenía menos de 3mil millones de años para así btener restricciones en la naturaleza de la energía oscura, uno de los resultados experimentales más misteriosos de la física moderna. eBOSS Trabaja en un régimen de desplazamiento al rojo entre $0.6 < z < 2.2$. El uso de espectroscopía le permite alcanzar galaxias, e.g., LRG y Galaxias de Línea de Emisión (ELG), entre un rango de $0.6 < z < 1.1$ y QSO con $z > 0.9$ lo que permite estudiar la expansión del Universo utilizando BAO y el crecimiento de las estructuras utilizando las distorsiones espaciales del desplazamiento al rojo a gran escala. De esta manera, el sondeo está alcanzando distancias inexploradas nunca antes por otros mapas tridimensionales de estructura a gran escala, lo que se traduce en un volumen mucho más extenso. Esa región corresponde a la época en la que el Universo estaba en transición de desaceleración debido a los efectos de gravedad, a la actual época de expansión acelerada ($0.6 < z < 3$). Las observaciones con eBOSS han arrojado datos más precisos, del $1 \, \%$ en $z=0.7$ para LRG, $2 \, \%$ en $0.85$ para ELG y $2 \, \%$ en $z =1.5$ y en 1.4 mejoró la medición de BAO por el Bosque de Lyman-$\alpha$ en 120 mil espectros nuevos, así como las mediciones de las distorsiones del espacio de corrimiento al rojo (RSD) y la no-Gaussianidad del campo de densidad primordial. La figura (\ref{fig:2.18}) muestra el alcance que tendrá tanto BOSS como eBOSS a diferentes corrimientos al rojo. 
    
        \begin{figure}[H]
            \centering
            \includegraphics[width=0.6\linewidth]{eboss_coverage.png}
            \caption{\footnotesize Distribución del volumen cubierto por BOSS y eBOSS en diferentes corrimientos al rojo. (Fuente: https://www.sdss.org/surveys/eboss/)}
           \label{fig:2.18}
        \end{figure}
        
    \subsubsection{DESI}
    
    El {\textit{Dark Energy Spectroscopic Instrument}} (DESI) tiene como principal objetivo estudiar la naturaleza de la energía oscura, cómo es la dinámica de la densidad de energía y cómo afecta la manera en que la materia se agrupa. Dos efectos cosmológicos serán medidos: las BAO y las RSD a través de mapas que darán la oportunidad de entender más a profundidad la cosmología y física de las galaxias, QSO y gas intergaláctico. Estos mapas tridimensionales serán creados a partir de imágenes de un tercio del cielo nocturno, cubriran un enorme volumen y un amplio rango de corrimientos al rojo y será posible entonces seleccionar galaxias a partir de sus espectros. 
    
    DESI es un instrumento terrestre que se encontrará en el Telescopio Mayall de 4 metros en el Observatorio Nacional de Kitt Peak, ubicado al suroeste de Tucson, Arizona. El telescopio Mayall es un telescopio reflector con un espejo primario de 4 metros que se asienta sobre una montura ecuatorial. Es el más grande de los 22 telescopios ubicados en Kitt Peak. Sucesor de eBOSS, se superpondrá con el {\textit{Dark Energy Survey}} (DES) y el {\textit{Large Synoptic Survey Telescope}} (LSST). Está diseñado para tomar espectros controlados robóticamente de hasta 5 mil espectros simultáneamente, y se espera que observe en longitudes de onda de 360 nm a 980 nm por fibras que alimentan 10 espectrógrafos de tres brazos durante un período de cinco años, cubriendo así 14 mil $deg^2$.
    
    El catálogo de DESI estará basado en 4 tipos de galaxias principalmente que suman un total de 24 millones de objetos, con $z = 0.4 \rightarrow{3.5}$. Tenemos: 
    
        \begin{enumerate}
            \item Galaxias Brillantes (BG): Debido a su alto brillo, son objetos muy fáciles de identificar lo que permite entender el Universo más tardío así como también el más cercano en la actualidad, la era de la expansión acelerada. DESI creará mapas de estos objetos hasta un corrimiento al rojo de 0.4 y una magnitud límite de 20. 
            
            \item Galaxias Luminosas Rojas (LRG): Son galaxias masivas y su población estelar es mayormente vieja. Debido a su color rojo son fáciles de seleccionar en muestros. Con estos objetos DESI alcanzará un corrimiento al rojo de 1. 
            
            \item Galaxias de Línea de Emisión (ELG): son galaxias con una fuerte formación estelar y con población de estrellas jóvenes, por lo que producen líneas de emisión de longitud de onda muy distinguible. Son la muestra más grande que usará DESI, y alcanzará un $z =1.6$. Mayor interés en galaxias que producen la línea de emisión [OII].
            
            \item Cuásares (QSO): Gracias a su alta luminosidad, DESI podrá tener un mayor alcance, con un corrimiento al rojo de 3.5 y un poco más. Estos objetos son galaxias que albergan un SMBH en su centro, con un disco que acreta grandes cantidades de gas que brilla al alcanzar velocidades relativistas. La importancia de incluir QSO es debido a que ellos son trazadores de nubes de hidrógeno neutro, lo que a su vez los hace trazadores de la distribución de DM vía el Bosque Lyman-$\alpha$, detectado a un $z>2.1$, y el cual profundizaremos en la siguiente sección. 
            
            Adicionalmente, DESI incluirá observaciones de estrellas de nuestra Galaxia aportando información de su composición química y de la cinemática, permitiendo un mejor entendimiento del rol que juega la DM en nuestra vecindad. 
        \end{enumerate}
        
        
        \begin{figure}[H]
            \centering
            \includegraphics[width=0.75\linewidth]{DESI_2.png}
            \caption{\footnotesize Profundidad que alcanzará DESI y distribución de los $\sim 24$ millones de objetos a considerar. Como indica la figura, la región interna blanca representa los 4 millones de GLR, la zona púrpura son los 18 millones de ELG, la parte verde son las 0.7 millones de QSO detectados por el Bosque Lyman-$\alpha$, y por último se tiene la zona entre la región verde y púrpura, donde descansan los 1.7 millones de QSO trazadores.  (Fuente: P. McDonald, 2014, Talk)}
           \label{fig:2.19}
        \end{figure}
        

%%%%%%%%%%%%%%%%%%%%%%%%%%%%%%%%%%%%%%%%%%%%%%%%%%%%%%%%%%

 \chapter{Cosmología con el Bosque Lyman-$\alpha$}
 
    En galaxias normales, la luz observada en el óptico e infrarrojo cercano proviene principalmente de estrellas además de algunas pequeñas contribuciones de polvo y gas. Las estrellas pueden ser bien descritas por el Espectro de Planck, en el cual la temperatura juega un rol significativo y esta, a su vez, va a depender de la masa y etapa evolutiva de la estrella. Pocas estrellas tienen altas temperaturas que pueden superar los $40.000$ K, mientras que la
    mayoría tienen temperaturas menores a los  $3.000$ K que, debido a su baja luminosidad, poco contribuyen al espectro de una galaxia, el cual puede aproximarse como una superposición de espectros estelares. La {\textit{distribución de energía de un espectro}} (SED) de Planck es estrecha alrededor de su máximo ($h \nu \sim 3 \, K_B T$) por lo que el espectro de una galaxia se confina a rangos entre ($\sim 4.000 - \sim 20.000 $) \AA, pero, si la galaxia tiene regiones de formación estelar activas, este rango se amplía a frecuencias más altas, extendiendo la emisión hasta el infrarrojo (IR) lejano. Por otro lado, también existen algunas galaxias cuya distribución de energía es más amplia todavía, mostrando rangos de emisión que se extienden incluyendo longitudes de onda de radio, Rayos X, e incluso, Rayos Gamma, la cual proviene de una región central muy pequeña. De esta manera, se tiene una galaxia activa y de esa región central deriva el nombre de {\textit{Núcleos Activos de Galaxias}}  (AGN). Estas galaxias se consideran bastante normales excepto por su intensa actividad nuclear. 
    
    
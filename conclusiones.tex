\chapter{Conclusión}

	El estudio de la sincronización representa un desafío fascinante y diverso,  capaz de proveer herramientas útiles  a cualquier campo de la ciencia en donde éste se manifieste; desde la biología y ecología pasando por la física y la química hasta la sociología y las artes. Nuestro mundo experimenta día a día distintas maneras y formas de interacción que permiten la aparición de sincronías gracias a las conexiones espontáneas de elementos, que tal vez sean imperceptibles pero son suficientes, como para generar un comportamiento de “coincidencia” o “simultaneidad” de diversas maneras.
	
	En particular, el fenómeno de sincronización esta presente en el cerebro humano; que es un órgano estructura compleja, compuesto por una infinidad de neuronas que mediante su interacción cumple con sus tareas perfectamente ejecutadas, en un sin fin de funciones que necesita un organismo vivo para su óptimo rendimiento y plenitud de sus capacidades. En \'el la sincronización se encuentra presente como recurso vital en las funciones del sistema nervioso central surgiendo en muchas de las habilidades indispensables que requiere el ser humano para vivir en virtuosismo y comodidad. 
	
	Por otra parte, la manifestación de sincronización en las redes neuronales del cerebro también se relaciona con algunas disfunciones de los procesos de desarrollo cognitivo y motor como en la epilepsia, la esquizofrenia y el p\'arkinson, entre otras patologías neurológicas.

	En este trabajo se estudió la dinámica de una red de neuronas mediante la construcción de una red de mapas caóticos acoplados donde cada neurona est\'a representada por un oscilador caótico y excitable, propuesto por Chialvo, mientras que las interconexiones siguen una dinámica de pesos Hebbiana, controlada mediante la variación de los parámetros de plasticidad $\delta$ y la fuerza de acoplamiento entre los elementos $\epsilon$. Con el modelo se encontró que el estado de sincronización de la red est\'a fuertemente ligado al acoplamiento que existe entre sus elementos mientras que depende en menor proporción de la plasticidad que la red neuronal presente.
	
	Para estudiar el efecto que tiene un campo externo sobre el sistema sincronizado, se sometió a una red neuronal a una perturbación siguiendo cuatro estrategias en primer lugar se aplicó una perturbación de magnitud constante a todos los elementos de la red en cada iteraci\'on, y se determin\'o que es posible desincronizar los elementos de la red para ciertos valores de la magnitud de la perturbación, ver figura \ref{fig:TS}, manteniendo en las neuronas su característica de ser caóticas y excitables. 
	
	Aunque con la primera estrategia fue suficiente para lograr el objetivo de desincronizar la red, \'esta requiere de la aplicación de una perturbación relativamente grande, a todas las neuronas y de manera constante; lo cual implicaría en la vida real a someter al cerebro a un choque eléctrico en todo momento. Mediante las otras tres estrategias se busca intervenir menos en la red neuronal se propuso reducir la magnitud de la perturbación, su frecuencia y el n\'umero de elementos a los cuales se les aplica.
	
	Como nuestra segunda estrategia se sometió a la red de neuronal a una perturbación que afecta a todos los elementos pero ahora con cierta frecuencia y, aunque no se obtuvieron resultados significativos con respecto a la reducción del valor de la magnitud de la perturbación, se demuestra que es posible desincronizar la red sin la necesidad de aplicar la perturbación en todo momento, como lo muestra la figura \ref{fig:TV}, lo cual resulto ser una mejora considerable con respecto a la anterior estrategia.
	
	 Buscando de reducir la cantidad de elementos a los cuales se perturban, la tercera estrategia consiste en seleccionar con cierta probabilidad en cada iteraci\'on a los elementos que son sometidos a la perturbación. En este caso se encontró que es posible desincronizar la red sin necesidad de afectar a todos los elementos de la red, logrando también una reducción del valor de la magnitud de la perturbación, como se ve en la figura \ref{fig:AS}. Finalmente combinando la segunda y la tercera estrategia, resulto en una cuarta y \'ultima estrategia con la cual se logra de manera efectiva desincronizar la red, como se ve en la figura \ref{fig:AV} ,afectando a algunas neuronas a veces, siendo así la que resulta ser la menos invasiva.
	 
	La aplicación de perturbaciones externas a una red neuronal, siguiendo las estrategias planteadas en este trabajo, demostró ser capaz de afectar y mantener desincronizados los elementos de la red que espontáneamente se encuentra en fase sincronizada. De esta manera podemos decir que este planteamiento teórico puede ayudar en el diseño de un marcapasos cerebral que eviten el estado de sincronización relacionado a alguna patología neurológica como lo son la epilepsia, la esquizofrenia y el p\'arkinson.